\chapter{Design \& Specification}
  
% Provide abstract level of how I intend to compare them
% How do I make sure the implementation is like for like
% 

Here, I will provide an abstract level of how I compared the performance of each content generation algorithm and how I made sure each implementation could produce as similar/like-for-like results as possible (and where they \textit{couldn't} do so).

\section{Performance}

With the L-System implementation, I had no problems running the game very quickly on my machine, and quickly got results.

With Poisson Disk Sampling, the higher the number of rejection samples (that is, the higher the maximum number of times a cell was sampled before it was either accepted or ultimately rejected), the longer it took to generate a complete level layout, and even, due to the nature of the tile map compared to the algorithm's \textit{usual} use (of scattering dots on a plane), it was not maximal (not all points had cells painted for them; some cells had their tiles overwritten as well). Using 8 rejection samples was usually enough to yield a satisfactroy level layout.

Voronoï Cells took the longest to compute on average. 

\section{Layouts}

Of the 4 implementations I made, the Noise and Poisson Disk Sampling implementation were by far the most similar, followed by the L-System implementation, and then the Voronoï Cells implementation, which was far and away the most unique.

While the noise implementations were 