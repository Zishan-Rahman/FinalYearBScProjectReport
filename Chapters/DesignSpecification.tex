\chapter{Design \& Specification} \label{Design}
  
% Provide abstract level of how I intend to compare them
% How do I make sure the implementation is like for like
% 

Here, I will provide an abstract level of how the performance of each content generation algorithm was compared and how it was ensured that each implementation could produce as similar/like-for-like results as possible (and where they \textit{couldn't} do so). More specifics and quantified data are delved into in Chapter \ref{Evaluation}.

\section{What I Was Looking For In The ``Best" Implementation Of My Scenario}

To help determine what was wanted from these implementations and how similar they should be, it was decided, after repeatedly trying out the implementations as they were continually worked on and finessed, that they should adhere to the following \textit{specification}, which best fit the chosen scenario:

\begin{itemize}
    \item They should produce levels that are able to be completed every time; that is, the player character should be able to collect the ring and win the game every single time.
    \begin{itemize}
        \item This was part of the motivation between the idea that there should be free space to move around from the outset.
    \end{itemize}
    \item There should ideally be an ample amount of free space to move around.
    \begin{itemize}
        \item While the golem can ``eat" trees to gradually create new free space over time, it was highly desired for the levels to \textit{start} off relatively spatial in this regard, while also still containing a very highly ample amount of trees and buildings within the boundaries of the levels.
    \end{itemize}
    \item The golem and ring should be spawned separately from the generated environment.
    \begin{itemize}
        \item By \textit{always} placing the tree and golem in an empty space that \textit{was not} always the top-right corner (or spawning at a tree for the Voronoi cells implementation where all cells are occupied), it would ensure that the player can start and win the game by roaming on empty space and consuming trees on the way to collect the ring.
        \item This was \textit{also} part of the motivation behind having free space from the outset.
    \end{itemize}
    \item The trees and buildings should be scattered around \textit{with purpose} (that is, they should \textit{look} very random while also being highly calculated behind the scenes), such that it should be quite hard, but not necessarily impossible, to discern which algorithm was behind which level.
    \begin{itemize}
        \item In the final implementations, like in Figure \ref{fig:simplexsmoothdefault1}, the levels they generate should \textit{seem} relatively similar to one another, albeit not necessarily identical.
    \end{itemize}
    \item Levels should be generated fairly quickly while still adhering to the other principles in my specification.
    \begin{itemize}
        \item While each algorithm did not need to be the \textit{fastest}, necessarily, any processing times above 500ms would not be advisable.
    \end{itemize}
\end{itemize}

\section{Comparing Layouts-Wise}

Comparing the layouts involved trying out the implementations throughout their continuous development for this project. This involved tweaking values and calculations around. For instance, the Voronoi cells implementation involved the use of two different distance calculations that produced wildly differing cells and cell shapes. The ``FastNoiseLite" class in Godot includes various configurable properties for noise images, such as the frequency of the noise, the fractal type used to alter the generated noise and even the noise 

\section{Comparing Performance-Wise}

Comparing each implementation performance-wise was far easier than comparing them layout-wise, since this did not rely on assessing layouts by eye. Checking the quantified performance time for all algorithms involved both a print statement in the ``\_ready" function \textbf{and} the processing time showing in the opening story dialog, the text of which was \textit{also} set in the ``\_ready" function.

Calculating the processing time in the ``\_ready" function for all implementations involves:

\begin{enumerate}
    \item The calculations and processing of the algorithm itself.
    \item The painting of cells using results returned by the algorithm.
    \item The random placement of the golem.
    \item The random placement of the ring.
\end{enumerate}

