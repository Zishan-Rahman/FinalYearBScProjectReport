\chapter{Results \& Evaluation} \label{Evaluation}

Here, we will discuss how the implementations of the algorithms in our scenario were tested and ensured that they ran as they should.

\section{Software Testing}

Due to the nature of the project (being several implementations of a computer game), the testing behind this project has solely revolved around trial-and-error, messing around with the exported variables in the Godot editor to see how things worked and what configurations worked best for our scenario. This involved taking many screenshots of generated levels and examining things by eye, seeing how layouts compared across implementations.

Despite this, it was eventually decided to run some simple performance tests to see how long each algorithm ran. These tests took some of the custom export variables from the scene scripts and ran them several times, with an average time calculated to the nearest millisecond. The results of these tests are all in table in the \hyperref[Appendix]{Appendix}.

\section{Comparing the Different Algorithms and Drawing Conclusions on Which Ones Are Best}

\subsection{Performance} \label{performance}

With the L-System implementation, there were no problems whatsoever running the game very quickly on the author of this report's machine, and quickly got satisfactory results. The table in Figure \ref{fig:table3} shows that the processing times remained miniscule, even as the maximum length of the axiom increased. With the default values for all export variables, it took an average of 16ms for the L-System implementation to generate levels, by far the fastest out of all the implementations that were created for this report, as shown in table \ref{fig:table6}.

While the Noise implementation was slower than the L-System implementation by a magnitude, it was still satisfactorily quick. Some timed tests were run in which this paper's author tested how some of the properties affected the time it took to generate the noise. These tests are referred to in tables \ref{fig:table1}, in which each noise algorithm was paired with each cellular distance function, and \ref{fig:table2}, in which each noise algorithm was paired with each fractal type. With the default values set \hyperref[noisedefaults]{here}, it was found that it took an average of 81ms, as shown in table \ref{fig:table6}.

With Poisson Disk Sampling, the higher the number of rejection samples (that is, the higher the maximum number of times a cell was sampled before it was either accepted or ultimately rejected), the longer it took to generate a complete level layout, and even then, due to the nature of the tile map compared to the algorithm's \textit{usual} use (of scattering dots on a plane), it was not maximal (not all points had cells painted for them; some cells had their tiles overwritten as well). Using 8 rejection samples was usually enough to yield a satisfactory level layout while also keeping level creation times to a satisfactory minimum. It took an average of 268ms to work with 8 rejection samples in the tests in table \ref{fig:table4}, and 197ms in some additional tests done in table \ref{fig:table6}. If the rejection samples were set to a too high value, there was a high  chance that the game would hang and not return any cell points at all, because it just took \textit{way} too long to process. However, that does not always happen; as referenced in the caption of table \ref{fig:table4}, on one occasion, when the value of rejection samples was set to 18, the game \textit{did} stall, and had to be restarted again so that enough processing times could be recorded.

Voronoï Cells took the longest to compute on average. Computations with the Euclidean distance measurement took longer than those measured with the Manhattan distance, and the number of random starting points (and therefore the number of unique Voronoï cells in a single tesselation) increased level generation times as well. Both of those results are solidly proven in table \ref{fig:table5}, as well as table \ref{fig:table6}, in which, even \textit{with} the default values set, it \textit{still} took 455ms on average, far longer than any of the other implementations.

\subsection{Layouts}

Of the 4 implementations that were made for this project, the Noise and Poisson Disk Sampling implementation were by far the most similar, followed by the L-System implementation, and then the Voronoï Cells implementation, which was far and away the most unique.

While the noise implementations varied greatly depending on what settings were used, and the way the implementation was designed allowed for very many possibilities as to how the noise would turn out (and how it would affect the final level), the results that were returned produced the most similar results to that of the Poisson Disk Sampling implementation had the following configurations:

\begin{itemize} \label{noisedefaults}
    \item Noise Type (``noise\_type"): Simplex Smooth
    \item Fractal Type (``fractal\_type"): Fractal None
    \item Cellular Distance Type/Function (``cellular\_distance\_type"): Distance Euclidean
    \item Noise Frequency (``noise\_frequency"): 0.894
    \item Tree Cap (``tree\_cap"): -0.048
    \item Building Cap (``building\_cap"): -0.252
    \item Building Overtakes Tree (``building\_overtakes\_tree"): 0.12
\end{itemize}

The default noise frequency in ``FastNoiseLite" is 0.01, which results in smoother and less disparate noise. As seen in Figure \ref{fig:simplexsmooth0.01}, the smoother noise and lower frequency results in a distinct kind of level layout in which represents some of the noise values in the image very clearly, such that tiles (both buildings and trees) are bunched together in partially interconnected groups, forming long, large lines of painted tiles. To describe this as best as possible, it is easy to discern that the level layout was determined from a noise image. Using a higher noise frequency to produce rougher noise, and more disparate level layouts, yields results like in Figure \ref{fig:simplexsmoothdefault1}, which makes it very similar to the layouts yielded in the Poisson Disk Sampling implementation and, to a lesser extent, the L-System implementation. While the author of this report's personal tastes are fond of the former kind of level layout, part of the aim of this project was to compare in terms of which could produce the most similar, and, compared to the L-System and Poisson Disk Sampling implementations, the Noise implementation with the frequency set to 0.01 was far too distinct, hence the want to change it up.

\begin{figure}[H]
    \centering
    \includegraphics[width=0.8\textwidth]{Images/simplex_smooth_default_1.png}
    \caption{A level of our scenario generated in the Simplex Noise implementation, using all of the default values \hyperref[noisedefaults]{shown here}. The level took a total of 99 milliseconds to be made.}
    \label{fig:simplexsmoothdefault1}
\end{figure}

\begin{figure}[H]
    \centering
    \includegraphics[width=0.8\textwidth]{Images/simplex_smooth_0.01_frequency.png}
    \caption{A level of our scenario generated in the Simplex Noise implementation, setting the noise frequency to 0.01 (the default value for noise frequency in ``FastNoiseLite") and using the rest of the defaults \hyperref[noisedefaults]{shown here}. The level took a total of 104 milliseconds to be made.}
    \label{fig:simplexsmooth0.01}
\end{figure}

The Poisson Disk Sampling implementation also produced similar levels, with the main difference higher level processing times, as shown in section \ref{performance}. The lower the number of rejection samples, the quicker the processing times, but the lower the number of tiles painted in the tile map. This is shown in Figures \ref{fig:poisson2} and \ref{fig:poisson3}, with the latter being a particularly egregious example. Figure \ref{fig:poisson1}, on the other hand, shows a level created with the default number of rejection samples (8), and is similar to the level generated with Simplex noise in Figure \ref{fig:simplexsmoothdefault1}.

\newpage

I also have 3 other export variables:

\begin{itemize}
    \item ``point\_radius", which sets the distance between points during calculation, in that no point can be within a radius distance of other points. By default it is set to 1.0. The higher the value, the more spaced apart painted tiles are, as seen in Figure \ref{fig:poisson4}.
    \item ``paint\_building\_probability", which determines whether or not to print a building tile in lieu of a tree tile at a cell (overwriting any existing tile if there is any at that cell). By default it is set to 0.125, the higher the value (between 0.0 and 1.0 inclusive), the more likely a building is to be painted and the more buildings that will appear in the final level, as seen in Figure \ref{fig:poisson5}.
    \item ``region\_size", which, by default, is set to the current tile map size (72, 40), although that does not show properly in the editor. These values can be changed for a smaller region size, which can result in faster processing times, but this is not best advised for our chosen scenario, due to the fact that not all cells in the current tile map will be covered this way.
\end{itemize}

\begin{figure}[H]
    \centering
    \includegraphics[width=0.8\textwidth]{Images/default-poisson-result.png}
    \caption{A level layout set with the default number of rejection samples (8). This level took 222ms to create.}
    \label{fig:poisson1}
\end{figure}

\begin{figure}[H]
    \centering
    \includegraphics[width=0.8\textwidth]{Images/poisson-3-samples.png}
    \caption{A level layout set with the number of rejection samples set to 3 instead of 8. There are a somewhat smaller number of painted tiles in this level than the level in Figure \ref{fig:poisson1}. This level took 87ms to create.}
    \label{fig:poisson2}
\end{figure}

\begin{figure}[H]
    \centering
    \includegraphics[width=0.8\textwidth]{Images/poisson-1-sample.png}
    \caption{A level layout set with the number of rejection samples set to 1 instead of 8. This level barely has any cells painted on it, certainly far less than the levels shown in Figures \ref{fig:poisson1} and \ref{fig:poisson2}. This level took 3ms to create.}
    \label{fig:poisson3}
\end{figure}

\begin{figure}[H]
    \centering
    \includegraphics[width=0.8\textwidth]{Images/poisson-radius-1.564.png}
    \caption{A level layout set with the radius (``point\_radius") to 1.564 instead of the default 1.0. As you can see, this results in further spaced-apart cells and more empty space. This level took 135ms to create.}
    \label{fig:poisson4}
\end{figure}

\begin{figure}[H]
    \centering
    \includegraphics[width=0.8\textwidth]{Images/poisson-buildings-0.5.png}
    \caption{A level layout set with the probability of buildings being painted over trees set to 0.5 instead of the default 0.125. As you can see, this results in more building tiles being painted in the tile map than usual. This level took 293ms to create.}
    \label{fig:poisson5}
\end{figure}