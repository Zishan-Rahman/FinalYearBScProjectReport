\chapter{Legal, Social, Ethical and Professional Issues} \label{Issues}
% Your report should include a chapter with a reasoned discussion about legal, social ethical and professional issues within the context of your project problem. You should also demonstrate that you are aware of the regulations governing your project area and the Code of Conduct \& Code of Good Practice issued by the British Computer Society, and that you have applied their principles, where appropriate, as you carried out your project.

Throughout the course of this project, I made sure I abode by the principles set out in the Code of Conduct \& Code of Good Practice issued by the British Computer Society\cite{bcscodeofconduct}, acting with integrity, honesty and transparency in the way I handled potential licensing issues with my work and the other work I used as both reference and inspiration. Throughout my report and project video, I have thoroughly discussed and elaborated on the way external code, articles and other references were used in both my writing and my software artefacts. I have also denoted references and inspirations for code in my artefacts via the inclusion of comments in script files. To further ensure full transparency in my research, all of my citations, even remotely tangental ones, are cited in my bibliography as appropriate. In this chapter, the ways in which I dealt with the licenses of code references in my implementations are detailed in the following section \ref{howuse}, while the details on how I plan to release my \textit{own} code and report, both for public access, are detailed in section \ref{howrelease}.  

\section{Using Other People's Resources} \label{howuse}

As I continually worked on my project, I made sure the resources I worked with were freely available to use in an academic context like this.

For example, the Unity tutorial I used as an inspiration of my Godot Poisson Disk Sampling implementation\cite{seblaguetuteYT} has its project files under the MIT License\cite{seblaguetuteGH}, a permissive open-source license which means it can be freely used and adapted with, even commercially.\cite{mitlicense} This meant that I could base my Godot implementation on his Unity implementation without fear of any legal implications. Nonetheless, to act with integrity, I have denoted properly, in this report and in code comments, that I have taken from and adapted his work, citing it accordingly in my bibliography as well.

As aforementioned in chapter \ref{voronoi1}, the JavaScript code example I used from the Procedural Content Generation wiki\cite{pcgwikivoronoi}, for my Voronoi Cells implementation, was submitted by an anonymous Wikidot contributor in 2017. Like most if not all of the Wiki's contents, it is licensed under the Creative Commons Attribution-ShareAlike 3.0 License (all contents of the wiki follow this license unless otherwise specified); that is, the article and its contents (including the JavaScript code example) can be freely used and adapted, subject to the condition that the original source is attributed \textbf{and} that any transformed work, \textit{like my implementation}, \textbf{must} be published under the same or a compatible license.\cite{cc_at_sa_3} Since there are no listed compatible source code licenses I can use in lieu of this license\cite{cc_compat}, I must therefore abide by the license contents of the original article in my source code, since my implementation and the original JavaScript code are similar to a noticeable, although not entirely like for like, degree. I go over how I will release all my project artefacts to the public, as well as the \LaTeX{} source code and BIB\TeX{} citations of this report, in the following section \ref{howrelease}.

Projects I have used as references on a smaller scale were also accounted for. While the Godot 3 TileMap noise tutorial I referenced in chapter \ref{impperlin1} is up on GitHub, it has no readily attached license to it.\cite{gingergd3tutorialGH} However, I did not take or adapt any substantial code from it and the scripting APIs for Godot 3 and Godot 4 are vastly different, \textit{especially} in the context of tile maps, so I fully believe that my implemented code will not pose an issue, and I plan to post this, and all my self-produced artefacts, on GitHub under conditions that I will explain in the following section \ref{howrelease}.

To produce the screenshot in Figure \ref{fig:lattice2}, I forked an existing Godot 3 project on GitHub\cite{codatGD3LSystemGH}, taken from a YouTube video tutorial on how to use an L-System to draw line graphics in Godot\cite{codatGD3LSystemYT}, converted it to Godot 4 and added an additional set of rules to it based on the example lattice grammar featured in chapter \ref{lbasic}.\cite{codatGD4LSystemGH} While I did the conversion and added my own code contributions to the fork, I do not regard this as a substantial part of my project, and thus have not included it in my source code listings in chapter \ref{Code}. The person behind the code \hyperlink{https://www.youtube.com/watch?v=eY9XkJERiG0&lc=UgwXjzr7jheuC9hH18h4AaABAg}{has previously denoted appreciation of other people's forks of his code}, so I do not believe the lack of readily available code license in his original repository is a substantial issue here. Nonetheless, since small parts of his code have been adapted to work with my L-System implementation, I have emailed him directly for his permission to do both that and add the permissive MIT license to my fork, and I got his permission from a private email conversation we had on Tuesday 18\textsuperscript{th} April 2023:

\textit{``Well, haven't done anything with the channel in years, everything is more or less up for grabs. So you have my permission to use anything and/or everything however you like. You can add the MIT license to your fork.}

\textit{You can refer to this conversation if needed!"}

I have let him know that I am citing his work in both my dissertation and the released artefact. I have already added the license to my fork.\cite{codatGD4LSystemGH}

Any usage of external screenshots in Figures throughout this document a are properly cited and linked to in my bibliography. Screenshots that were produced by me are clearly denoted as such in Figure captions. Any usage of code snippets in Figures were written by me, and any external references used as bases for these snippets were clearly and properly cited as such.

Some of the self-produced screenshots, specifically the ones in Figures \ref{fig:lsysiter0}, \ref{fig:lsysiter1}, \ref{fig:lsysiter2} and \ref{fig:lsysiter3}, are taken from a Godot project I created when I was still learning about how L-Systems worked for this dissertation. I decided to include it in my source code listings in chapter \ref{Code}, primarily because much of the code in there (specifically in the script file DemoNode.gd) is used in the final project, but also because the commit history shows the process I initially went through in building an L-System that could handle multiple grammar rules, as detailed in chapters \ref{lcomplex}, \ref{implsys1} and \ref{implsys2}. Do note, however, that it is not as important to the project's motives as my main 4 algorithm implementations. 

\section{How I Will Release My Own Artefacts} \label{howrelease}

I have planned for my source code to both the dissertation and artefacts to be released on GitHub for public access. In order to do so, I have to assign all my repositories licenses so that my usage intentions and any repository usage conditions are clearly defined. 

I have chosen to use the Creative Commons Attribution-Sharealike 4.0 license, as described in \hyperref[howuse]{the previous section}, for both my report and all my artefacts. The aforementioned license allows for commercial and non-commercial usage on the condition that (1) the concerned product is properly attributed to when used and (2) any adaptations of and modifications to this work are released under the same license (or a compatible one, or a later revision of it).\cite{cc_at_sa_4} The concerned product can be used \textit{verbatim} (i.e. as is) without having to share their work under this license, but when it is adapted upon, \textit{then} the share-alike conditions apply.

Although it is not widely considered good practice to apply Creative Commons licenses to code\cite{cc_faq_code}, I still feel that CC-BY-SA-4.0 is the best license for my project overall. As well as resolving any complications with my Voronoi cells implementation, again as discussed in \hyperref[howuse]{the previous section} (CC-BY-SA-4.0 is compatible with the CC-BY-SA-3.0 license used in the original JavaScript code, as CC-BY-SA-3.0 allows for licensing newly adapted works under later versions of the license\cite{cc_compat}), it also ensures that my original Godot implementations are still available for public viewing and modification while also ensuring others can still modify it and everyone else who \textit{isn't} modifying it has the freedom to view these modifications. In that regard, it \textit{is} similar to copyleft licenses such as the GPL and LGPL, though, as of the time of this publication, only the former is listed as compatible with CC-BY-SA-4.0.\cite{cc_compat}

In an academic context, it ensures that the work I have put in behind this report, as well as the implementations, are still available when the code is taken and then built upon by someone else (and that these modifications are also made available for others to see how my work was built upon). Applying the license to this dissertation, as well as my artefacts, allows me to ensure that all aspects of my work are viewable by everyone, and any improvements made to this work, by me or anyone else, are also viewable and publicly recognised.
