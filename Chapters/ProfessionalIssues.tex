\chapter{Legal, Social, Ethical and Professional Issues} \label{Issues}
% Your report should include a chapter with a reasoned discussion about legal, social ethical and professional issues within the context of your project problem. You should also demonstrate that you are aware of the regulations governing your project area and the Code of Conduct \& Code of Good Practice issued by the British Computer Society, and that you have applied their principles, where appropriate, as you carried out your project.

Throughout the course of this project, the author of this report made sure he abode by the principles set out in the Code of Conduct \& Code of Good Practice issued by the British Computer Society\cite{bcscodeofconduct}, acting with integrity, honesty and transparency in the way potential licensing issues with the work produced for this dissertation, and the other work used as both reference and inspiration, were all properly handled. Throughout this report, the ways external code, articles and other references were used in both the writing and the software artefacts have been thoroughly discussed and also elaborated on. References and inspirations for code in those artefacts were also clearly and transparently denoted via the inclusion of comments in script files. To further ensure full transparency in the research done behind this project, every single citation, even remotely tangental ones, are cited in the bibliography of this report as appropriate. In this chapter, the ways in which the licenses of code references in the implementations were adequately dealt with are detailed in the following section \ref{howuse}, while the details on plans behind the author of this report releasing his \textit{own} code and report, both for public access, are detailed in section \ref{howrelease}.  

\section{Using Other People's Resources} \label{howuse}

As this project has been continually worked on, it has been ensured, with confidence, that the resources used were freely available to use in an academic context such as this.

For example, the Unity tutorial used as an inspiration of our Godot Poisson Disk Sampling implementation\cite{seblaguetuteYT} has its project files under the MIT License\cite{seblaguetuteGH}, a permissive open-source license which means it can be freely used and adapted with, even commercially.\cite{mitlicense} This meant that our Godot implementation could use his Unity implementation as a basis without fear of any legal implications. Nonetheless, to act with integrity, it has been denoted properly, in this report and in code comments, that his work has been taken from and adapted, citing it accordingly in the bibliography as well.

As aforementioned in chapter \ref{voronoi1}, the JavaScript code example taken from the Procedural Content Generation wiki\cite{pcgwikivoronoi}, for the Voronoi Cells implementation, was submitted by an anonymous Wikidot contributor in 2017. Like most if not all of the Wiki's contents, it is licensed under the Creative Commons Attribution-ShareAlike 3.0 License (all contents of the wiki follow this license unless otherwise specified); that is, the article and its contents (including the JavaScript code example) can be freely used and adapted, subject to the condition that the original source is attributed \textbf{and} that any transformed work, \textit{like the Godot implementation}, \textbf{must} be published under the same or a compatible license.\cite{cc_at_sa_3} Since there are no listed compatible source code licenses that can be use in lieu of this license\cite{cc_compat}, the Voronoi cells implementation must therefore abide by the license contents of the original article in the source code, since both it and the original JavaScript code are similar to a noticeable, although not entirely like for like, degree (even though \textit{our} implementation is in GDScript and not JavaScript). We will go over how all of the project artefacts will eventually be released to the public, as well as the \LaTeX{} source code and BIB\TeX{} citations of this report, in the following section \ref{howrelease}.

Projects that have been used as references on a smaller scale were also accounted for. While the Godot 3 TileMap noise tutorial referenced in chapter \ref{impperlin1} is up on GitHub, it has no readily attached license to it.\cite{gingergd3tutorialGH} However, since no substantial code from it whatsoever has been taken from or adapted, and the scripting APIs for Godot 3 and Godot 4 are vastly different, \textit{especially} in the context of tile maps, it is therefore highly justified that \textit{our} implemented code will not pose an issue, and so this, and all the other self-produced artefacts, can be posted on GitHub under conditions that are further explained in the following section \ref{howrelease}.

To produce the screenshot in Figure \ref{fig:lattice2}, the author of this report forked an existing Godot 3 project on GitHub\cite{codatGD3LSystemGH}, taken from a YouTube video tutorial on how to use an L-System to draw line graphics in Godot\cite{codatGD3LSystemYT}, converted it to Godot 4 and added an additional set of rules to it based on the example lattice grammar featured in chapter \ref{lbasic}.\cite{codatGD4LSystemGH} While the conversion was done by the author of this report, and some other own code contributions of his were also to the fork, the author of this report does not regard this as a substantial part of his project, and thus have not included it in the source code listings in chapter \ref{Code}. The person behind the code \hyperlink{https://www.youtube.com/watch?v=eY9XkJERiG0&lc=UgwXjzr7jheuC9hH18h4AaABAg}{has previously denoted appreciation of other people's forks of his code}, so the lack of readily available code license in his original repository is not believed to be a substantial issue here. Nonetheless, since small parts of his code have been adapted to work with the L-System implementation, he has been emailed him directly for his permission to do both that and add the permissive MIT license to the new fork, and his permission was received from a private email conversation had between the author of this report and him on Tuesday 18\textsuperscript{th} April 2023:

\textit{``Well, haven't done anything with the channel in years, everything is more or less up for grabs. So you have my permission to use anything and/or everything however you like. You can add the MIT license to your fork.}

\textit{You can refer to this conversation if needed!"}

He has been made aware that his work will be properly and clearly cited in both the dissertation and the released artefact. The MIT license has already added the license to the new fork.\cite{codatGD4LSystemGH}

Any usage of external screenshots in Figures throughout this document a are properly cited and linked to in the bibliography. Screenshots that were self-produced by the author of this report are clearly denoted as such in Figure captions. Any usage of code snippets in Figures were written by him, and any external references used as bases for these snippets were clearly and properly cited as such.

\section{How The Author's Own Artefacts Will Be Released} \label{howrelease}

The author of this report has planned for his source code to both the dissertation and artefacts to be released on GitHub for public access. In order to do so, all of his repositories must be properly assigned licenses so that his usage intentions and any repository usage conditions are clearly defined. 

For both the report and all the artefacts, the Creative Commons Attribution-Sharealike 4.0 license, as described in \hyperref[howuse]{the previous section}, was chosen and assigned. The aforementioned license allows for commercial and non-commercial usage on the condition that (1) the concerned product is properly attributed to when used and (2) any adaptations of and modifications to this work are released under the same license (or a compatible one, or a later revision of it).\cite{cc_at_sa_4} The concerned product can be used \textit{verbatim} (i.e. as is) without having to share their work under this license, but when it is adapted upon, \textit{then} the share-alike conditions apply.

Although it is not widely considered good practice to apply Creative Commons licenses to code\cite{cc_faq_code}, the author still believes that CC-BY-SA-4.0 is the best license for his project overall. As well as resolving any complications with the Voronoi cells implementation, again as discussed in \hyperref[howuse]{the previous section} (CC-BY-SA-4.0 is compatible with the CC-BY-SA-3.0 license used in the original JavaScript code, as CC-BY-SA-3.0 allows for licensing newly adapted works under later versions of the license\cite{cc_compat}), it also ensures that all of the original Godot implementations are still available for public viewing and modification while also ensuring others can still modify it and everyone else who \textit{isn't} modifying it has the freedom to view these modifications. In that regard, it \textit{is} similar to copyleft licenses such as the GPL and LGPL, though, as of the time of this publication, only the former is listed as compatible with CC-BY-SA-4.0.\cite{cc_compat}

In an academic context, it ensures that all of the work and code that has been put into this report by the author of this report, as well as the implementations, are still available when the code is taken and then built upon by someone else (and that these modifications are also made available for others to see how the author's work was built upon). Applying the license to this dissertation, as well as the artefacts, allows him to ensure that all aspects of the work the author has done are viewable by everyone, and any improvements made to this work, by him or anyone else, are also viewable and publicly recognised.

\subsubsection{Note on the Godot Project Used to Create Some Screenshots in Chapter \ref{alglsys} \& Why It Has Been Included With The Rest of The Artefacts}

Some of the self-produced screenshots, specifically the ones in Figures \ref{fig:lsysiter0}, \ref{fig:lsysiter1}, \ref{fig:lsysiter2} and \ref{fig:lsysiter3}, are taken from a Godot project created by the author of this report early on, during the learning process on how L-Systems worked for this dissertation. It has eventually been decided to include it in the source code listings in chapter \ref{Code}, primarily because much of the code in there (specifically in the script file DemoNode.gd) is used in the final project, but also because the commit history shows the process the author of this report initially went through in building an L-System that could handle multiple grammar rules, as detailed in chapters \ref{lcomplex}, \ref{implsys1} and \ref{implsys2}. Do note, however, that it is not as important to the project's motives as the main four algorithm implementations. 