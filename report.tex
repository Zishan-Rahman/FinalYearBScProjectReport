%%%%%%%%%%%%%%%%%%%%%%%%%%%%%%%%%
% 6CCS3PRJ Final Year Individual Project Report
% zishan.rahman@kcl.ac.uk
%%%%%%%%%%%%%%%%%%%%%%%%%%%%%%%%%
\documentclass[11pt]{informatics-report}
\usepackage{etoc}
\usepackage{fontspec}
\usepackage{fontawesome}
\usepackage{color}
\usepackage{hyperref}
\usepackage[justification=Centering]{caption}
\usepackage{float}
\newcommand{\link}[1]{\href{#1}{#1}}
\usepackage[square,sort,comma,numbers]{natbib} %References

%TC:subst \verbatiminput input

% https://tex.stackexchange.com/a/353234

\usepackage{calc}
\usepackage[usenames,dvipsnames,svgnames,table]{xcolor}
\usepackage{pdfpages,graphicx}
\usepackage{mdframed}
\usepackage{listings}
\definecolor{light-gray}{gray}{0.92}

\definecolor{mainColor}{RGB}{211, 47, 47} % some dark red

\renewcommand\lstlistingname{Code}
\lstset{
	language=Python,
	numbers=left,
	numbersep= 7mm,
	numberstyle=\color{Black},
	stepnumber=1,
	tabsize=3,
	breakatwhitespace=false,
	breaklines=true,
	captionpos=b,
	basicstyle=\color{Black}\ttfamily,
	commentstyle=\color{LimeGreen},
	keywordstyle=\color{BurntOrange}\bfseries,
	stringstyle=\color{WildStrawberry},
	keywords={var, func, extends},
	frame=leftline,
	framesep=0mm,
	xleftmargin=3mm,% marge ajouté à gauche du tableau (à configurer en dernier pour l'alignement global du tableau)
	framesep=2mm, %distance texte bord du cadre (limite de la background color)
	framerule=0mm,
	abovecaptionskip=5mm,
	aboveskip=\baselineskip,
	belowskip=\baselineskip
}

% \usepackage{tcolorbox}
% \tcbuselibrary{skins,breakable,listings}
% \newtcblisting[use counter=lstlisting]{codeblock}[2][]{%
% 	enhanced,noparskip,breakable,colback=light-gray,colframe=DarkSlateGray,opacitybacktitle=.8,%
% 	fonttitle=\bfseries,before upper={\hspace*{-1em}\includegraphics[height=\baselineskip]{example-image-a}~#2},%
% 	title after break={\centering\footnotesize\itshape\strut\lstlistingname~\thelstlisting~--~continued},%
% 	listing only,listing options={xleftmargin=-1mm},after upper={\centering\strut\lstlistingname~\thelstlisting:~#2},
% 	frame hidden,arc=0pt,outer arc=0pt,boxrule=0pt,frame code={\draw[gray,line width=2mm] ([xshift=-0.5pt]frame.north west) -- ([xshift=-0.5pt]frame.south west);},#1}

%%%%%%%%%%%%%%%%%%%%%%%%%%%%%%%%%
% Front Matter - project title, name, supervisor name and date
%%%%%%%%%%%%%%%%%%%%%%%%%%%%%%%%%
\title{6CCS3PRJ Final Year\\\vspace{0.2cm}Implementing Procedural Content Generation Algorithms in a Tile Map RPG in the Godot Game Engine}
\author{Zishan Rahman}
\studentID{20071291}
\supervisor{Senir Dinar}

\date{\today}

\abstractFile{FrontMatter/abstract.tex}
\ackFile{FrontMatter/acknowledgements.tex} %Remove line if you do not want acknowledgements

\begin{document}
\createFrontMatter
\onehalfspacing
\tableofcontents{}
\doublespacing

%%%%%%%%%%%%%%%%%%%%%%%%%%%%%%%%%
% Report Content
%%%%%%%%%%%%%%%%%%%%%%%%%%%%%%%%%
% You can write each chapter directly here or in a separate .tex file and use the include command.

\chapter{Introduction}
%This is one of the most important components of the report. It should begin with a clear statement of what the project is about so that the nature and scope of the project can be understood by a lay reader. It should summarise everything that you set out to achieve, provide a clear summary of the project's background and relevance to other work, and give pointers to the remaining sections of the report, which will contain the bulk of the technical material.

Procedural Content Generation, or PCG, refers to the use of algorithms and programming in lieu of human handiwork to design and implement various contents in video games, such as levels, terrains, trees and cities. A PCG algorithm is ontogenetic when it tries to produce a foreseeable end result as it goes along. For this project, several well-known ontogenetic algorithms have been implemented in a basic 2D tile-map-oriented RPG-like game, using the open-source Godot game engine, and then comparing how each algorithm carries out the creation of levels in said game, both performance-wise and comparing the kinds of level layouts generated by each algorithm. The aim here is to weigh up the best algorithm for the chosen scenario based on how similar and how different each implementation of the algorithm provides with its level layouts, and show how every algorithm was well integrated into the chosen scenario and adapt the scenario and the algorithm as needed.

\section{Report Structure}

First, there will be some \hyperref[Background]{background} provided into the work done behind this dissertation, demonstrating some understanding of PCG in video games and eventually justifying the choice to use Godot, why a 2D tile map RPG was chosen to adapt to a PCG context and why each algorithm was chosen to implement each PCG algorithm into the defined scenario.

The main \hyperref[Body]{report body} will go through firstly how each algorithms works, and secondly go into how they were implemented into the chosen scenario at a surface-level explanation. This report goes into further detail in the \hyperref[Implementation]{Implementation} section.

The \hyperref[Design]{Design \& Specification} section will firstly go through what was sought after in every implementation of the chosen scenario, in order to be able to compare each implementation with one another and then determine how each algorithm was ranked according to the relevant criteria in the \hyperref[Evaluation]{Evaluation} section.

The \hyperref[Implementation]{Implementation} section will go into further detail than done in \hyperref[Body]{the report body} as to how each algorithm implemented and any code issues were worked around, including GDScript code snippets where needed.

The \hyperref[Issues]{Legal, Social, Ethical and Professional Issues} section will discuss how, firstly, any issues with any external code references used for any software artefacts were eventually worked around, and how integrity was practiced while doing so. Secondly, this section will go through the plans set out to make both the dissertation and the artefacts behind it publicly available while still taking care of any potential professional issues. 

The \hyperref[Evaluation]{Results \& Evaluation} section will go through some of the quantifiable results obtained in experimenting with theses implementations and any custom values that were set during those experiments. This report has included these tables in the \hyperref[Appendix]{Appendix}, so as not to break the flow of the report itself. This section will also discuss the conclusions obtained and how each algorithm was ranked in terms of how they turned out with the given scenario, as well as how similar and different the produced level layouts were.

Finally, in the \hyperref[Conclusion]{Conclusion and Future Work} section, there will be a final summary of the conclusions obtained and discussed earlier, in addition to what was gained by the author of this report as a games programmer and student from this project, and where this project and its aims could be taken further.

\chapter{Background}
%The background should set the project into context by motivating the subject matter and relating it to existing published work. The background will include a critical evaluation of the existing literature in the area in which your project work is based and should lead the reader to understand how your work is motivated by and related to existing work.

For my BSc individual project, I will be researching procedural content generation (PCG) algorithms and then implementing them each in a small 3D game made with the Godot Engine (and its domain-specific GDScript language).

\section{Procedural Generation: Background}

\href{https://en.wikipedia.org/wiki/Procedural_generation}{Procedural content generation} (usually referred to as simply ``procedural generation") refers to the creation of levels and other game objects programmatically and algorithmically, in lieu of a human being doing all the work. While procedural generation algorithms can be used to generate a myriad of things, from textures (for things like trees and clouds) to music (``generative music," as coined by legendary musician Brian Eno), by far its most common context is in automated level design, generating level layouts algorithmically in lieu of work from level designers. Game developers may opt to use procedural generation to save time and money designing levels or show off technical prowess in their games.

\href{https://en.wikipedia.org/wiki/Procedural\_generation#Video\_games}{Procedural generation in video games has a rich history.} Pioneering games such as Rogue (1980) took direct influence from tabletop role-playing games such as Dungeons and Dragons, and thus had a player navigate a randomly-generated world that expanded further as they went on. Such games spawned the \emph{roguelike} and \emph{roguelite} genres, which experienced immense popularity in the last decade. In the realm of first-person shooters, 2004's .kkrieger, as seen in Figure \ref{fig:kkrieger}, used procedural generation to create intricate 3D levels and fit them all into a game that takes up just 96 kilobytes of space. 

%\begin{figure}[H]
%	\centering
%	\includegraphics[width=0.75\textwidth]{kkrieger.png}
%	\caption{The game .kkrieger, which uses procedural generation to design maps while keeping the game at a 96 kilobyte file size.\\Source: \link{https://www.researchgate.net/figure/The-game-kkrieger-has-a-file-size-of-96-kb\_fig1\_320722498}}
%	\label{fig:kkrieger}
%\end{figure}

Other games that use procedural generation in its levels include Elite (originally published in 1984), Elite: Dangerous (2012), Minecraft (2009), No Man's Sky (2012) and Spelunky (2013). \href{https://youtu.be/Uqk5Zf0tw3o}{The latter game's use of procedural generation has notably been covered by video games journalist Mark Brown in a YouTube video.}

%\begin{figure}[H]
%	\centering
%	\includegraphics[width=0.7\textwidth]{spelunky.jpg}
%	\caption{\href{https://spelunkyworld.com/}{The roguelike game Spelunky}, which uses procedural generation to build intricate levels for the player character to explore.\\Source: \link{https://store.steampowered.com/app/239350/Spelunky/}}
%	\label{fig:spelunky}
%\end{figure}

In many cases, these games end up having a \textbf{large} number of different environments that each game could generate for its players. However, by procedurally generating them upon the \textit{loading} of the game level, in lieu of loading a layout from disk, they can save a lot of space (albeit with a considerable need for processing power, depending on the game's and algorithms' performance), as seen with .kkrieger.% in Figure \ref{fig:kkrieger}.

Using one or some different procedural generation algorithms, such as the use of Perlin, Simplex or other noise, Voronoï disks and also poisson disk generation, among others, games can load a seed to randomly generate a level every time it is played, meaning no two playthroughs of a game with procedurally generated content are ever the same.

\section{Justifying My Choice of Engine: Godot}

While a myriad of resources exist for procedurally generated game contents exist for Unity and Unreal, I want to implement them in Godot, for several reasons:

\begin{itemize}
	\item It's the engine I have the most experience with, having already developed 2 published web games with it.
	\item It's not got as many resources on procedural generation compared to Unity, Unreal and some other popular game engines, particularly on the side of academic research (that is, there aren't as many papers on procedural generation that pertain to Godot as they do to Unity, Unreal and other engines).
	\begin{itemize}
		\item However, it is still very powerful and feature-rich (it has its own Open Simplex noise class, for example) and I'm sure I can make procedural generation algorithms work on it.
	\end{itemize}
	\item Compared to Unity and Unreal, Godot is a very light engine with a feature-rich editor, clocking in at under 100MB, with editors for Windows, macOS, Linux and even the web browser. 
\end{itemize}

By the end of my allotted time, I plan to have implemented several procedurally generated environments in small Godot games, using a myriad of methods (such as Voronoï cell and poisson disk generation) in a myriad of contexts (anything from platformers to first-person games). With these games, I plan for the final report to be the centrepiece of my project, with it containing my research on how each environment was implemented, as well as my findings on the algorithms themselves and how they work.

This is more a research-oriented project than an implementation-oriented project, but the implementations will nonetheless prove that Godot is just as adept at procedural content generation as the other major players in the game engine space, and I will have gained immense knowledge on PCG in the process.

\subsection{Note on Differing Versions of Godot}

Godot currently is at version 3, but concurrently there is also Godot 4, which is nearing its release and is in working condition. The latter version of Godot contains several new features and breaking changes, so any project made in Godot 3 won't readily be compatible with Godot 4 (and vice-versa) without making the necessary changes and conversions. I have access to both versions of Godot and, for all the Godot projects I create, which version I'm using will be clearly denoted and clarified.
\chapter{Report Body}
%The central part of the report usually consists of three or four chapters detailing the technical work undertaken during the project. {\bf{\textcolor{red}{The structure of these chapters is highly project dependent}}}. They can reflect the chronological development of the project, e.g. design, implementation, experimentation, optimisation, evaluation, etc (although this is not always the best approach). However you choose to structure this part of the report, you should make it clear how you arrived at your chosen approach in preference to other alternatives. In terms of the software that you produce, you should describe and justify the design of your programs at some high level, e.g. using OMT, Z, VDL, etc., and you should document any interesting problems with, or features of, your implementation. Integration and testing are also important to discuss in some cases. You may include fragments of your source code in the main body of the report to illustrate points; the full source code is included in an appendix to your written report.

\section{Algorithms}

\subsection{Lindenmayer Systems}

Hungarian academic Aristid Lindenmayer devised a mathematical model for the reproduction of fungi in 1967.\cite{LINDENMAYER1968300} His model involved a string of symbols, each unique symbol denoting a specific action and/or branch. Essentially, running that initial string, called the \emph{axiom}, through a set of rules (called a \emph{grammar}) gives us an ever-expanding string that is then taken as instructions to draw something from. Lindenmayer Systems, or L-Systems, have since been used in several scenarios beyond its initial purpose of modelling fungi, from trees to fractals. In video games, they are frequently used to aid in the creation of foliage in several environments, as well as buildings and, here, level layouts.

\subsubsection{A Basic 0L-System}

The most basic form of L-System is a \emph{0L}-System, 0 in this case referring to the fact that the grammar is \emph{context-free}.

For this example\cite{lsyspaulbourke}, consider an alphabet $V$, which consists of the following symbols:

\newcommand{\F}{\mbox{F}}

$$ \F, +, - $$

where $\F$ means ``to go forward", and $+$ and $-$ denote turning right or left (respectively) a set number of degrees $\o$.

Take an axiom $\omega$, for example:

$$ \F+\F+\F+\F $$

And a set of rules $P$ which, in this case, is of size 1:

$$ \F \rightarrow \F+\F-\F-\F\F+\F+\F-\F $$

We can represent this \emph{parametric} L-system in the following form:\cite{enwiki:1124510226}

$$ G = (V, \omega, P) $$

To implement $G$ in Godot, we can take each rule and replace each string in accordance to our one rule, using the replace method, like so:

\begin{codeblock}{Simple String Replacement for an L-System with 1 rule}
string = string.replace(rule["from"], rule["to"]) #Here the rules were stored in dictionaries.
\end{codeblock}

The first 3 iterations of this operation are shown here:

\begin{figure}[H]
	\centering
	\includegraphics[width=0.5\textwidth]{Images/initial-l-system-iteration-0.png}
	\caption{The axiom of the aforementioned simple L-System with just one rule. String size: 8.\\Source: Own work.}
	\label{fig:lsysiter0}
\end{figure}

\begin{figure}[H]
	\centering
	\includegraphics[width=0.5\textwidth]{Images/initial-l-system-iteration-1.png}
	\caption{The first iteration of the aforementioned simple L-System with just one rule. String size: 59.\\Source: Own work.}
	\label{fig:lsysiter1}
\end{figure}

\begin{figure}[H]
	\centering
	\includegraphics[width=0.5\textwidth]{Images/initial-l-system-iteration-2.png}
	\caption{The second iteration of the aforementioned simple L-System with just one rule. String size: 475.\\Source: Own work.}
	\label{fig:lsysiter2}
\end{figure}

\begin{figure}[H]
	\centering
	\includegraphics[width=0.5\textwidth]{Images/initial-l-system-iteration-3.png}
	\caption{The third iteration of the aforementioned simple L-System with just one rule. String size: 3803. The string is too large to show in the window, as you can see here.\\Source: Own work.}
	\label{fig:lsysiter3}
\end{figure}

The resulting string can be used to draw a lattice.\cite{lsyspaulbourke} Examples of the above grammar in action are below.

\begin{figure}[H]
    \centering
    \includegraphics[height=0.25\textheight]{Images/lsys03.png}
    \caption{A lattice generated with the example grammar on a custom-written Classic Mac OS application specifically written for working with L-Systems.\cite{lsyspaulbourke}}
    \label{fig:lattice1}
\end{figure}

\begin{figure}[H]
    \centering
    \includegraphics[width=0.75\textwidth]{Images/gd4lattice.png}
    \caption{A lattice generated with the example grammar on a Godot project for drawing from L-Systems. Source: Initial project written by YouTuber Codat\cite{codatGD3LSystemYT}\cite{codatGD3LSystemGH}, and converted to Godot 4 (with the addition of the lattice grammar) by me.\cite{codatGD4LSystemGH}}
    \label{fig:lattice2}
\end{figure}

\subsubsection{A More Complex D0L-System With More Than One Rule}

For handling more than one rule, we can instead use a new string buffer variable where, for each character in our string, we can attain a new string and append it to our string buffer. The resulting string is then returned and interpreted. This can be represented in Godot like so:

\begin{codeblock}{String replacement for an L-System with 2 rules}
func get_new_replacement(character: String) -> String:
	for rule in rules:
		if rule["from"] == character:
			return rule["to"]
	return ""

func replace_string(string: String) -> String:
	var new_string = ""
	for character in string:
		new_string += get_new_replacement(character)
	return new_string
\end{codeblock}

This can \textit{then} be used to handle more complex grammars that can handle more than one rule in which characters in strings are replaced by other strings of variable length, as before.

The grammar in the following example represents a D0L-System\cite{lsystemintro}, a \textbf{deterministic} L-System using a context-free grammar; the grammar in the first example was \textit{also} deterministic.

\newcommand{\A}{\mbox{a}}
\newcommand{\B}{\mbox{b}}

For this example, consider a new grammar $G$ with the alphabet $V$, where $\A$ and $\B$ are the only symbols. We start with the following axiom $\omega$, which is just $\A$. We now have a set of rules $P$ which is, this time, of size \textit{2}:

$$ \A \rightarrow \A\B $$
$$ \B \rightarrow \A $$

The first few steps of the resulting derivation can be modelled like so:

\begin{figure}[H]
    \centering
    \includegraphics[scale=0.5]{Images/derivationtree.png}
    \caption{The first few steps of a derivation of our example grammar.\cite{lsystemintro}}
    \label{fig:derivationtree}
\end{figure}

\subsection{Voronoi Cells}

Named after the Ukranian mathematician Georgy Voronoy, Voronoi cells work by taking a map of points, and randomly selecting a group of points. Within that selected group, cells are formed by calculating, in each point of the grid, the closest of the selected points to it. That is, each cell represents the group of points that are the closest to that random point (including that point in the group as well). The final arrangement of cells represents a Voronoi Diagram or Voronoi Tesselation.

Distances between points can be calculated with either the Euclidean distance:

$$ d_{E}(p, q) = \sqrt{(q_x - p_x)^2 + (q_y - p_y)^2} $$

or the Manhattan distance:

$$ d_{M}(p, q) = |q_x - p_x| + |q_y - p_y| $$

With the Euclidean distance producing a more ``triangulated" tesselation than the Manhattan distance, the geometry of which is more "blocky" and resembles taxicabs (hence its alternate name "Taxicab Geometry").

\subsection{Poisson Disk Sampling}

Poisson disk distributions are an easy way to randomly scatter objects across a field. It's commonly used for tree placement and placement of other random objects.

\subsection{Simplex Noise}

Kenneth Perlin designed a type of noise named after himself (Perlin Noise), in which each pixel of noise is affected by its surroundings. 

\section{Implementations}

Here I will describe, at surface level, the methods I went about implementing the above algorithms and what references I used.

\subsection{Lindenmayer System}

The implementation of an L-System was very simple. I took inspiration from a YouTube video on implementing an L-System for drawing line graphics in Godot by Codat.\cite{codatGD3LSystemYT} In the code from the Godot 3 project he made in that video\cite{codatGD3LSystemGH}\cite{codatGD3LSystemYT}, he created a custom ``Rule" class in GDScript, with which he defined new rules.

\subsection{Voronoi Cells}

\subsection{Poisson Disk Sampling}

\subsection{Simplex Noise}

The Simplex Noise implementation works with Godot's built-in Noise library. Within a Sprite2D node's Texture attribute, I set a new ``NoiseTexture2D" field inside of it. In its ``Noise" attribute I created a new ``FastNoiseLite" scene, which generates a noise texture for us to use. The seed can be set in the sprite's script file.

As with my other implementations, there are two separate arrays, one for trees and another for buildings. For each cell in the TileMap, I then took the noise pixel from the generated texture at that exact point (scaling with the cell size accordingly), and then, depending on the value retrieved, decided, firstly, whether or not to place a plant/tree tile there and, secondly, whether or not to place a building tile there. As a result, not every cell in the TileMap has tiles on it. On any one of those empty cells, the Player tile will then get placed.
\chapter{Design \& Specification}
  
% Provide abstract level of how I intend to compare them
% How do I make sure the implementation is like for like
% 

Here, I will provide an abstract level of how I compared the performance of each content generation algorithm and how I made sure each implementation could produce as similar/like-for-like results as possible (and where they \textit{couldn't} do so).

\section{Performance}

With the L-System implementation, I had no problems running the game very quickly on my machine, and quickly got results.

With Poisson Disk Sampling, the higher the number of rejection samples (that is, the higher the maximum number of times a cell was sampled before it was either accepted or ultimately rejected), the longer it took to generate a complete level layout, and even, due to the nature of the tile map compared to the algorithm's \textit{usual} use (of scattering dots on a plane), it was not maximal (not all points had cells painted for them; some cells had their tiles overwritten as well). Using 8 rejection samples was usually enough to yield a satisfactroy level layout.

Voronoï Cells took the longest to compute on average. 

\section{Layouts}

Of the 4 implementations I made, the Noise and Poisson Disk Sampling implementation were by far the most similar, followed by the L-System implementation, and then the Voronoï Cells implementation, which was far and away the most unique.

While the noise implementations were 
\chapter{Implementation}

% Go deeper into how I got each algorithm to work
% How I decide

Here I will go a bit deeper as to how I made each algorithm work. Where possible, I plan to use code snippets from the work I have done to justify how and why things were implemented the way they were.

\section{Poisson Disk Sampling}

To be able to access the inner and outer grid sizes in my implementation of this algorithm, since GDScript does not have a concept of different \verb|Array|s and lists, I stored the lengths of the inner and outer grid in local variables in the \verb|generate_points| function. Those local variables, \verb|grid_x_axis_size| and \verb|grid_y_axis_size| as shown in Figures \ref{fig:pds1} and \ref{fig:pds2}, essentially store the same grid size values as in Lague's implementation, right down to performing the same division in a ceiling function, to the inner grid and the outer grid respectively. Since these dimensions would also be needed for \verb|is_valid|, instead of creating 2 more script variables, I instead took them in as 2 additional method parameters, as shown in Figures \ref{fig:pds3} and \ref{fig:pds4}, and used them accordingly when calculating the maximum and minimum bounds for searching the nearest points of the cell, as shown in \ref{fig:pds5}. Doing it this way ensured that the computation of this algorithm would stay efficient and not stall with an adequate (not too high) number of rejection samples.

\begin{figure}[H]
    \centering
    \begin{lstlisting}
var grid_x_axis_size: int = ceili(sample_region_size.x/cell_size)
var grid_y_axis_size: int = ceili(sample_region_size.y/cell_size)
    \end{lstlisting}
    \caption{The lines used to determine the inner and outer dimensions of the grid array.}
    \label{fig:pds1}
\end{figure}

\begin{figure}[H]
    \centering
    \begin{lstlisting}
for i in range(grid_x_axis_size):
	grid.append([])
	for j in range(grid_y_axis_size):
		grid[i].append(0)
    \end{lstlisting}
    \caption{The nested for-loop that initialises the grid array. First, each inner array is initialised and inserted, then a number of zeroes, determined by the grid's y-dimension, are inserted.}
    \label{fig:pds2}
\end{figure}

\begin{figure}[H]
    \centering
    \begin{lstlisting}
if is_valid(candidate, sample_region_size, cell_size, radius, points, grid, grid_x_axis_size, grid_y_axis_size):
    \end{lstlisting}
    \caption{The line that uses the grid's x and y dimensions as parameters. This calls the is\_valid method using those additional parameters (see Figure \ref{fig:pds4}).}
    \label{fig:pds3}
\end{figure}

\begin{figure}[H]
    \centering
    \begin{lstlisting}
func is_valid(candidate: Vector2, sample_region_size: Vector2, cell_size: float, radius: float, points: Array[Vector2], grid: Array[Array], grid_x_axis_size: int, grid_y_axis_size: int) -> bool
    \end{lstlisting}
    \caption{The function is\_valid, which takes in 2 additional parameters denoting the x and y dimensions of the grid array used in generate\_points.}
    \label{fig:pds4}
\end{figure}

\begin{figure}[H]
    \centering
    \begin{lstlisting}
var search_end_x: int = min(cell_x + 2, grid_x_axis_size - 1)
var search_end_y: int = min(cell_y + 2, grid_y_axis_size - 1)
    \end{lstlisting}
    \caption{The relevant lines of code in is\_valid that reference the grid's x and y dimensions, stored in additional variables as aforementioned.}
    \label{fig:pds5}
\end{figure}

\section{Voronoï Cells}

The original JavaScript implementation, as mentioned before, had a \verb|randRange| function that I took out, but there was also an additional \verb|mapSize| parameter in \verb|definePoints| that, in \textit{my} \verb|define_points| function, didn't really need, since I made sure the map's dimensions were readily accessible via the \verb|x_tile_range| and \verb|y_tile_range| script variable. I therefore took out the second parameter in \verb|define_points|, as shown in Figure \ref{fig:voronoi1}, and substituted it with \verb|x_tile_range| and \verb|y_tile_range| accordingly, as shown in Figure \ref{fig:voronoi3}.

The type of each Voronoï cell was determined by taking, and then deleting, a value from the \verb|types| array. Said array is local to that function, and it is initialised by duplicating the \verb|trees| array, then appending it with the \verb|buildings| array, making sure the same type cannot be used for a Voronoï cell twice. Duplicating the array before merging it essentially makes sure that the \textit{original} \verb|trees| array is not affected by deletions performed on the \verb|types| array. This computation is shown in Figure \ref{fig:voronoi2}, and the deletion operation is shown in Figure \ref{fig:voronoi4}.

\begin{figure}[H]
    \centering
    \begin{lstlisting}
func define_points(num_points: int) -> void:
    \end{lstlisting}
    \caption{The define\_points function header, with no argument for the map's size. The num\_points value that gets taken in during runtime is determined by the script's export variable random\_starting\_points.}
    \label{fig:voronoi1}
\end{figure}

\begin{figure}[H]
    \centering
    \begin{lstlisting}
var types: Array[Vector2i] = trees.duplicate()
types.append_array(buildings)
    \end{lstlisting}
    \caption{The types array being initialised in define\_points, with its values taken from the trees and buildings arrays, such that no type can be used for a cell twice, while also making sure that the original trees and buildings arrays are not affected by the deletions on types.}
    \label{fig:voronoi2}
\end{figure}

\begin{figure}[H]
    \centering
    \begin{lstlisting}
var x: int = randi_range(0, x_tile_range)
var y: int = randi_range(0, y_tile_range)
    \end{lstlisting}
    \caption{Godot's built-in randi\_range function being used in place of a self-defined one in define\_points.}
    \label{fig:voronoi3}
\end{figure}

\begin{figure}[H]
    \centering
    \begin{lstlisting}
var type: Vector2i = types.pick_random()
types.erase(type)
    \end{lstlisting}
    \caption{The types of each Voronoï cell being picked and the erased in define\_points.}
    \label{fig:voronoi4}
\end{figure}
\chapter{Legal, Social, Ethical and Professional Issues} \label{Issues}
% Your report should include a chapter with a reasoned discussion about legal, social ethical and professional issues within the context of your project problem. You should also demonstrate that you are aware of the regulations governing your project area and the Code of Conduct \& Code of Good Practice issued by the British Computer Society, and that you have applied their principles, where appropriate, as you carried out your project.

\section{Using Other People's Resources}

Throughout my project, I made sure the resources I worked with were freely available to use in an academic context like this.
For example, the Unity tutorial I used as an inspiration of my Godot Poisson Disk Sampling implementation\cite{seblaguetuteYT} has its project files under the MIT License\cite{seblaguetuteGH}, a permissive open-source license which means it can be freely used and adapted with, even commercially.
The JavaScript code example I used from the Procedural Content Generation wiki, for my Voronoi Cells implementation, was submitted by an anonymous Wikidot contributor in 2017 and, like most if not all of the Wiki's contents, is licensed under the Creative Commons Attribution-ShareAlike 3.0 License; that is, the article and its contents (including the JavaScript code example) can be freely used and adapted, subject to the condition that the original source is attributed \textbf{and} that any transformed work, \textit{like my implementation}, \textbf{must} be published under the same or a compatible license. Since there are no listed compatible source code licenses I can use in lieu of this license, I must therefore abide by the license contents of the original article in my source code, since my implementation and the original JavaScript code are similar to a noticeable, but not entirely like for like, degree.

\section{How I Will Release My Own Artefacts}

\chapter{Results \& Evaluation} \label{Evaluation}

Here, we will discuss how the implementations of the algorithms in our scenario were tested and ensured that they ran as they should.

\section{Software Testing}

Due to the nature of the project (being several implementations of a computer game), the testing behind this project has solely revolved around trial-and-error, messing around with the exported variables in the Godot editor to see how things worked and what configurations worked best for our scenario. This involved taking many screenshots of generated levels and examining things by eye, seeing how layouts compared across implementations.

Despite this, it was eventually decided to run some simple performance tests to see how long each algorithm ran. These tests took some of the custom export variables from the scene scripts and ran them several times, with an average time calculated to the nearest millisecond. The results of these tests are all in table in the \hyperref[Appendix]{Appendix}.

\section{Comparing the Different Algorithms and Drawing Conclusions on Which Ones Are Best}

\subsection{Performance} \label{performance}

With the L-System implementation, there were no problems whatsoever running the game very quickly on the author of this report's machine, and quickly got satisfactory results. The table in Figure \ref{fig:table3} shows that the processing times remained miniscule, even as the maximum length of the axiom increased. With the default values for all export variables, it took an average of 16ms for the L-System implementation to generate levels, by far the fastest out of all the implementations that were created for this report, as shown in table \ref{fig:table6}.

While the Noise implementation was slower than the L-System implementation by a magnitude, it was still satisfactorily quick. Some timed tests were run in which this paper's author tested how some of the properties affected the time it took to generate the noise. These tests are referred to in tables \ref{fig:table1}, in which each noise algorithm was paired with each cellular distance function, and \ref{fig:table2}, in which each noise algorithm was paired with each fractal type. With the default values set \hyperref[noisedefaults]{here}, it was found that it took an average of 81ms, as shown in table \ref{fig:table6}.

With Poisson Disk Sampling, the higher the number of rejection samples (that is, the higher the maximum number of times a cell was sampled before it was either accepted or ultimately rejected), the longer it took to generate a complete level layout, and even then, due to the nature of the tile map compared to the algorithm's \textit{usual} use (of scattering dots on a plane), it was not maximal (not all points had cells painted for them; some cells had their tiles overwritten as well). Using 8 rejection samples was usually enough to yield a satisfactory level layout while also keeping level creation times to a satisfactory minimum. It took an average of 268ms to work with 8 rejection samples in the tests in table \ref{fig:table4}, and 197ms in some additional tests done in table \ref{fig:table6}. If the rejection samples were set to a too high value, there was a high  chance that the game would hang and not return any cell points at all, because it just took \textit{way} too long to process. However, that does not always happen; as referenced in the caption of table \ref{fig:table4}, on one occasion, when the value of rejection samples was set to 18, the game \textit{did} stall, and had to be restarted again so that enough processing times could be recorded.

Voronoï Cells took the longest to compute on average. Computations with the Euclidean distance measurement took longer than those measured with the Manhattan distance, and the number of random starting points (and therefore the number of unique Voronoï cells in a single tesselation) increased level generation times as well. Both of those results are solidly proven in table \ref{fig:table5}, as well as table \ref{fig:table6}, in which, even \textit{with} the default values set, it \textit{still} took 455ms on average, far longer than any of the other implementations.

\subsection{Layouts}

Of the 4 implementations that were made for this project, the Noise and Poisson Disk Sampling implementation were by far the most similar, followed by the L-System implementation, and then the Voronoï Cells implementation, which was far and away the most unique.

While the noise implementations varied greatly depending on what settings were used, and the way the implementation was designed allowed for very many possibilities as to how the noise would turn out (and how it would affect the final level), the results that were returned produced the most similar results to that of the Poisson Disk Sampling implementation had the following configurations:

\begin{itemize} \label{noisedefaults}
    \item Noise Type (``noise\_type"): Simplex Smooth
    \item Fractal Type (``fractal\_type"): Fractal None
    \item Cellular Distance Type/Function (``cellular\_distance\_type"): Distance Euclidean
    \item Noise Frequency (``noise\_frequency"): 0.894
    \item Tree Cap (``tree\_cap"): -0.048
    \item Building Cap (``building\_cap"): -0.252
    \item Building Overtakes Tree (``building\_overtakes\_tree"): 0.12
\end{itemize}

The default noise frequency in ``FastNoiseLite" is 0.01, which results in smoother and less disparate noise. As seen in Figure \ref{fig:simplexsmooth0.01}, the smoother noise and lower frequency results in a distinct kind of level layout in which represents some of the noise values in the image very clearly, such that tiles (both buildings and trees) are bunched together in partially interconnected groups, forming long, large lines of painted tiles. To describe this as best as possible, it is easy to discern that the level layout was determined from a noise image. Using a higher noise frequency to produce rougher noise, and more disparate level layouts, yields results like in Figure \ref{fig:simplexsmoothdefault1}, which makes it very similar to the layouts yielded in the Poisson Disk Sampling implementation and, to a lesser extent, the L-System implementation. While the author of this report's personal tastes are fond of the former kind of level layout, part of the aim of this project was to compare in terms of which could produce the most similar, and, compared to the L-System and Poisson Disk Sampling implementations, the Noise implementation with the frequency set to 0.01 was far too distinct, hence the want to change it up.

\begin{figure}[H]
    \centering
    \includegraphics[width=0.8\textwidth]{Images/simplex_smooth_default_1.png}
    \caption{A level of our scenario generated in the Simplex Noise implementation, using all of the default values \hyperref[noisedefaults]{shown here}. The level took a total of 99 milliseconds to be made.}
    \label{fig:simplexsmoothdefault1}
\end{figure}

\begin{figure}[H]
    \centering
    \includegraphics[width=0.8\textwidth]{Images/simplex_smooth_0.01_frequency.png}
    \caption{A level of our scenario generated in the Simplex Noise implementation, setting the noise frequency to 0.01 (the default value for noise frequency in ``FastNoiseLite") and using the rest of the defaults \hyperref[noisedefaults]{shown here}. The level took a total of 104 milliseconds to be made.}
    \label{fig:simplexsmooth0.01}
\end{figure}

The Poisson Disk Sampling implementation also produced similar levels, with the main difference higher level processing times, as shown in section \ref{performance}. The lower the number of rejection samples, the quicker the processing times, but the lower the number of tiles painted in the tile map. This is shown in Figures \ref{fig:poisson2} and \ref{fig:poisson3}, with the latter being a particularly egregious example. Figure \ref{fig:poisson1}, on the other hand, shows a level created with the default number of rejection samples (8), and is similar to the level generated with Simplex noise in Figure \ref{fig:simplexsmoothdefault1}.

\newpage

I also have 3 other export variables:

\begin{itemize}
    \item ``point\_radius", which sets the distance between points during calculation, in that no point can be within a radius distance of other points. By default it is set to 1.0. The higher the value, the more spaced apart painted tiles are, as seen in Figure \ref{fig:poisson4}.
    \item ``paint\_building\_probability", which determines whether or not to print a building tile in lieu of a tree tile at a cell (overwriting any existing tile if there is any at that cell). By default it is set to 0.125, the higher the value (between 0.0 and 1.0 inclusive), the more likely a building is to be painted and the more buildings that will appear in the final level, as seen in Figure \ref{fig:poisson5}.
    \item ``region\_size", which, by default, is set to the current tile map size (72, 40), although that does not show properly in the editor. These values can be changed for a smaller region size, which can result in faster processing times, but this is not best advised for our chosen scenario, due to the fact that not all cells in the current tile map will be covered this way.
\end{itemize}

\begin{figure}[H]
    \centering
    \includegraphics[width=0.8\textwidth]{Images/default-poisson-result.png}
    \caption{A level layout set with the default number of rejection samples (8). This level took 222ms to create.}
    \label{fig:poisson1}
\end{figure}

\begin{figure}[H]
    \centering
    \includegraphics[width=0.8\textwidth]{Images/poisson-3-samples.png}
    \caption{A level layout set with the number of rejection samples set to 3 instead of 8. There are a somewhat smaller number of painted tiles in this level than the level in Figure \ref{fig:poisson1}. This level took 87ms to create.}
    \label{fig:poisson2}
\end{figure}

\begin{figure}[H]
    \centering
    \includegraphics[width=0.8\textwidth]{Images/poisson-1-sample.png}
    \caption{A level layout set with the number of rejection samples set to 1 instead of 8. This level barely has any cells painted on it, certainly far less than the levels shown in Figures \ref{fig:poisson1} and \ref{fig:poisson2}. This level took 3ms to create.}
    \label{fig:poisson3}
\end{figure}

\begin{figure}[H]
    \centering
    \includegraphics[width=0.8\textwidth]{Images/poisson-radius-1.564.png}
    \caption{A level layout set with the radius (``point\_radius") to 1.564 instead of the default 1.0. As you can see, this results in further spaced-apart cells and more empty space. This level took 135ms to create.}
    \label{fig:poisson4}
\end{figure}

\begin{figure}[H]
    \centering
    \includegraphics[width=0.8\textwidth]{Images/poisson-buildings-0.5.png}
    \caption{A level layout set with the probability of buildings being painted over trees set to 0.5 instead of the default 0.125. As you can see, this results in more building tiles being painted in the tile map than usual. This level took 293ms to create.}
    \label{fig:poisson5}
\end{figure}
\chapter{Conclusion and Future Work} \label{Conclusion}

% The project's conclusions should list the key things that have been learnt as a consequence of engaging in your project work. For example, ``The use of overloading in C++ provides a very elegant mechanism for transparent parallelisation of sequential programs'', or ``The overheads of linear-time n-body algorithms makes them computationally less efficient than $O(n \log n)$ algorithms for systems with less than 100000 particles''. Avoid tedious personal reflections like ``I learned a lot about C++ programming...'', or ``Simulating colliding galaxies can be real fun...''. It is common to finish the report by listing ways in which the project can be taken further. This might, for example, be a plan for turning a piece of software or hardware into a marketable product, or a set of ideas for possibly turning your project into an MPhil or PhD.

To conclude, the author of this report gained a wealth of knowledge about the way some of the most popular procedural content generation algorithms work, and how they are typically integrated into working games. He also learnt how he could leverage the features of the Godot game engine for some of them; for example, the ``FastNoiseLite" class allows a Godot game developer to generate noise textures in Value, Perlin and even Simplex noise and then modify them accordingly with additional frequency settings, fractal types and cellular distance functions. By implementing them in a self-designed 2D tiled RPG scenario, he was able to get 4 procedural generation algorithms well-integrated into working games, proving Godot's technical proficiency in making these kinds of games work, and proving his own abilities as a games programmer. He was also able to compare the implementations of his chosen algorithms in such a way that the differences, in terms of both performance times and the kinds of levels they produced, could very easily be discerned. The motives of this project can be pushed still further by measuring and comparing the performances of these algorithms in Big-O notation, including even more ontogenic algorithms such as Worley Noise, the Diamond-Square algorithm, Markov Chains and Cellular Automata, as well as telelogical algorithms such as the Rain Drop algorithm and Reaction-Diffusion systems, using a larger tile map on all of these algorithms and even using a different, more intensive scenario entirely, such as a 3D walking simulator/open-world game. With procedural generation for level design, the possibilites are practically endless.

%%%%%%%%%%%%%%%%%%%%%%%%%%%%%%%%%
% References
%%%%%%%%%%%%%%%%%%%%%%%%%%%%%%%%%
\bibliographystyle{plain}
\bibliography{Bibliography/reference}
% \nocite{*}
\addcontentsline{toc}{section}{Bibliography}

%%%%%%%%%%%%%%%%%%%%%%%%%%%%%%%%%
% Appendices
%%%%%%%%%%%%%%%%%%%%%%%%%%%%%%%%%
\appendix
\chapter{Extra Information} \label{Appendix}
\section{Tables \& Test Cases}
% The appendices contain information that is peripheral to the main body of the report. Information typically included in the Appendix are things like tables, proofs, graphs, test cases or any other material that would break up the theme of the text if it appeared in the body of the report. It is necessary to include your source code listings in an appendix that is separate from the body of your written report (see the information on Program Listings below).

This section contains tables and test cases mentioned in the \hyperref[Evaluation]{Evaluation} section of this report. They are included here as they remain peripheral to the main body of this report, and would break up the theme and flow of the text if it appeared in the body, as would the source code listings featured in chapter \ref{Code} and the detailed user guide featured in chapter \ref{Guide}.

\begin{figure}[H]
    \centering
    \begin{center}
    \resizebox{\textwidth}{!}{%
    \begin{tabular}{|l|ccccc|}
    \hline
     &
      \multicolumn{5}{c|}{{\color[HTML]{FE0000} Noise type}} \\ \hline
    {\color[HTML]{00009B} Cellular Distance Function} &
      \multicolumn{1}{c|}{{\color[HTML]{FE0000} Simplex Smooth}} &
      \multicolumn{1}{c|}{{\color[HTML]{FE0000} Simplex}} &
      \multicolumn{1}{c|}{{\color[HTML]{FE0000} Perlin}} &
      \multicolumn{1}{c|}{{\color[HTML]{FE0000} Value}} &
      {\color[HTML]{FE0000} Value Cubic} \\ \hline
    {\color[HTML]{00009B} Euclidean} &
      \multicolumn{1}{c|}{\begin{tabular}[c]{@{}c@{}}81ms\\ 84ms\\ 83ms\\ AVG: 83ms\end{tabular}} &
      \multicolumn{1}{c|}{\begin{tabular}[c]{@{}c@{}}83ms\\ 88ms\\ 83ms\\ AVG: 85ms\end{tabular}} &
      \multicolumn{1}{c|}{\begin{tabular}[c]{@{}c@{}}80ms\\ 74ms\\ 74ms\\ AVG: 76ms\end{tabular}} &
      \multicolumn{1}{c|}{\begin{tabular}[c]{@{}c@{}}82ms\\ 84ms\\ 83ms\\ AVG: 83ms\end{tabular}} &
      \begin{tabular}[c]{@{}c@{}}92ms\\ 105ms\\ 74ms\\ AVG: 90ms\end{tabular} \\ \hline
    {\color[HTML]{00009B} Euclidean Squared} &
      \multicolumn{1}{c|}{\begin{tabular}[c]{@{}c@{}}81ms\\ 77ms\\ 84ms\\ AVG: 81ms\end{tabular}} &
      \multicolumn{1}{c|}{\begin{tabular}[c]{@{}c@{}}74ms\\ 91ms\\ 79ms\\ AVG: 81ms\end{tabular}} &
      \multicolumn{1}{c|}{\begin{tabular}[c]{@{}c@{}}81ms\\ 73ms\\ 80ms\\ AVG: 78ms\end{tabular}} &
      \multicolumn{1}{c|}{\begin{tabular}[c]{@{}c@{}}90ms\\ 119ms\\ 93ms\\ AVG: 101ms\end{tabular}} &
      \begin{tabular}[c]{@{}c@{}}76ms\\ 109ms\\ 82ms\\ AVG: 89ms\end{tabular} \\ \hline
    {\color[HTML]{00009B} Manhattan} &
      \multicolumn{1}{c|}{\begin{tabular}[c]{@{}c@{}}83ms\\ 107ms\\ 82ms\\ AVG: 91ms\end{tabular}} &
      \multicolumn{1}{c|}{\begin{tabular}[c]{@{}c@{}}93ms\\ 101ms\\ 87ms\\ AVG: 94ms\end{tabular}} &
      \multicolumn{1}{c|}{\begin{tabular}[c]{@{}c@{}}82ms\\ 72ms\\ 80ms\\ AVG: 78ms\end{tabular}} &
      \multicolumn{1}{c|}{\begin{tabular}[c]{@{}c@{}}80ms\\ 81ms\\ 94ms\\ AVG: 85ms\end{tabular}} &
      \begin{tabular}[c]{@{}c@{}}81ms\\ 72ms\\ 101ms\\ AVG: 85ms\end{tabular} \\ \hline
    {\color[HTML]{00009B} Hybrid (Euclidean \& Manhattan)} &
      \multicolumn{1}{c|}{\begin{tabular}[c]{@{}c@{}}77ms\\ 96ms\\ 82ms\\ AVG: 85ms\end{tabular}} &
      \multicolumn{1}{c|}{\begin{tabular}[c]{@{}c@{}}87ms\\ 122ms\\ 87ms\\ AVG: 99ms\end{tabular}} &
      \multicolumn{1}{c|}{\begin{tabular}[c]{@{}c@{}}85ms\\ 74ms\\ 83ms\\ AVG: 81ms\end{tabular}} &
      \multicolumn{1}{c|}{\begin{tabular}[c]{@{}c@{}}85ms\\ 85ms\\ 87ms\\ AVG: 86ms\end{tabular}} &
      \begin{tabular}[c]{@{}c@{}}76ms\\ 77ms\\ 77ms\\ AVG: 77ms\end{tabular} \\ \hline
    \end{tabular}%
    }
    \end{center}
    \caption{A table denoting some performance tests done with comparing the noise algorithms, the cellular distance functions and the combination pairs between them. This was done on the Noise implementation of the game; the ``tests" were simply checking how long it took to create levels on the author of this report's computer, and all the other script variables were assigned to their default values as described in chapter \ref{Design}. Each noise type and cellular distance function pair was run 3 times, with the mean time (including potential outliers) calculated afterwards \textit{to the nearest integer}. Be advised that the author of this report did these tests on his computer, so on different computers, results can, and likely \textit{will}, vary.}
    \label{fig:table1}
\end{figure}

\begin{figure}[H]
    \centering
    \begin{center}
    \resizebox{\textwidth}{!}{%
    \begin{tabular}{|l|ccccc|}
    \hline
     &
      \multicolumn{5}{c|}{{\color[HTML]{FE0000} Noise type}} \\ \hline
    {\color[HTML]{00009B} Fractal Type} &
      \multicolumn{1}{c|}{{\color[HTML]{FE0000} Simplex Smooth}} &
      \multicolumn{1}{c|}{{\color[HTML]{FE0000} Simplex}} &
      \multicolumn{1}{c|}{{\color[HTML]{FE0000} Perlin}} &
      \multicolumn{1}{c|}{{\color[HTML]{FE0000} Value}} &
      {\color[HTML]{FE0000} Value Cubic} \\ \hline
    {\color[HTML]{00009B} None} &
      \multicolumn{1}{c|}{\begin{tabular}[c]{@{}c@{}}92ms\\ 86ms\\ 123ms\\ AVG: 100ms\end{tabular}} &
      \multicolumn{1}{c|}{\begin{tabular}[c]{@{}c@{}}89ms\\ 152ms\\ 99ms\\ AVG: 113ms\end{tabular}} &
      \multicolumn{1}{c|}{\begin{tabular}[c]{@{}c@{}}84ms\\ 98ms\\ 95ms\\ AVG: 92ms\end{tabular}} &
      \multicolumn{1}{c|}{\begin{tabular}[c]{@{}c@{}}101ms\\ 86ms\\ 97ms\\ AVG: 95ms\end{tabular}} &
      \begin{tabular}[c]{@{}c@{}}77ms\\ 88ms\\ 86ms\\ AVG: 84ms\end{tabular} \\ \hline
    {\color[HTML]{00009B} FBM (Fractional Brownian Motion)} &
      \multicolumn{1}{c|}{\begin{tabular}[c]{@{}c@{}}77ms\\ 93ms\\ 87ms\\ AVG: 86ms\end{tabular}} &
      \multicolumn{1}{c|}{\begin{tabular}[c]{@{}c@{}}81ms\\ 87ms\\ 93ms\\ AVG: 87ms\end{tabular}} &
      \multicolumn{1}{c|}{\begin{tabular}[c]{@{}c@{}}73ms\\ 79ms\\ 73ms\\ AVG: 75ms\end{tabular}} &
      \multicolumn{1}{c|}{\begin{tabular}[c]{@{}c@{}}78ms\\ 137ms\\ 82ms\\ AVG: 99ms\end{tabular}} &
      \begin{tabular}[c]{@{}c@{}}68ms\\ 64ms\\ 87ms\\ AVG: 73ms\end{tabular} \\ \hline
    {\color[HTML]{00009B} Ridged} &
      \multicolumn{1}{c|}{\begin{tabular}[c]{@{}c@{}}23ms\\ 25ms\\ 23ms\\ AVG: 24ms\end{tabular}} &
      \multicolumn{1}{c|}{\begin{tabular}[c]{@{}c@{}}74ms\\ 69ms\\ 70ms\\ AVG: 71ms\end{tabular}} &
      \multicolumn{1}{c|}{\begin{tabular}[c]{@{}c@{}}15ms\\ 16ms\\ 16ms\\ AVG: 16ms\end{tabular}} &
      \multicolumn{1}{c|}{\begin{tabular}[c]{@{}c@{}}27ms\\ 28ms\\ 26ms\\ AVG: 27ms\end{tabular}} &
      \begin{tabular}[c]{@{}c@{}}14ms\\ 9ms\\ 11ms\\ AVG: 11ms\end{tabular} \\ \hline
    {\color[HTML]{00009B} Ping Pong} &
      \multicolumn{1}{c|}{\begin{tabular}[c]{@{}c@{}}59ms\\ 54ms\\ 58ms\\ AVG: 57ms\end{tabular}} &
      \multicolumn{1}{c|}{\begin{tabular}[c]{@{}c@{}}67ms\\ 77ms\\ 64ms\\ AVG: 69ms\end{tabular}} &
      \multicolumn{1}{c|}{\begin{tabular}[c]{@{}c@{}}108ms\\ 105ms\\ 111ms\\ AVG: 108ms\end{tabular}} &
      \multicolumn{1}{c|}{\begin{tabular}[c]{@{}c@{}}128ms\\ 71ms\\ 72ms\\ AVG: 90ms\end{tabular}} &
      \begin{tabular}[c]{@{}c@{}}163ms\\ 172ms\\ 164ms\\ AVG: 166ms\end{tabular} \\ \hline
    \end{tabular}%
    }
    \end{center}
    \caption{A table denoting some performance tests done with comparing the noise algorithms, the fractal types and the combination pairs between them. This was done on the Noise implementation of the game; the ``tests" were simply checking how long it took to create levels on the author of this report's computer, and all the other script variables were assigned to their default values as described in chapter \ref{Design}. Each noise type and fractal type pair was run 3 times, with the mean time (including potential outliers) calculated afterwards \textit{to the nearest integer}. Be advised that the author of this report did these tests on his computer, so on different computers, results can, and likely \textit{will}, vary.}
    \label{fig:table2}
\end{figure}

\begin{figure}[H]
    \centering
    \begin{center}
    \resizebox{\textwidth}{!}{%
    \begin{tabular}{|c|c|c|c|}
    \hline
    \begin{tabular}[c]{@{}c@{}}use\_custom\_axiom = false\\ axiom = ``OWB"\end{tabular} &
      \begin{tabular}[c]{@{}c@{}}use\_custom\_axiom = true\\ upper\_limit = 3\end{tabular} &
      \begin{tabular}[c]{@{}c@{}}use\_custom\_axiom = true\\ upper\_limit = 10\end{tabular} &
      \begin{tabular}[c]{@{}c@{}}use\_custom\_axiom = true\\ upper\_limit = 25\end{tabular} \\ \hline
    \begin{tabular}[c]{@{}c@{}}21ms\\ 17ms\\ 21ms\\ 20ms\\ 20ms\\ AVG: 20ms\end{tabular} &
      \begin{tabular}[c]{@{}c@{}}25ms (length = 2)\\ 13ms (length = 2)\\ 21ms (length = 1)\\ 16ms (length = 1)\\ 11ms (length = 2)\\ AVG: 17ms\end{tabular} &
      \begin{tabular}[c]{@{}c@{}}20ms (length = 4)\\ 11ms (length = 9)\\ 21ms (length = 8)\\ 18ms (length = 5)\\ 11ms (length = 4)\\ AVG: 16ms\end{tabular} &
      \begin{tabular}[c]{@{}c@{}}9ms (length = 25)\\ 14ms (length = 2)\\ 21ms (length = 21)\\ 20ms (length = 24)\\ 15ms (length = 14)\\ AVG: 16ms\end{tabular} \\ \hline
    \end{tabular}%
    }
    \end{center}
    \caption{A table denoting some performance tests done with comparing the lengths of axioms used in L-Systems. Obviously, this was done on the L-System implementation of the game; the ``tests" were simply checking how long it took to create levels on the author of this report's computer, as well as how long the randomly generated axioms were (where appropriate), and all the other script variables were assigned to their default values. Each of the shown settings were run 5 times, with the mean time (including potential outliers) calculated afterwards \textit{to the nearest integer}. Be advised that the author of this report did these tests on his computer, so on different computers, results can, and likely \textit{will}, vary.}
    \label{fig:table3}
\end{figure}

\begin{figure}[H]
    \centering
    \begin{center}
    \resizebox{\textwidth}{!}{%
    \begin{tabular}{|l|c|c|c|c|}
    \hline
    rejection\_samples &
      3 &
      8 &
      13 &
      18 \\ \hline
    time &
      \begin{tabular}[c]{@{}c@{}}170ms\\ 103ms\\ 111ms\\ AVG: 128ms\end{tabular} &
      \begin{tabular}[c]{@{}c@{}}337ms\\ 224ms\\ 242ms\\ AVG: 268ms\end{tabular} &
      \begin{tabular}[c]{@{}c@{}}444ms\\ 392ms\\ 388ms\\ AVG: 408ms\end{tabular} &
      {\color[HTML]{FE0000} \begin{tabular}[c]{@{}c@{}}503ms\\ 505ms\\ 670ms\\ AVG: 559ms\end{tabular}} \\ \hline
    \end{tabular}%
    }
    \end{center}
    \caption{A table denoting some performance tests done with comparing the number of rejection samples used for Poisson Disk Sampling. Obviously, this was done on the Poisson Disk Sampling/Distribution implementation of the game; the ``tests" were simply checking how long it took to create levels on the author of this report's computer, and all the other script variables were assigned to their default values. Each of the shown settings were run 3 times, with the mean time (including potential outliers) calculated afterwards \textit{to the nearest integer}. The bottom cell of the rightmost column, with tests done with 18 rejection samples, is highlighted red because, while testing with 18 rejection samples, at one time the program hung without returning any cell points within 10 seconds. The test had to be retaken another time. Be advised that the author of this report did these tests on his computer, so on different computers, results can, and likely \textit{will}, vary.}
    \label{fig:table4}
\end{figure}

\begin{figure}[H]
    \centering
    \begin{center}
    \resizebox{\textwidth}{!}{%
    \begin{tabular}{|l|cccc|}
    \hline
     &
      \multicolumn{4}{c|}{{\color[HTML]{FE0000} Random Starting Points}} \\ \hline
    {\color[HTML]{00009B} Distance type} &
      \multicolumn{1}{c|}{{\color[HTML]{FE0000} 15}} &
      \multicolumn{1}{c|}{{\color[HTML]{FE0000} 20}} &
      \multicolumn{1}{c|}{{\color[HTML]{FE0000} 30}} &
      {\color[HTML]{FE0000} 40} \\ \hline
    {\color[HTML]{00009B} Euclidean distance} &
      \multicolumn{1}{c|}{\begin{tabular}[c]{@{}c@{}}393ms\\ 385ms\\ 362ms\\ AVG: 380ms\end{tabular}} &
      \multicolumn{1}{c|}{\begin{tabular}[c]{@{}c@{}}496ms\\ 504ms\\ 497ms\\ AVG: 499ms\end{tabular}} &
      \multicolumn{1}{c|}{\begin{tabular}[c]{@{}c@{}}775ms\\ 744ms\\ 723ms\\ AVG: 747ms\end{tabular}} &
      \begin{tabular}[c]{@{}c@{}}968ms\\ 970ms\\ 967ms\\ AVG: 968ms\end{tabular} \\ \hline
    {\color[HTML]{00009B} Manhattan Distance} &
      \multicolumn{1}{c|}{\begin{tabular}[c]{@{}c@{}}364ms\\ 346ms\\ 346ms\\ AVG: 352ms\end{tabular}} &
      \multicolumn{1}{c|}{\begin{tabular}[c]{@{}c@{}}441ms\\ 470ms\\ 437ms\\ AVG:449ms\end{tabular}} &
      \multicolumn{1}{c|}{\begin{tabular}[c]{@{}c@{}}645ms\\ 630ms\\ 650ms\\ AVG: 642ms\end{tabular}} &
      \begin{tabular}[c]{@{}c@{}}843ms\\ 835ms\\ 908ms\\ AVG: 862ms\end{tabular} \\ \hline
    \end{tabular}%
    }
    \end{center}
    \caption{A table denoting some performance tests done with comparing the distance calculation algorithms, the number of random starting points and the combination pairs between them. This was done on the Voronoi Cells implementation of the game, and thus the number of random starting points corresponds directly to the number of unique Voronoi cells generated for each level. The ``tests" were simply checking how long it took to create levels on the author of this report's computer. Each of the shown settings were run 3 times, with the mean time (including potential outliers) calculated afterwards \textit{to the nearest integer}. Be advised that the author of this report did these tests on his computer, so on different computers, results can, and likely \textit{will}, vary.}
    \label{fig:table5}
\end{figure}

\begin{figure}[H]
    \centering
    \begin{center}
    \resizebox{\textwidth}{!}{%
    \begin{tabular}{|c|c|c|c|}
    \hline
    L-System &
      Perlin/Simplex Noise &
      Poisson Disk Sampling/Distribution &
      Voronoi Cells \\ \hline
    \begin{tabular}[c]{@{}c@{}}24ms\\ 10ms\\ 14ms\\ 14ms\\ 17ms\\ AVG: 16ms\end{tabular} &
      \begin{tabular}[c]{@{}c@{}}78ms\\ 78ms\\ 82ms\\ 91ms\\ 75ms\\ AVG: 81ms\end{tabular} &
      \begin{tabular}[c]{@{}c@{}}176ms\\ 190ms\\ 210ms\\ 216ms\\ 193ms\\ AVG: 197ms\end{tabular} &
      \begin{tabular}[c]{@{}c@{}}502ms\\ 425ms\\ 455ms\\ 449ms\\ 442ms\\ AVG: 455ms\end{tabular} \\ \hline
    \end{tabular}%
    }
    \end{center}
    \caption{A table denoting some performance tests done with comparing all of the algorithms implemented with the chosen scenario. This was done on every single implementation implementation of the game; the ``tests" were simply checking how long it took to create levels on the author of this report's computer, and every implementation was tested with its default variable values. Each of the implementations were run 5 times, with the mean time (including potential outliers) calculated afterwards \textit{to the nearest integer}. Be advised that the author of this report did these tests on his computer, so on different computers, results can, and likely \textit{will}, vary.}
    \label{fig:table6}
\end{figure}
\chapter{User Guide} \label{Guide}
% You must provide an adequate user guide for your software. The guide should provide easily understood instructions on how to use your software. A particularly useful approach is to treat the user guide as a walk-through of a typical session, or set of sessions, which collectively display all of the features of your package. Technical details of how the package works are rarely required. Keep the guide concise and simple. The extensive use of diagrams, illustrating the package in action, can often be particularly helpful. The user guide is sometimes included as a chapter in the main body of the report, but is often better included in an appendix to the main report.

\section{Opening Godot}

To run the projects in the .zip file, extract the projects into one folder. Then open Godot 4 (all projects in the source code listings folder are Godot 4 projects, \textbf{not} Godot 3 projects). When you start Godot for the first time, the project manager should be completely empty, without any projects, as described in Figure \ref{fig:godot1}. Projects have to be imported either one-by-one (by clicking ``Import" and going to the relevant project and opening it) or by clicking "Scan", then going to a folder of Godot projects and selecting it. The projects can then be opened in the project manager and edited as needed in Godot. Click ``Scan", then go to the folder where you extracted the projects, then click the ``Select Current Folder" button, as shown in Figure \ref{fig:godot2}, and all the projects should show up in the editor, as shown in Figure \ref{fig:godot3}. You can then double click on any one project (or click on it once and click the ``Edit" button) to open it in the Godot editor, an example of which is shown in Figure \ref{fig:godot4}. Alternatively, to run the project itself without opening the editor, using the currently saved values for exported script variables where appropriate, click on the project \textit{once} and click the ``Run" button.

\begin{figure}[H]
    \centering
    \includegraphics[width=\textwidth]{Images/open_godot.png}
    \caption{The Godot editor, when it is opened for the first time, does not show any projects in the editor (the Steam version bundles several example projects). Projects need to be imported either one-by-one or by scanning a folder of Godot projects.}
    \label{fig:godot1}
\end{figure}

\begin{figure}[H]
    \centering
    \includegraphics[width=\textwidth]{Images/scan-folder.png}
    \caption{You can click the "Scan" button in the project manager (in Figure \ref{fig:godot1}), then go to the relevant folder where your project are in Godot's built-in file explorer. Here, I have extracted my artefacts into a separate folder called ``Scan this folder" as an example.}
    \label{fig:godot2}
\end{figure}

\begin{figure}[H]
    \centering
    \includegraphics[width=\textwidth]{Images/projects-scanned.png}
    \caption{Once some Godot projects have been imported into the project manager, you should be able to easily view the list and double-click on any one of the projects to edit them, which will open the editor after closing the project manager. You could also click the ``Edit" button, or click ``Run" to run the game without having to open the editor itself.}
    \label{fig:godot3}
\end{figure}

\begin{figure}[H]
    \centering
    \includegraphics[width=\textwidth]{Images/godot-editor.png}
    \caption{The Godot editor open with the Voronoi cells project as an example. A visual description of the editor's contents is in chapter \ref{editor}.}
    \label{fig:godot4}
\end{figure}

\section{The Godot Editor} \label{editor}

As you open up the Godot editor, you will see the main scene view in the center, as shown in Figure \ref{fig:godot4}, using the Voronoi cells implementation as an example. The left-hand side shows the scene tree at the top, and the file system (from the root folder of the project) at the bottom. Meanwhile, the right hand side shows the currently selected node's export variables, \textit{including the custom export variables} defined in the node's script file, and two other tabs, ``Node" (which shows a list of signals for the scene that could be called in a script) and ``History" (which shows the sequence of recent actions performed on the scene during the current session). Above this is also a set of buttons which can be used for playing the project and/or the current scene. I go over how to run the current project in chapter \ref{runproject}.

\section{Custom Export Variables} 

When you click on some of the scenes in the projects, there may be some "exported" variables from scripts that are visible to you in the editor (examples include the "Distance" and "Random Starting Points" variables in the Voronoi Cells project). You can hover over the variable names in the editor and it will show a brief description of what the variable correlates to in the script. I go over the different export variables across all of my artefacts in this section.

\subsection{Lindenmayer System}

All of the custom export variables I defined for your use are in the child node ``LSystem" (it is saved into its own scene file, but it is cheifly a child node of the root node ``TileMap"). Open the ``TileMap" scene (if it is not already opened for you when you launch the Godot editor) and click on the ``LSystem" node in the scene tree to edit it.

\begin{enumerate}
    \item axiom
    \begin{itemize}
        \item \textbf{Type:} String
        \item \textbf{Documentation:} The starting string from which the grammar starts applying its rules. Here it may be self defined, or randomly defined when ``use\_random\_axiom" is true.
        \item \textbf{Default value:} ``OWB"
    \end{itemize}
    \item use\_random\_axiom
    \begin{itemize}
        \item \textbf{Type:} Boolean (bool) 
        \item \textbf{Documentation:} Uses a random axiom with the currently set grammar, computed upon runtime, with a length up to (but not strictly) the value of upper\_limit. For example, if upper\_limit is set to 15, the generated axiom can be 15 characters, or it can be just 5 characters.
        \item \textbf{Default value:} true
    \end{itemize}
    \item upper\_limit
    \begin{itemize}
        \item \textbf{Type:} Integer (int)
        \item \textbf{Documentation:} Defines how many characters a random axiom can have MAXIMUM. Only used when ``use\_random\_axiom" is true.
        \item \textbf{Default value:} 10
    \end{itemize}
    \item use\_custom\_ruleset
    \begin{itemize}
        \item \textbf{Type:} Boolean (bool)
        \item \textbf{Documentation:} Allows the use of a customly defined ruleset made through amending the rules array in the editor.
        \item \textbf{Default value:} false
    \end{itemize}
    \item ruleset
    \begin{itemize}
        \item \textbf{Type:} String enumeration of the following choices:
        \begin{enumerate}
            \item ``Default"
            \item ``More Buildings (IMPOSSIBLE)"
            \item ``More Trees"
            \item ``More Space"
        \end{enumerate}
        \item \textbf{Documentation:} Denotes a series of pre-defined rulesets for this L-System grammar, of alphabet O (blank space), W (trees and fauna) and B (buildings), that can be chosen and then used on runtime. Can choose between a default ruleset, a ruleset that produces more buildings, a ruleset that produces more trees and a ruleset that produces more empty space.
        \item \textbf{Default value:} ``Default"
    \end{itemize}
    \item rules
    \begin{itemize}
        \item \textbf{Type:} Array of dictionaries
        \item \textbf{Documentation:} The set of rules that the L-System grammar uses. Shows the ``default" ruleset in the Godot editor for the user to see. If ``use\_custom\_ruleset" is set to true, this array can be edited with a custom defined ruleset that will be used on runtime, so long as it adheres to the alphabet of O (blank space), W (trees and fauna) and B (buildings).
        \item \textbf{Additional information:} The ``\_get\_ruleset" function uses the String value in ``ruleset" to set the value for ``rules" on runtime, \textit{if} ``use\_custom\_ruleset" is false (which it \textit{is} by default).
        \item \textbf{Default value:} The ``Default" grammar, as shown in Figure \ref{fig:defaultgrammar}.
    \end{itemize}
\end{enumerate}

\begin{figure}[H]
    \centering
    \begin{lstlisting}
    [
    	{
    		"from": "O",
    		"to": "OWO"
    	},
    	{
    		"from": "W",
    		"to": "WB"
    	},
    	{
    		"from": "B",
    		"to": "BWO"
    	}
    ]
    \end{lstlisting}
    \caption{The ``Default" grammar used for the ``rules" export variable, stored in the constant ``DEFAULT" in l\_system.gd. See l\_system.gd for the other grammars represented as arrays of dictionaries.}
    \label{fig:defaultgrammar}
\end{figure}

\subsection{Perlin/Simplex Noise}

\begin{enumerate}
    \item noise\_type
    \begin{itemize}
        \item \textbf{Type:} String enumeration of the following choices:
        \begin{enumerate}
            \item ``Perlin"
            \item ``Simplex"
            \item ``Simplex Smooth"
            \item ``Value"
            \item ``Value Cubic"
        \end{enumerate}
        \item \textbf{Documentation:} Defines the type of noise generation algorithm to use. Equates to the ``noise\_type" property in FastNoiseLite.
        \item \textbf{Default value:} ``Value Cubic"
    \end{itemize}
    \item fractal\_type
    \begin{itemize}
        \item \textbf{Type:} Enumeration of the following choices:
        \begin{enumerate}
            \item ``Fractal None"
            \item ``Fractal FBM"
            \item ``Fractal Ridged"
            \item ``Fractal Ping Pong"
        \end{enumerate}
        \item \textbf{Documentation:} Defines the type of method used to combine octaves of a noise image into a fractal. Directly equates to the FractalType enumeration in FastNoiseLite.
        \item \textbf{Default value:} ``Fractal None" 
    \end{itemize}
    \item cellular\_distance\_type
    \begin{itemize}
        \item \textbf{Type:} Enumeration of the following choices:
        \begin{enumerate}
            \item ``Distance Euclidean"
            \item ``Distance Euclidean Squared"
            \item ``Distance Manhattan"
            \item ``Distance Hybrid"
        \end{enumerate}
        \item \textbf{Documentation:} Defines the function used to calculate the distance between the nearest/second-nearest point(s). Directly equates to the CellularDistanceFunction enumeration in FastNoiseLite.
        \item \textbf{Default value:} ``Distance Euclidean"
    \end{itemize}
    \item noise\_frequency
    \begin{itemize}
        \item \textbf{Type:} Floating point number (float) between 0.0 and 1.0 inclusive
        \item \textbf{Documentation:} Defines the frequency of the generated noise, the higher the frequency, the rougher and more granular the noise.
        \item \textbf{Additional information:} The default value for ``frequency" in the ``FastNoiseClass" is 0.01, resulting in very smooth and distinct noise. 
        \item \textbf{Default value:} 0.894
    \end{itemize}
    \item tree\_cap
    \begin{itemize}
        \item \textbf{Type:} Floating point number (float) between -1.0 and 1.0 inclusive
        \item \textbf{Documentation:} Defines the upper limit to set for painting a tree tile on a specific noise pixel. If the value returned by the ``get\_noise\_2d" method (in FastNoiseLite) is smaller than this, then it gets painted.
        \item \textbf{Default value:} -0.048
    \end{itemize}
    \item building\_cap
    \begin{itemize}
        \item \textbf{Type:} Floating point number (float) between -1.0 and 1.0 inclusive
        \item \textbf{Documentation:} Defines the upper limit to set for painting a building tile on a specific noise pixel. If the value returned by the ``get\_noise\_2d" method (in FastNoiseLite) is smaller than this, then it gets painted. If the value of ``building\_cap" is smaller than ``tree\_cap," then decide whether or not to paint a building cell there with ``building\_overtakes\_tree."
        \item \textbf{Default value:} -0.252
    \end{itemize}
    \item building\_overtakes\_tree
    \begin{itemize}
        \item \textbf{Type:} Floating point number (float) between 0.0 and 0.5 inclusive
        \item \textbf{Documentation:} Only used when ``building\_cap" is smaller than ``tree\_cap." Determines the probability that a building tile would be painted in a cell where a tree tile was, or could be, also painted. Whether or not the cell actually is painted over is decided on computation time.
        \item \textbf{Default value:} 0.12
    \end{itemize}
\end{enumerate}

\subsection{Poisson Disk Sampling}

\begin{enumerate}
    \item paint\_building\_probability
    \begin{itemize}
        \item \textbf{Type:} Floating point number (float) between 0.0 and 1.0 inclusive
        \item \textbf{Documentation:} The probability that a building gets painted at a cell in lieu of a tree. The higher this probability, the more likely a building tile gets painted instead of a tree tile. 
        \item \textbf{Default value:} 0.125
    \end{itemize}
    \item point\_radius
    \begin{itemize}
        \item \textbf{Type:} Floating point number (float) between 0.5 and 2.5 inclusive
        \item \textbf{Documentation:} The radius value used to measure distances between points for the algorithm. The longer the radius, the further apart points are during the algorithm's processing, and the further apart painted cells are in the game.
        \item \textbf{Default value:} 1.0
    \end{itemize}
    \item region\_size
    \begin{itemize}
        \item \textbf{Type:} Vector2
        \item \textbf{Documentation:} The size of the region in which the algorithm is performed. Set to the "default" tile map size (72, 40) in the script, shown as (0, 0) in the Godot editor. Can be changed to use a smaller region for the algorithm itself, of course resulting in less cells covered within the boundaries set for this game, though the algorithm will perform faster due to less cells being checked.
        \item \textbf{Default value:} 
        \begin{itemize}
            \item \textbf{x}: The value in ``x\_tile\_range" (72)
            \item \textbf{y}: The value in ``y\_tile\_range" (40)
        \end{itemize}
    \end{itemize}
    \item rejection\_samples
    \begin{itemize}
        \item \textbf{Type:} Integer (int) between 1 andc 50 inclusive
        \item \textbf{Documentation:} The maximum number of times a cell is checked before it is ignored. A cell can be accepted and painted on before this maximum number is reached. The higher this value, the more times a cell is checked, therefore the higher the algorithm's processing time.
        \item \textbf{Default value:} 8
    \end{itemize}
\end{enumerate}

\subsection{Voronoï Cells}

\begin{enumerate}
    \item distance
    \begin{itemize}
        \item \textbf{Type:} String enumeration of the following choices:
        \begin{enumerate}
            \item ``Euclidean distance"
            \item ``Manhattan distance"
        \end{enumerate}
        \item \textbf{Documentation:} Determines whether or not the Euclidean or Manhattan distance formula is used for calculation of the deltas between points within Voronoi cells.
        \item \textbf{Default value:} ``Manhattan distance"
    \end{itemize}
    \item random\_starting\_points
    \begin{itemize}
        \item \textbf{Type:} Integer (int) between 15 and 40 inclusive
        \item \textbf{Documentation:} Determines the number of points randomly picked from at the start. Therefore, it also determines the number of cells in our Voronoi tesselation.
        \item \textbf{Default value:} 20
    \end{itemize}
\end{enumerate}

\subsection{The Basic L-System Demo Used to Create the Screenshots in Chapter \ref{alglsys}}

There is only one export variable for this: ``choices", which allows you to choose which one of the three provided rulesets to use. ``choices" is the default ruleset, and either ``deterministic" or ``basic" can be chosen. It is in the ``DemoNode" scene, the only scene of this Godot project. Quoting the documentation comment, this export variable ``Allows you to decide which ruleset to use. See the script file for the sources of said rulesets." The ruleset is assigned with the ``set\_values" function.

\newpage

\section{Running the Godot Projects} \label{runproject}

To \textit{run} the current project in the Godot editor, go to the bar above the Inspector, Node and History tabs on the right-hand side. You will find a \faPlay{} button which will play the main scene of the project (in my artefacts, the main scenes have already been set; if it were not already set, you would have been asked to set one). If closing the window to stop the project does not work, hit the \faStop{} button to end it. 

As described in section \ref{commonalities}, only the close button of both the popup dialogs in the game seems to work properly for the moment, but this does not adversely affect the game functioning properly, nor does it adversely affect this project, so this is trivial.

\chapter{Source Code} \label{Code}

\localtableofcontents{}

\section{Instructions}
% Complete source code listings must be submitted as an appendix to the report. The project source codes are usually spread out over several files/units. You should try to help the reader to navigate through your source code by providing a ``table of contents'' (titles of these files/units and one line descriptions). The first page of the program listings folder must contain the following statement certifying the work as your own: ``I verify that I am the sole author of the programs contained in this folder, except where explicitly stated to the contrary''. Your (typed) signature and the date should follow this statement.

% All work on programs must stop once the code is submitted to KEATS. You are required to keep safely several copies of this version of the program and you must use one of these copies in the project examination. Your examiners may ask to see the last-modified dates of your program files, and may ask you to demonstrate that the program files you use in the project examination are identical to the program files you have uploaded to KEATS. Any attempt to demonstrate code that is not included in your submitted source listings is an attempt to cheat; any such attempt will be reported to the KCL Misconduct Committee.

% \textbf{You may find it easier to firstly generate a PDF of your source code using a text editor and then merge it to the end of your report. There are many free tools available that allow you to merge PDF files.}

If needed, use the table of contents provided to browse through the code listings in this section. Each listing folder will have a description, a link to its public GitHub repository, and a listing for each readable source file. Use the .zip folder containing the project artefacts to edit and run them in Godot.

\textit{``\textbf{I verify that I am the sole author of the programs contained in this folder, except where explicitly stated to the contrary.}"}
\textit{- Zishan Rahman, 21\textsuperscript{st} April 2023}

\section{GD4LSystemRPG}

% \localtableofcontents{}

\subsection{.gitattributes}

\begin{lstlisting}
# Normalize EOL for all files that Git considers text files.
* text=auto eol=lf
\end{lstlisting}

\subsection{.gitignore}

\begin{lstlisting}
# Godot 4+ specific ignores
.godot/
\end{lstlisting}

\subsection{project.godot}

\begin{lstlisting}
; Engine configuration file.
; It's best edited using the editor UI and not directly,
; since the parameters that go here are not all obvious.
;
; Format:
;   [section] ; section goes between []
;   param=value ; assign values to parameters

config_version=5

[application]

config/name="LSystem RPG"
run/main_scene="res://tile_map.tscn"
config/features=PackedStringArray("4.0", "Forward Plus")
config/icon="res://icon.svg"

[display]

window/size/viewport_height=640

[rendering]

environment/defaults/default_clear_color=Color(0, 0, 0, 1)
\end{lstlisting}

\subsection{l\textunderscore{}system.tscn}

\begin{lstlisting}
[gd_scene load_steps=2 format=3 uid="uid://d0v18e7ms571f"]

[ext_resource type="Script" path="res://l_system.gd" id="1_elydp"]

[node name="LSystem" type="Node"]
script = ExtResource("1_elydp")
\end{lstlisting}

\subsection{l\textunderscore{}system.gd}

\begin{lstlisting}
extends TileMap

@onready var l_system: LSystem = $LSystem

var x_tile_range: int = ProjectSettings.get_setting("display/window/size/viewport_width") / tile_set.tile_size.x
var y_tile_range: int = ProjectSettings.get_setting("display/window/size/viewport_height") / tile_set.tile_size.y

const PLAYER_SPRITE: Vector2i = Vector2i(24, 7)
var player_placement_cell: Vector2i
const rings: Array[Vector2i] = [
	Vector2i(43, 6),
	Vector2i(44, 6),
	Vector2i(45, 6),
	Vector2i(46, 6)
]
var ring_placement_cell: Vector2i

# Called when the node enters the scene tree for the first time.
func _ready() -> void:
	randomize()
	var start_time: float = Time.get_ticks_msec()
	l_system.paint()
	place_player()
	place_ring()
	var new_time: float = Time.get_ticks_msec() - start_time
	print("Time taken: " + str(new_time) + "ms")
	$AcceptDialog.dialog_text = "You're a hollow Golem who seeks the ultimate treasure; a ring that's got something on top of it. It's somewhere in this large village and barely visible to your naked eyes, which took us " + str(new_time) + " milliseconds to generate (" + str(new_time / 1000.0) + " seconds), but you'll stop at nothing to get what you want. You can chow down every tree and fauna that stands in your way of the ring, but your Achilles heel is any bricks and mortar, which WILL make you stop at your tracks. Since it's easy to get lost in here, we'll tell you that you're in position " + str(player_placement_cell) + " in this big village of size " + str(Vector2i(x_tile_range, y_tile_range)) + ". The ring is in position " + str(ring_placement_cell) + ", but it is YOUR job to traverse the village, chow down the trees and get it for yourself, so are you ready to attain the treasure that is rightfully yours?!"
	$AcceptDialog.visible = true
	$AcceptDialog.confirmed.connect(_on_AcceptDialog_closed)
	$AcceptDialog.canceled.connect(_on_AcceptDialog_closed)
	$WinDialog.confirmed.connect(_on_WinDialog_confirmed)
	$WinDialog.canceled.connect(_on_WinDialog_canceled)
	get_tree().paused = true

func _on_WinDialog_confirmed() -> void:
	get_tree().reload_current_scene()

func _on_WinDialog_canceled() -> void:
	get_tree().quit()

func _on_AcceptDialog_closed() -> void:
	$AcceptDialog.visible = false
	get_tree().paused = false

func _get_random_placement_cell() -> Vector2i:
	return Vector2i(randi() % x_tile_range, randi() % y_tile_range)

func place_player() -> void:
	player_placement_cell = _get_random_placement_cell()
	while get_used_cells(0).has(player_placement_cell):
		player_placement_cell = _get_random_placement_cell()
	set_cell(0, player_placement_cell, 0, PLAYER_SPRITE)

func place_ring() -> void:
	ring_placement_cell = _get_random_placement_cell()
	while get_used_cells(0).has(ring_placement_cell):
		ring_placement_cell = _get_random_placement_cell()
	set_cell(0, ring_placement_cell, 0, rings.pick_random())

func _is_not_out_of_bounds(cell: Vector2i) -> bool:
	return cell.x >= 0 and cell.x < x_tile_range and cell.y >= 0 and cell.y < y_tile_range

func _physics_process(_delta: float) -> void:
	var previous_cell: Vector2i = player_placement_cell
	var direction: Vector2i = Vector2i.ZERO
	if Input.is_action_pressed("ui_up"): direction = Vector2i.UP
	elif Input.is_action_pressed("ui_down"): direction = Vector2i.DOWN
	elif Input.is_action_pressed("ui_left"): direction = Vector2i.LEFT
	elif Input.is_action_pressed("ui_right"): direction = Vector2i.RIGHT
	var new_placement_cell: Vector2i = player_placement_cell + direction
	if (not get_used_cells(0).has(new_placement_cell) or l_system.trees.has(get_cell_atlas_coords(0, new_placement_cell)) or new_placement_cell == ring_placement_cell) and _is_not_out_of_bounds(new_placement_cell):
		player_placement_cell = new_placement_cell
		set_cell(0, previous_cell, 0) # deletes contents of previous cell (atlas_coords = Vector2i(-1, -1))
		set_cell(0, player_placement_cell, 0, PLAYER_SPRITE)
		if player_placement_cell == ring_placement_cell:
			$WinDialog.visible = true
			get_tree().paused = true

# ALGORITHM AND CUSTOM EXPORT VARIABLES ARE IN LSYSTEM NODE
\end{lstlisting}

\subsection{tile\textunderscore{}map.tscn}

\begin{lstlisting}
[gd_scene load_steps=6 format=3 uid="uid://bwhvtqld3yo8m"]

[ext_resource type="TileSet" uid="uid://c168x78r0tful" path="res://Tiles.tres" id="1_l3nwg"]
[ext_resource type="Script" path="res://tile_map.gd" id="2_wrxl8"]
[ext_resource type="PackedScene" uid="uid://d0v18e7ms571f" path="res://l_system.tscn" id="3_ktw1n"]
[ext_resource type="PackedScene" uid="uid://cau5jgogdnf53" path="res://accept_dialog.tscn" id="4_060oh"]
[ext_resource type="PackedScene" uid="uid://b5q8ovcigrvyr" path="res://win_dialog.tscn" id="5_3s48a"]

[node name="TileMap" type="TileMap"]
tile_set = ExtResource("1_l3nwg")
format = 2
layer_0/name = "Things"
script = ExtResource("2_wrxl8")

[node name="LSystem" parent="." instance=ExtResource("3_ktw1n")]

[node name="AcceptDialog" parent="." instance=ExtResource("4_060oh")]

[node name="WinDialog" parent="." instance=ExtResource("5_3s48a")]
\end{lstlisting}

\subsection{tile\textunderscore{}map.gd}

\begin{lstlisting}
extends TileMap

@onready var l_system: LSystem = $LSystem

var x_tile_range: int = ProjectSettings.get_setting("display/window/size/viewport_width") / tile_set.tile_size.x
var y_tile_range: int = ProjectSettings.get_setting("display/window/size/viewport_height") / tile_set.tile_size.y

const PLAYER_SPRITE: Vector2i = Vector2i(24, 7)
var player_placement_cell: Vector2i
const rings: Array[Vector2i] = [
	Vector2i(43, 6),
	Vector2i(44, 6),
	Vector2i(45, 6),
	Vector2i(46, 6)
]
var ring_placement_cell: Vector2i

# Called when the node enters the scene tree for the first time.
func _ready() -> void:
	randomize()
	var start_time: float = Time.get_ticks_msec()
	l_system.paint()
	place_player()
	place_ring()
	var new_time: float = Time.get_ticks_msec() - start_time
	print("Time taken: " + str(new_time) + "ms")
	$AcceptDialog.dialog_text = "You're a hollow Golem who seeks the ultimate treasure; a ring that's got something on top of it. It's somewhere in this large village and barely visible to your naked eyes, which took us " + str(new_time) + " milliseconds to generate (" + str(new_time / 1000.0) + " seconds), but you'll stop at nothing to get what you want. You can chow down every tree and fauna that stands in your way of the ring, but your Achilles heel is any bricks and mortar, which WILL make you stop at your tracks. Since it's easy to get lost in here, we'll tell you that you're in position " + str(player_placement_cell) + " in this big village of size " + str(Vector2i(x_tile_range, y_tile_range)) + ". The ring is in position " + str(ring_placement_cell) + ", but it is YOUR job to traverse the village, chow down the trees and get it for yourself, so are you ready to attain the treasure that is rightfully yours?!"
	$AcceptDialog.visible = true
	$AcceptDialog.confirmed.connect(_on_AcceptDialog_closed)
	$AcceptDialog.canceled.connect(_on_AcceptDialog_closed)
	$WinDialog.confirmed.connect(_on_WinDialog_confirmed)
	$WinDialog.canceled.connect(_on_WinDialog_canceled)
	get_tree().paused = true

func _on_WinDialog_confirmed() -> void:
	get_tree().reload_current_scene()

func _on_WinDialog_canceled() -> void:
	get_tree().quit()

func _on_AcceptDialog_closed() -> void:
	$AcceptDialog.visible = false
	get_tree().paused = false

func _get_random_placement_cell() -> Vector2i:
	return Vector2i(randi() % x_tile_range, randi() % y_tile_range)

func place_player() -> void:
	player_placement_cell = _get_random_placement_cell()
	while get_used_cells(0).has(player_placement_cell):
		player_placement_cell = _get_random_placement_cell()
	set_cell(0, player_placement_cell, 0, PLAYER_SPRITE)

func place_ring() -> void:
	ring_placement_cell = _get_random_placement_cell()
	while get_used_cells(0).has(ring_placement_cell):
		ring_placement_cell = _get_random_placement_cell()
	set_cell(0, ring_placement_cell, 0, rings.pick_random())

func _is_not_out_of_bounds(cell: Vector2i) -> bool:
	return cell.x >= 0 and cell.x < x_tile_range and cell.y >= 0 and cell.y < y_tile_range

func _physics_process(_delta: float) -> void:
	var previous_cell: Vector2i = player_placement_cell
	var direction: Vector2i = Vector2i.ZERO
	if Input.is_action_pressed("ui_up"): direction = Vector2i.UP
	elif Input.is_action_pressed("ui_down"): direction = Vector2i.DOWN
	elif Input.is_action_pressed("ui_left"): direction = Vector2i.LEFT
	elif Input.is_action_pressed("ui_right"): direction = Vector2i.RIGHT
	var new_placement_cell: Vector2i = player_placement_cell + direction
	if (not get_used_cells(0).has(new_placement_cell) or l_system.trees.has(get_cell_atlas_coords(0, new_placement_cell)) or new_placement_cell == ring_placement_cell) and _is_not_out_of_bounds(new_placement_cell):
		player_placement_cell = new_placement_cell
		set_cell(0, previous_cell, 0) # deletes contents of previous cell (atlas_coords = Vector2i(-1, -1))
		set_cell(0, player_placement_cell, 0, PLAYER_SPRITE)
		if player_placement_cell == ring_placement_cell:
			$WinDialog.visible = true
			get_tree().paused = true

# ALGORITHM AND CUSTOM EXPORT VARIABLES ARE IN LSYSTEM NODE
\end{lstlisting}

\subsection{accept\_dialog.tscn}

\begin{lstlisting}
[gd_scene format=3 uid="uid://cau5jgogdnf53"]

[node name="AcceptDialog" type="AcceptDialog"]
title = "Tree-Munching Time!"
position = Vector2i(326, 100)
size = Vector2i(500, 421)
mouse_passthrough = true
ok_button_text = "Bring it on!"
dialog_text = "You're a hollow Golem who seeks the ultimate treasure; a ring that's got something on top of it. It's somewhere in this large village and barely visible to your naked eyes, but you'll stop at nothing to get what you want. You can chow down every tree and fauna that stands in your way of the ring, but your Achilles heel is any bricks and mortar, which will make you stop at your tracks. Are you ready to attain your treasure?w Golem in a black-and-white world, in search for your most desired treasure. It's a ring with something on top of it. And you'll stop at nothing to get what you want. You can chow down every tree and fauna that stands in your way of the ring, but your Achilles heel is any bricks and mortar, which will make you stop at your tracks. Are you ready to attain the treasure that is rightfully yours?!"
dialog_autowrap = true
\end{lstlisting}

\subsection{win\_dialog.tscn}

\begin{lstlisting}
[gd_scene format=3 uid="uid://b5q8ovcigrvyr"]

[node name="WinDialog" type="ConfirmationDialog"]
title = "You Found the Treasure!"
position = Vector2i(326, 100)
size = Vector2i(500, 421)
mouse_passthrough = true
ok_button_text = "Get Me a New Village"
dialog_text = "You found your treasure! Well done, you!

Would you like to travel to a new village in the hopes of finding another ring? Or would you like to take your treasure home now?"
dialog_autowrap = true
cancel_button_text = "Get Me Out of Here"
\end{lstlisting}

\subsection{icon.svg.import}

\begin{lstlisting}
[remap]

importer="texture"
type="CompressedTexture2D"
uid="uid://b45qexb3wmhym"
path="res://.godot/imported/icon.svg-218a8f2b3041327d8a5756f3a245f83b.ctex"
metadata={
"vram_texture": false
}

[deps]

source_file="res://icon.svg"
dest_files=["res://.godot/imported/icon.svg-218a8f2b3041327d8a5756f3a245f83b.ctex"]

[params]

compress/mode=0
compress/high_quality=false
compress/lossy_quality=0.7
compress/hdr_compression=1
compress/normal_map=0
compress/channel_pack=0
mipmaps/generate=false
mipmaps/limit=-1
roughness/mode=0
roughness/src_normal=""
process/fix_alpha_border=true
process/premult_alpha=false
process/normal_map_invert_y=false
process/hdr_as_srgb=false
process/hdr_clamp_exposure=false
process/size_limit=0
detect_3d/compress_to=1
svg/scale=1.0
editor/scale_with_editor_scale=false
editor/convert_colors_with_editor_theme=false
\end{lstlisting}

\subsection{roguelikeSheet\textunderscore{}transparent.png.import}

\begin{lstlisting}
[remap]

importer="texture"
type="CompressedTexture2D"
uid="uid://13ktp0qup5xb"
path="res://.godot/imported/roguelikeSheet_transparent.png-22f6b70da04549e371d1f15fe9d96005.ctex"
metadata={
"vram_texture": false
}

[deps]

source_file="res://roguelikeSheet_transparent.png"
dest_files=["res://.godot/imported/roguelikeSheet_transparent.png-22f6b70da04549e371d1f15fe9d96005.ctex"]

[params]

compress/mode=0
compress/high_quality=false
compress/lossy_quality=0.7
compress/hdr_compression=1
compress/normal_map=0
compress/channel_pack=0
mipmaps/generate=false
mipmaps/limit=-1
roughness/mode=0
roughness/src_normal=""
process/fix_alpha_border=true
process/premult_alpha=false
process/normal_map_invert_y=false
process/hdr_as_srgb=false
process/hdr_clamp_exposure=false
process/size_limit=0
detect_3d/compress_to=1
\end{lstlisting}

\section{GD4VoronoiRPG}

% \localtableofcontents{}

\subsection{.gitattributes}

\begin{lstlisting}
# Normalize EOL for all files that Git considers text files.
* text=auto eol=lf
\end{lstlisting}

\subsection{.gitignore}

\begin{lstlisting}
# Godot 4+ specific ignores
.godot/
\end{lstlisting}

\subsection{project.godot}

\begin{lstlisting}
; Engine configuration file.
; It's best edited using the editor UI and not directly,
; since the parameters that go here are not all obvious.
;
; Format:
;   [section] ; section goes between []
;   param=value ; assign values to parameters

config_version=5

[application]

config/name="Voronoi Cells"
run/main_scene="res://tile_map.tscn"
config/features=PackedStringArray("4.0", "Forward Plus")
config/icon="res://icon.svg"

[display]

window/size/viewport_height=640

[input]

reset_position={
"deadzone": 0.5,
"events": [Object(InputEventKey,"resource_local_to_scene":false,"resource_name":"","device":-1,"window_id":0,"alt_pressed":false,"shift_pressed":false,"ctrl_pressed":false,"meta_pressed":false,"pressed":false,"keycode":71,"physical_keycode":0,"key_label":0,"unicode":103,"echo":false,"script":null)
, Object(InputEventMouseButton,"resource_local_to_scene":false,"resource_name":"","device":-1,"window_id":0,"alt_pressed":false,"shift_pressed":false,"ctrl_pressed":false,"meta_pressed":false,"button_mask":2,"position":Vector2(75, 12),"global_position":Vector2(78, 44),"factor":1.0,"button_index":2,"pressed":true,"double_click":false,"script":null)
]
}

[rendering]

environment/defaults/default_clear_color=Color(0, 0, 0, 1)
\end{lstlisting}

\subsection{tile\textunderscore{}map.tscn}

\begin{lstlisting}
[gd_scene load_steps=7 format=3 uid="uid://d6lxnr5bdh1w"]

[ext_resource type="Texture2D" uid="uid://cpign73sfbsrt" path="res://monochrome_packed.png" id="1_o183d"]
[ext_resource type="Script" path="res://tile_map.gd" id="2_lf4lw"]
[ext_resource type="PackedScene" path="res://accept_dialog.tscn" id="3_y08lj"]
[ext_resource type="PackedScene" path="res://win_dialog.tscn" id="4_fkys0"]

[sub_resource type="TileSetAtlasSource" id="TileSetAtlasSource_6h0bd"]
texture = ExtResource("1_o183d")
0:0/0 = 0
1:0/0 = 0
2:0/0 = 0
3:0/0 = 0
4:0/0 = 0
5:0/0 = 0
6:0/0 = 0
7:0/0 = 0
8:0/0 = 0
9:0/0 = 0
10:0/0 = 0
11:0/0 = 0
12:0/0 = 0
13:0/0 = 0
14:0/0 = 0
15:0/0 = 0
16:0/0 = 0
17:0/0 = 0
18:0/0 = 0
19:0/0 = 0
20:0/0 = 0
21:0/0 = 0
22:0/0 = 0
23:0/0 = 0
24:0/0 = 0
25:0/0 = 0
26:0/0 = 0
27:0/0 = 0
28:0/0 = 0
29:0/0 = 0
30:0/0 = 0
31:0/0 = 0
32:0/0 = 0
33:0/0 = 0
34:0/0 = 0
35:0/0 = 0
36:0/0 = 0
37:0/0 = 0
38:0/0 = 0
39:0/0 = 0
40:0/0 = 0
41:0/0 = 0
42:0/0 = 0
43:0/0 = 0
44:0/0 = 0
45:0/0 = 0
46:0/0 = 0
47:0/0 = 0
48:0/0 = 0
0:1/0 = 0
1:1/0 = 0
2:1/0 = 0
3:1/0 = 0
4:1/0 = 0
5:1/0 = 0
6:1/0 = 0
7:1/0 = 0
8:1/0 = 0
9:1/0 = 0
10:1/0 = 0
11:1/0 = 0
12:1/0 = 0
13:1/0 = 0
14:1/0 = 0
15:1/0 = 0
16:1/0 = 0
17:1/0 = 0
18:1/0 = 0
19:1/0 = 0
20:1/0 = 0
21:1/0 = 0
22:1/0 = 0
23:1/0 = 0
24:1/0 = 0
25:1/0 = 0
26:1/0 = 0
27:1/0 = 0
28:1/0 = 0
29:1/0 = 0
30:1/0 = 0
31:1/0 = 0
32:1/0 = 0
33:1/0 = 0
34:1/0 = 0
35:1/0 = 0
36:1/0 = 0
37:1/0 = 0
38:1/0 = 0
39:1/0 = 0
40:1/0 = 0
41:1/0 = 0
42:1/0 = 0
43:1/0 = 0
44:1/0 = 0
45:1/0 = 0
46:1/0 = 0
47:1/0 = 0
48:1/0 = 0
0:2/0 = 0
1:2/0 = 0
2:2/0 = 0
3:2/0 = 0
4:2/0 = 0
5:2/0 = 0
6:2/0 = 0
7:2/0 = 0
8:2/0 = 0
9:2/0 = 0
10:2/0 = 0
11:2/0 = 0
12:2/0 = 0
13:2/0 = 0
14:2/0 = 0
15:2/0 = 0
16:2/0 = 0
17:2/0 = 0
18:2/0 = 0
19:2/0 = 0
20:2/0 = 0
21:2/0 = 0
22:2/0 = 0
23:2/0 = 0
24:2/0 = 0
25:2/0 = 0
26:2/0 = 0
27:2/0 = 0
28:2/0 = 0
29:2/0 = 0
30:2/0 = 0
31:2/0 = 0
32:2/0 = 0
33:2/0 = 0
34:2/0 = 0
35:2/0 = 0
36:2/0 = 0
37:2/0 = 0
38:2/0 = 0
39:2/0 = 0
40:2/0 = 0
41:2/0 = 0
42:2/0 = 0
43:2/0 = 0
44:2/0 = 0
45:2/0 = 0
46:2/0 = 0
47:2/0 = 0
48:2/0 = 0
0:3/0 = 0
1:3/0 = 0
2:3/0 = 0
3:3/0 = 0
4:3/0 = 0
5:3/0 = 0
6:3/0 = 0
7:3/0 = 0
8:3/0 = 0
9:3/0 = 0
10:3/0 = 0
11:3/0 = 0
12:3/0 = 0
13:3/0 = 0
14:3/0 = 0
15:3/0 = 0
16:3/0 = 0
17:3/0 = 0
18:3/0 = 0
19:3/0 = 0
20:3/0 = 0
21:3/0 = 0
22:3/0 = 0
23:3/0 = 0
24:3/0 = 0
25:3/0 = 0
26:3/0 = 0
27:3/0 = 0
28:3/0 = 0
29:3/0 = 0
30:3/0 = 0
31:3/0 = 0
32:3/0 = 0
33:3/0 = 0
34:3/0 = 0
35:3/0 = 0
36:3/0 = 0
37:3/0 = 0
38:3/0 = 0
39:3/0 = 0
40:3/0 = 0
41:3/0 = 0
42:3/0 = 0
43:3/0 = 0
44:3/0 = 0
45:3/0 = 0
46:3/0 = 0
47:3/0 = 0
48:3/0 = 0
0:4/0 = 0
1:4/0 = 0
2:4/0 = 0
3:4/0 = 0
4:4/0 = 0
5:4/0 = 0
6:4/0 = 0
7:4/0 = 0
8:4/0 = 0
9:4/0 = 0
10:4/0 = 0
11:4/0 = 0
12:4/0 = 0
13:4/0 = 0
14:4/0 = 0
15:4/0 = 0
16:4/0 = 0
17:4/0 = 0
18:4/0 = 0
19:4/0 = 0
20:4/0 = 0
21:4/0 = 0
22:4/0 = 0
23:4/0 = 0
24:4/0 = 0
25:4/0 = 0
26:4/0 = 0
27:4/0 = 0
28:4/0 = 0
29:4/0 = 0
30:4/0 = 0
31:4/0 = 0
32:4/0 = 0
33:4/0 = 0
34:4/0 = 0
35:4/0 = 0
36:4/0 = 0
37:4/0 = 0
38:4/0 = 0
39:4/0 = 0
40:4/0 = 0
41:4/0 = 0
42:4/0 = 0
43:4/0 = 0
44:4/0 = 0
45:4/0 = 0
46:4/0 = 0
47:4/0 = 0
48:4/0 = 0
0:5/0 = 0
1:5/0 = 0
2:5/0 = 0
3:5/0 = 0
4:5/0 = 0
5:5/0 = 0
6:5/0 = 0
7:5/0 = 0
8:5/0 = 0
9:5/0 = 0
10:5/0 = 0
11:5/0 = 0
12:5/0 = 0
13:5/0 = 0
14:5/0 = 0
15:5/0 = 0
16:5/0 = 0
17:5/0 = 0
18:5/0 = 0
19:5/0 = 0
20:5/0 = 0
21:5/0 = 0
22:5/0 = 0
23:5/0 = 0
24:5/0 = 0
25:5/0 = 0
26:5/0 = 0
27:5/0 = 0
28:5/0 = 0
29:5/0 = 0
30:5/0 = 0
31:5/0 = 0
32:5/0 = 0
33:5/0 = 0
34:5/0 = 0
35:5/0 = 0
36:5/0 = 0
37:5/0 = 0
38:5/0 = 0
39:5/0 = 0
40:5/0 = 0
41:5/0 = 0
42:5/0 = 0
43:5/0 = 0
44:5/0 = 0
45:5/0 = 0
46:5/0 = 0
47:5/0 = 0
48:5/0 = 0
0:6/0 = 0
1:6/0 = 0
2:6/0 = 0
3:6/0 = 0
4:6/0 = 0
5:6/0 = 0
6:6/0 = 0
7:6/0 = 0
8:6/0 = 0
9:6/0 = 0
10:6/0 = 0
11:6/0 = 0
12:6/0 = 0
13:6/0 = 0
14:6/0 = 0
15:6/0 = 0
16:6/0 = 0
17:6/0 = 0
18:6/0 = 0
19:6/0 = 0
20:6/0 = 0
21:6/0 = 0
22:6/0 = 0
23:6/0 = 0
24:6/0 = 0
25:6/0 = 0
26:6/0 = 0
27:6/0 = 0
28:6/0 = 0
29:6/0 = 0
30:6/0 = 0
31:6/0 = 0
32:6/0 = 0
33:6/0 = 0
34:6/0 = 0
35:6/0 = 0
36:6/0 = 0
37:6/0 = 0
38:6/0 = 0
39:6/0 = 0
40:6/0 = 0
41:6/0 = 0
42:6/0 = 0
43:6/0 = 0
44:6/0 = 0
45:6/0 = 0
46:6/0 = 0
47:6/0 = 0
48:6/0 = 0
0:7/0 = 0
1:7/0 = 0
2:7/0 = 0
3:7/0 = 0
4:7/0 = 0
5:7/0 = 0
6:7/0 = 0
7:7/0 = 0
8:7/0 = 0
9:7/0 = 0
10:7/0 = 0
11:7/0 = 0
12:7/0 = 0
13:7/0 = 0
14:7/0 = 0
15:7/0 = 0
16:7/0 = 0
17:7/0 = 0
18:7/0 = 0
19:7/0 = 0
20:7/0 = 0
21:7/0 = 0
22:7/0 = 0
23:7/0 = 0
24:7/0 = 0
25:7/0 = 0
26:7/0 = 0
27:7/0 = 0
28:7/0 = 0
29:7/0 = 0
30:7/0 = 0
31:7/0 = 0
32:7/0 = 0
33:7/0 = 0
34:7/0 = 0
35:7/0 = 0
36:7/0 = 0
37:7/0 = 0
38:7/0 = 0
39:7/0 = 0
40:7/0 = 0
41:7/0 = 0
42:7/0 = 0
43:7/0 = 0
44:7/0 = 0
45:7/0 = 0
46:7/0 = 0
47:7/0 = 0
48:7/0 = 0
0:8/0 = 0
1:8/0 = 0
2:8/0 = 0
3:8/0 = 0
4:8/0 = 0
5:8/0 = 0
6:8/0 = 0
7:8/0 = 0
8:8/0 = 0
9:8/0 = 0
10:8/0 = 0
11:8/0 = 0
12:8/0 = 0
13:8/0 = 0
14:8/0 = 0
15:8/0 = 0
16:8/0 = 0
17:8/0 = 0
18:8/0 = 0
19:8/0 = 0
20:8/0 = 0
21:8/0 = 0
22:8/0 = 0
23:8/0 = 0
24:8/0 = 0
25:8/0 = 0
26:8/0 = 0
27:8/0 = 0
28:8/0 = 0
29:8/0 = 0
30:8/0 = 0
31:8/0 = 0
32:8/0 = 0
33:8/0 = 0
34:8/0 = 0
35:8/0 = 0
36:8/0 = 0
37:8/0 = 0
38:8/0 = 0
39:8/0 = 0
40:8/0 = 0
41:8/0 = 0
42:8/0 = 0
43:8/0 = 0
44:8/0 = 0
45:8/0 = 0
46:8/0 = 0
47:8/0 = 0
48:8/0 = 0
0:9/0 = 0
1:9/0 = 0
2:9/0 = 0
3:9/0 = 0
4:9/0 = 0
5:9/0 = 0
6:9/0 = 0
7:9/0 = 0
8:9/0 = 0
9:9/0 = 0
10:9/0 = 0
11:9/0 = 0
12:9/0 = 0
13:9/0 = 0
14:9/0 = 0
15:9/0 = 0
16:9/0 = 0
17:9/0 = 0
18:9/0 = 0
19:9/0 = 0
20:9/0 = 0
21:9/0 = 0
22:9/0 = 0
23:9/0 = 0
24:9/0 = 0
25:9/0 = 0
26:9/0 = 0
27:9/0 = 0
28:9/0 = 0
29:9/0 = 0
30:9/0 = 0
31:9/0 = 0
32:9/0 = 0
33:9/0 = 0
34:9/0 = 0
35:9/0 = 0
36:9/0 = 0
37:9/0 = 0
38:9/0 = 0
39:9/0 = 0
40:9/0 = 0
41:9/0 = 0
42:9/0 = 0
43:9/0 = 0
44:9/0 = 0
45:9/0 = 0
46:9/0 = 0
47:9/0 = 0
48:9/0 = 0
0:10/0 = 0
1:10/0 = 0
2:10/0 = 0
3:10/0 = 0
4:10/0 = 0
5:10/0 = 0
6:10/0 = 0
7:10/0 = 0
8:10/0 = 0
9:10/0 = 0
10:10/0 = 0
11:10/0 = 0
12:10/0 = 0
13:10/0 = 0
14:10/0 = 0
15:10/0 = 0
16:10/0 = 0
17:10/0 = 0
18:10/0 = 0
19:10/0 = 0
20:10/0 = 0
21:10/0 = 0
22:10/0 = 0
23:10/0 = 0
24:10/0 = 0
25:10/0 = 0
26:10/0 = 0
27:10/0 = 0
28:10/0 = 0
29:10/0 = 0
30:10/0 = 0
31:10/0 = 0
32:10/0 = 0
33:10/0 = 0
34:10/0 = 0
35:10/0 = 0
36:10/0 = 0
37:10/0 = 0
38:10/0 = 0
39:10/0 = 0
40:10/0 = 0
41:10/0 = 0
42:10/0 = 0
43:10/0 = 0
44:10/0 = 0
45:10/0 = 0
46:10/0 = 0
47:10/0 = 0
48:10/0 = 0
0:11/0 = 0
1:11/0 = 0
2:11/0 = 0
3:11/0 = 0
4:11/0 = 0
5:11/0 = 0
6:11/0 = 0
7:11/0 = 0
8:11/0 = 0
9:11/0 = 0
10:11/0 = 0
11:11/0 = 0
12:11/0 = 0
13:11/0 = 0
14:11/0 = 0
15:11/0 = 0
16:11/0 = 0
17:11/0 = 0
18:11/0 = 0
19:11/0 = 0
20:11/0 = 0
21:11/0 = 0
22:11/0 = 0
23:11/0 = 0
24:11/0 = 0
25:11/0 = 0
26:11/0 = 0
27:11/0 = 0
28:11/0 = 0
29:11/0 = 0
30:11/0 = 0
31:11/0 = 0
32:11/0 = 0
33:11/0 = 0
34:11/0 = 0
35:11/0 = 0
36:11/0 = 0
37:11/0 = 0
38:11/0 = 0
39:11/0 = 0
40:11/0 = 0
41:11/0 = 0
42:11/0 = 0
43:11/0 = 0
44:11/0 = 0
45:11/0 = 0
46:11/0 = 0
47:11/0 = 0
48:11/0 = 0
0:12/0 = 0
1:12/0 = 0
2:12/0 = 0
3:12/0 = 0
4:12/0 = 0
5:12/0 = 0
6:12/0 = 0
7:12/0 = 0
8:12/0 = 0
9:12/0 = 0
10:12/0 = 0
11:12/0 = 0
12:12/0 = 0
13:12/0 = 0
14:12/0 = 0
15:12/0 = 0
16:12/0 = 0
17:12/0 = 0
18:12/0 = 0
19:12/0 = 0
20:12/0 = 0
21:12/0 = 0
22:12/0 = 0
23:12/0 = 0
24:12/0 = 0
25:12/0 = 0
26:12/0 = 0
27:12/0 = 0
28:12/0 = 0
29:12/0 = 0
30:12/0 = 0
31:12/0 = 0
32:12/0 = 0
33:12/0 = 0
34:12/0 = 0
35:12/0 = 0
36:12/0 = 0
37:12/0 = 0
38:12/0 = 0
39:12/0 = 0
40:12/0 = 0
41:12/0 = 0
42:12/0 = 0
43:12/0 = 0
44:12/0 = 0
45:12/0 = 0
46:12/0 = 0
47:12/0 = 0
48:12/0 = 0
0:13/0 = 0
1:13/0 = 0
2:13/0 = 0
3:13/0 = 0
4:13/0 = 0
5:13/0 = 0
6:13/0 = 0
7:13/0 = 0
8:13/0 = 0
9:13/0 = 0
10:13/0 = 0
11:13/0 = 0
12:13/0 = 0
13:13/0 = 0
14:13/0 = 0
15:13/0 = 0
16:13/0 = 0
17:13/0 = 0
18:13/0 = 0
19:13/0 = 0
20:13/0 = 0
21:13/0 = 0
22:13/0 = 0
23:13/0 = 0
24:13/0 = 0
25:13/0 = 0
26:13/0 = 0
27:13/0 = 0
28:13/0 = 0
29:13/0 = 0
30:13/0 = 0
31:13/0 = 0
32:13/0 = 0
33:13/0 = 0
34:13/0 = 0
35:13/0 = 0
36:13/0 = 0
37:13/0 = 0
38:13/0 = 0
39:13/0 = 0
40:13/0 = 0
41:13/0 = 0
42:13/0 = 0
43:13/0 = 0
44:13/0 = 0
45:13/0 = 0
46:13/0 = 0
47:13/0 = 0
48:13/0 = 0
0:14/0 = 0
1:14/0 = 0
2:14/0 = 0
3:14/0 = 0
4:14/0 = 0
5:14/0 = 0
6:14/0 = 0
7:14/0 = 0
8:14/0 = 0
9:14/0 = 0
10:14/0 = 0
11:14/0 = 0
12:14/0 = 0
13:14/0 = 0
14:14/0 = 0
15:14/0 = 0
16:14/0 = 0
17:14/0 = 0
18:14/0 = 0
19:14/0 = 0
20:14/0 = 0
21:14/0 = 0
22:14/0 = 0
23:14/0 = 0
24:14/0 = 0
25:14/0 = 0
26:14/0 = 0
27:14/0 = 0
28:14/0 = 0
29:14/0 = 0
30:14/0 = 0
31:14/0 = 0
32:14/0 = 0
33:14/0 = 0
34:14/0 = 0
35:14/0 = 0
36:14/0 = 0
37:14/0 = 0
38:14/0 = 0
39:14/0 = 0
40:14/0 = 0
41:14/0 = 0
42:14/0 = 0
43:14/0 = 0
44:14/0 = 0
45:14/0 = 0
46:14/0 = 0
47:14/0 = 0
48:14/0 = 0
0:15/0 = 0
1:15/0 = 0
2:15/0 = 0
3:15/0 = 0
4:15/0 = 0
5:15/0 = 0
6:15/0 = 0
7:15/0 = 0
8:15/0 = 0
9:15/0 = 0
10:15/0 = 0
11:15/0 = 0
12:15/0 = 0
13:15/0 = 0
14:15/0 = 0
15:15/0 = 0
16:15/0 = 0
17:15/0 = 0
18:15/0 = 0
19:15/0 = 0
20:15/0 = 0
21:15/0 = 0
22:15/0 = 0
23:15/0 = 0
24:15/0 = 0
25:15/0 = 0
26:15/0 = 0
27:15/0 = 0
28:15/0 = 0
29:15/0 = 0
30:15/0 = 0
31:15/0 = 0
32:15/0 = 0
33:15/0 = 0
34:15/0 = 0
35:15/0 = 0
36:15/0 = 0
37:15/0 = 0
38:15/0 = 0
39:15/0 = 0
40:15/0 = 0
41:15/0 = 0
42:15/0 = 0
43:15/0 = 0
44:15/0 = 0
45:15/0 = 0
46:15/0 = 0
47:15/0 = 0
48:15/0 = 0
0:16/0 = 0
1:16/0 = 0
2:16/0 = 0
3:16/0 = 0
4:16/0 = 0
5:16/0 = 0
6:16/0 = 0
7:16/0 = 0
8:16/0 = 0
9:16/0 = 0
10:16/0 = 0
11:16/0 = 0
12:16/0 = 0
13:16/0 = 0
14:16/0 = 0
15:16/0 = 0
16:16/0 = 0
17:16/0 = 0
18:16/0 = 0
19:16/0 = 0
20:16/0 = 0
21:16/0 = 0
22:16/0 = 0
23:16/0 = 0
24:16/0 = 0
25:16/0 = 0
26:16/0 = 0
27:16/0 = 0
28:16/0 = 0
29:16/0 = 0
30:16/0 = 0
31:16/0 = 0
32:16/0 = 0
33:16/0 = 0
34:16/0 = 0
35:16/0 = 0
36:16/0 = 0
37:16/0 = 0
38:16/0 = 0
39:16/0 = 0
40:16/0 = 0
41:16/0 = 0
42:16/0 = 0
43:16/0 = 0
44:16/0 = 0
45:16/0 = 0
46:16/0 = 0
47:16/0 = 0
48:16/0 = 0
0:17/0 = 0
1:17/0 = 0
2:17/0 = 0
3:17/0 = 0
4:17/0 = 0
5:17/0 = 0
6:17/0 = 0
7:17/0 = 0
8:17/0 = 0
9:17/0 = 0
10:17/0 = 0
11:17/0 = 0
12:17/0 = 0
13:17/0 = 0
14:17/0 = 0
15:17/0 = 0
16:17/0 = 0
17:17/0 = 0
18:17/0 = 0
19:17/0 = 0
20:17/0 = 0
21:17/0 = 0
22:17/0 = 0
23:17/0 = 0
24:17/0 = 0
25:17/0 = 0
26:17/0 = 0
27:17/0 = 0
28:17/0 = 0
29:17/0 = 0
30:17/0 = 0
31:17/0 = 0
32:17/0 = 0
33:17/0 = 0
34:17/0 = 0
35:17/0 = 0
36:17/0 = 0
37:17/0 = 0
38:17/0 = 0
39:17/0 = 0
40:17/0 = 0
41:17/0 = 0
42:17/0 = 0
43:17/0 = 0
44:17/0 = 0
45:17/0 = 0
46:17/0 = 0
47:17/0 = 0
48:17/0 = 0
0:18/0 = 0
1:18/0 = 0
2:18/0 = 0
3:18/0 = 0
4:18/0 = 0
5:18/0 = 0
6:18/0 = 0
7:18/0 = 0
8:18/0 = 0
9:18/0 = 0
10:18/0 = 0
11:18/0 = 0
12:18/0 = 0
13:18/0 = 0
14:18/0 = 0
15:18/0 = 0
16:18/0 = 0
17:18/0 = 0
18:18/0 = 0
19:18/0 = 0
20:18/0 = 0
21:18/0 = 0
22:18/0 = 0
23:18/0 = 0
24:18/0 = 0
25:18/0 = 0
26:18/0 = 0
27:18/0 = 0
28:18/0 = 0
29:18/0 = 0
30:18/0 = 0
31:18/0 = 0
32:18/0 = 0
33:18/0 = 0
34:18/0 = 0
35:18/0 = 0
36:18/0 = 0
37:18/0 = 0
38:18/0 = 0
39:18/0 = 0
40:18/0 = 0
41:18/0 = 0
42:18/0 = 0
43:18/0 = 0
44:18/0 = 0
45:18/0 = 0
46:18/0 = 0
47:18/0 = 0
48:18/0 = 0
0:19/0 = 0
1:19/0 = 0
2:19/0 = 0
3:19/0 = 0
4:19/0 = 0
5:19/0 = 0
6:19/0 = 0
7:19/0 = 0
8:19/0 = 0
9:19/0 = 0
10:19/0 = 0
11:19/0 = 0
12:19/0 = 0
13:19/0 = 0
14:19/0 = 0
15:19/0 = 0
16:19/0 = 0
17:19/0 = 0
18:19/0 = 0
19:19/0 = 0
20:19/0 = 0
21:19/0 = 0
22:19/0 = 0
23:19/0 = 0
24:19/0 = 0
25:19/0 = 0
26:19/0 = 0
27:19/0 = 0
28:19/0 = 0
29:19/0 = 0
30:19/0 = 0
31:19/0 = 0
32:19/0 = 0
33:19/0 = 0
34:19/0 = 0
35:19/0 = 0
36:19/0 = 0
37:19/0 = 0
38:19/0 = 0
39:19/0 = 0
40:19/0 = 0
41:19/0 = 0
42:19/0 = 0
43:19/0 = 0
44:19/0 = 0
45:19/0 = 0
46:19/0 = 0
47:19/0 = 0
48:19/0 = 0
0:20/0 = 0
1:20/0 = 0
2:20/0 = 0
3:20/0 = 0
4:20/0 = 0
5:20/0 = 0
6:20/0 = 0
7:20/0 = 0
8:20/0 = 0
9:20/0 = 0
10:20/0 = 0
11:20/0 = 0
12:20/0 = 0
13:20/0 = 0
14:20/0 = 0
15:20/0 = 0
16:20/0 = 0
17:20/0 = 0
18:20/0 = 0
19:20/0 = 0
20:20/0 = 0
21:20/0 = 0
22:20/0 = 0
23:20/0 = 0
24:20/0 = 0
25:20/0 = 0
26:20/0 = 0
27:20/0 = 0
28:20/0 = 0
29:20/0 = 0
30:20/0 = 0
31:20/0 = 0
32:20/0 = 0
33:20/0 = 0
34:20/0 = 0
35:20/0 = 0
36:20/0 = 0
37:20/0 = 0
38:20/0 = 0
39:20/0 = 0
40:20/0 = 0
41:20/0 = 0
42:20/0 = 0
43:20/0 = 0
44:20/0 = 0
45:20/0 = 0
46:20/0 = 0
47:20/0 = 0
48:20/0 = 0
0:21/0 = 0
1:21/0 = 0
2:21/0 = 0
3:21/0 = 0
4:21/0 = 0
5:21/0 = 0
6:21/0 = 0
7:21/0 = 0
8:21/0 = 0
9:21/0 = 0
10:21/0 = 0
11:21/0 = 0
12:21/0 = 0
13:21/0 = 0
14:21/0 = 0
15:21/0 = 0
16:21/0 = 0
17:21/0 = 0
18:21/0 = 0
19:21/0 = 0
20:21/0 = 0
21:21/0 = 0
22:21/0 = 0
23:21/0 = 0
24:21/0 = 0
25:21/0 = 0
26:21/0 = 0
27:21/0 = 0
28:21/0 = 0
29:21/0 = 0
30:21/0 = 0
31:21/0 = 0
32:21/0 = 0
33:21/0 = 0
34:21/0 = 0
35:21/0 = 0
36:21/0 = 0
37:21/0 = 0
38:21/0 = 0
39:21/0 = 0
40:21/0 = 0
41:21/0 = 0
42:21/0 = 0
43:21/0 = 0
44:21/0 = 0
45:21/0 = 0
46:21/0 = 0
47:21/0 = 0
48:21/0 = 0

[sub_resource type="TileSet" id="TileSet_3drs5"]
sources/0 = SubResource("TileSetAtlasSource_6h0bd")

[node name="TileMap" type="TileMap"]
tile_set = SubResource("TileSet_3drs5")
format = 2
script = ExtResource("2_lf4lw")

[node name="AcceptDialog" parent="." instance=ExtResource("3_y08lj")]

[node name="WinDialog" parent="." instance=ExtResource("4_fkys0")]
\end{lstlisting}

\subsection{tile\textunderscore{}map.gd}

\begin{lstlisting}
extends TileMap

const buildings: Array[Vector2i] = [
	Vector2i(0, 19),
	Vector2i(1, 19),
	Vector2i(2, 19),
	Vector2i(3, 19),
	Vector2i(4, 19),
	Vector2i(5, 19),
	Vector2i(6, 19),
	Vector2i(7, 19),
	Vector2i(8, 20),
	Vector2i(0, 20),
	Vector2i(1, 20),
	Vector2i(2, 20),
	Vector2i(3, 20),
	Vector2i(4, 20),
	Vector2i(5, 20),
	Vector2i(6, 20),
	Vector2i(7, 20),
	Vector2i(8, 20),
	Vector2i(0, 21),
	Vector2i(1, 21),
	Vector2i(2, 21),
	Vector2i(3, 21),
	Vector2i(4, 21),
	Vector2i(5, 21),
	Vector2i(6, 21),
	Vector2i(7, 21),
	Vector2i(8, 21)
]
const trees: Array[Vector2i] = [
	Vector2i(0,1),
	Vector2i(1,1),
	Vector2i(2,1),
	Vector2i(3,1),
	Vector2i(4,1),
	Vector2i(5,1),
	Vector2i(6,1),
	Vector2i(7,1),
	Vector2i(0,2),
	Vector2i(1,2),
	Vector2i(2,2),
	Vector2i(3,2),
	Vector2i(4,2)
]
const PLAYER_SPRITE: Vector2i = Vector2i(24, 7)
var player_placement_cell: Vector2i
const rings: Array[Vector2i] = [
	Vector2i(43, 6),
	Vector2i(44, 6),
	Vector2i(45, 6),
	Vector2i(46, 6)
]
var ring_placement_cell: Vector2i

var points: Array[Dictionary] = []
const EUCLIDEAN: String = "Euclidean distance"
const MANHATTAN: String = "Manhattan distance"
@export_enum(EUCLIDEAN, MANHATTAN) var distance: String = MANHATTAN
@export_range(10, 40, 1) var random_starting_points: int = 20
var x_tile_range: int = ProjectSettings.get_setting("display/window/size/viewport_width") / tile_set.tile_size.x
var y_tile_range: int = ProjectSettings.get_setting("display/window/size/viewport_height") / tile_set.tile_size.y

# Called when the node enters the scene tree for the first time.
func _ready() -> void:
	randomize()
	var start_time: float = Time.get_ticks_msec()
	define_points(random_starting_points)
	paint_points()
	place_player()
	place_ring()
	var new_time: float = Time.get_ticks_msec() - start_time
	print("Time taken: " + str(new_time) + "ms")
	$AcceptDialog.dialog_text = "You're a hollow Golem who seeks the ultimate treasure; a ring that's got something on top of it. It's somewhere in this large village and barely visible to your naked eyes, which took us " + str(new_time) + " milliseconds to generate (" + str(new_time / 1000.0) + " seconds), but you'll stop at nothing to get what you want. You can chow down every tree and fauna that stands in your way of the ring, but your Achilles heel is any bricks and mortar, which WILL make you stop at your tracks. Since it's easy to get lost in here, we'll tell you that you're in position " + str(player_placement_cell) + " in this big village of size " + str(Vector2i(x_tile_range, y_tile_range)) + ". It's also easy to get stuck here, so either press the G key or right click to spawn somewhere else where there is fauna (or even the ring!!), because this game actually WANTS you to win it. Ultimately, though, it is YOUR job to find the ring, so are you ready to attain the treasure that is rightfully yours?!"
	$AcceptDialog.visible = true
	$AcceptDialog.confirmed.connect(_on_AcceptDialog_closed)
	$AcceptDialog.canceled.connect(_on_AcceptDialog_closed)
	$WinDialog.confirmed.connect(_on_WinDialog_confirmed)
	$WinDialog.canceled.connect(_on_WinDialog_canceled)
	get_tree().paused = true

func _on_WinDialog_confirmed() -> void:
	get_tree().reload_current_scene()

func _on_WinDialog_canceled() -> void:
	get_tree().quit()

func _on_AcceptDialog_closed() -> void:
	$AcceptDialog.visible = false
	get_tree().paused = false

func _get_random_placement_cell() -> Vector2i:
	return Vector2i(randi() % x_tile_range, randi() % y_tile_range)

func place_player() -> void:
	player_placement_cell = _get_random_placement_cell()
	while buildings.has(get_cell_atlas_coords(0, player_placement_cell)) or player_placement_cell == ring_placement_cell:
		player_placement_cell = _get_random_placement_cell()
	set_cell(0, player_placement_cell, 0, PLAYER_SPRITE)

func place_ring() -> void:
	ring_placement_cell = _get_random_placement_cell()
	while buildings.has(get_cell_atlas_coords(0, ring_placement_cell)) or ring_placement_cell == player_placement_cell:
		ring_placement_cell = _get_random_placement_cell()
	set_cell(0, ring_placement_cell, 0, rings.pick_random())

func _is_not_out_of_bounds(cell: Vector2i) -> bool:
	return cell.x >= 0 and cell.x < x_tile_range and cell.y >= 0 and cell.y < y_tile_range

func _physics_process(_delta) -> void:
	var previous_cell: Vector2i = player_placement_cell
	var direction: Vector2i = Vector2i.ZERO
	if Input.is_action_pressed("ui_up"): direction = Vector2i.UP
	elif Input.is_action_pressed("ui_down"): direction = Vector2i.DOWN
	elif Input.is_action_pressed("ui_left"): direction = Vector2i.LEFT
	elif Input.is_action_pressed("ui_right"): direction = Vector2i.RIGHT
	elif Input.is_action_just_pressed("reset_position"): # Respawn player in a different part of the map
		player_placement_cell = _get_random_placement_cell()
		while buildings.has(get_cell_atlas_coords(0, player_placement_cell)): # This time, since we're not STARTING the game, we don't care whether or not the player magically lands on the ring
			player_placement_cell = _get_random_placement_cell()
		set_cell(0, player_placement_cell, 0, PLAYER_SPRITE)
		set_cell(0, previous_cell, 0) # replace the previous sprite
		return
	var new_placement_cell: Vector2i = player_placement_cell + direction
	if (not get_used_cells(0).has(new_placement_cell) or trees.has(get_cell_atlas_coords(0, new_placement_cell)) or new_placement_cell == ring_placement_cell) and _is_not_out_of_bounds(new_placement_cell):
		player_placement_cell = new_placement_cell
		set_cell(0, previous_cell, 0) # deletes contents of previous cell (atlas_coords = Vector2i(-1, -1))
		set_cell(0, player_placement_cell, 0, PLAYER_SPRITE)
		if player_placement_cell == ring_placement_cell:
			$WinDialog.visible = true
			get_tree().paused = true

# ALGORITHM BEGINS HERE

func paint_points() -> void:
	for point in points:
		set_cell(0, Vector2(point["x"], point["y"]), 0, point["type"])
		for citizen in point["citizens"]:
			if _is_in_bounds(point["x"], citizen["dx"], point["y"], citizen["dy"]):
				set_cell(0, Vector2(point["x"] + citizen["dx"], point["y"] + citizen["dy"]), 0, point["type"])

func _is_in_bounds(x: int, dx: int, y: int, dy: int) -> bool:
	return x + dx >= 0 and x + dx < x_tile_range and y + dy >= 0 and  y + dy < y_tile_range

func _squared(x: int) -> int:
	return x ** 2

func calculate_points_delta(x: int, y: int, p: int) -> float:
	if distance == EUCLIDEAN:
		return sqrt(_squared(points[p]["x"] - x) + _squared(points[p]["y"] - y))
	return abs(points[p]["x"] - x) + abs(points[p]["y"] - y)

func define_points(num_points: int) -> void:
	var types: Array[Vector2i] = trees.duplicate()
	types.append_array(buildings)
	for i in range(num_points):
		var x: int = randi_range(0, x_tile_range)
		var y: int = randi_range(0, y_tile_range)
		var type: Vector2i = types.pick_random()
		types.erase(type)
		points.append(
			{
				"type": type,
				"x": x,
				"y": y,
				"citizens": []
			}
		)
	for x in range(x_tile_range):
		for y in range(y_tile_range):
			var lowest_delta: Dictionary = {
				"point_id": 0,
				"delta": x_tile_range * y_tile_range
			}
			for p in range(len(points)):
				var delta: float = calculate_points_delta(x, y, p)
				if delta < lowest_delta["delta"]:
					lowest_delta = {
						"point_id": p,
						"delta": delta
					}
				var active_point: Dictionary = points[lowest_delta["point_id"]]
				var dx: int = x - active_point["x"]
				var dy: int = y - active_point["y"]
				active_point["citizens"].append(
					{
						"dx": dx,
						"dy": dy
					}
				)
\end{lstlisting}

\subsection{accept\_dialog.tscn}

\begin{lstlisting}
[gd_scene format=3 uid="uid://cau5jgogdnf53"]

[node name="AcceptDialog" type="AcceptDialog"]
title = "Tree-Munching Time!"
position = Vector2i(326, 100)
size = Vector2i(500, 421)
mouse_passthrough = true
ok_button_text = "Bring it on!"
dialog_text = "You're a hollow Golem who seeks the ultimate treasure; a ring that's got something on top of it. It's somewhere in this large village and barely visible to your naked eyes, but you'll stop at nothing to get what you want. You can chow down every tree and fauna that stands in your way of the ring, but your Achilles heel is any bricks and mortar, which will make you stop at your tracks. Are you ready to attain your treasure?w Golem in a black-and-white world, in search for your most desired treasure. It's a ring with something on top of it. And you'll stop at nothing to get what you want. You can chow down every tree and fauna that stands in your way of the ring, but your Achilles heel is any bricks and mortar, which will make you stop at your tracks. Are you ready to attain the treasure that is rightfully yours?!"
dialog_autowrap = true
\end{lstlisting}

\subsection{win\_dialog.tscn}

\begin{lstlisting}
[gd_scene format=3 uid="uid://b5q8ovcigrvyr"]

[node name="WinDialog" type="ConfirmationDialog"]
title = "You Found the Treasure!"
position = Vector2i(326, 100)
size = Vector2i(500, 421)
mouse_passthrough = true
ok_button_text = "Get Me a New Village"
dialog_text = "You found your treasure! Well done, you!

Would you like to travel to a new village in the hopes of finding another ring? Or would you like to take your treasure home now?"
dialog_autowrap = true
cancel_button_text = "Get Me Out of Here"
\end{lstlisting}

\subsection{icon.svg.import}

\begin{lstlisting}
[remap]

importer="texture"
type="CompressedTexture2D"
uid="uid://du4v6taw8ssax"
path="res://.godot/imported/icon.svg-218a8f2b3041327d8a5756f3a245f83b.ctex"
metadata={
"vram_texture": false
}

[deps]

source_file="res://icon.svg"
dest_files=["res://.godot/imported/icon.svg-218a8f2b3041327d8a5756f3a245f83b.ctex"]

[params]

compress/mode=0
compress/high_quality=false
compress/lossy_quality=0.7
compress/hdr_compression=1
compress/normal_map=0
compress/channel_pack=0
mipmaps/generate=false
mipmaps/limit=-1
roughness/mode=0
roughness/src_normal=""
process/fix_alpha_border=true
process/premult_alpha=false
process/normal_map_invert_y=false
process/hdr_as_srgb=false
process/hdr_clamp_exposure=false
process/size_limit=0
detect_3d/compress_to=1
svg/scale=1.0
editor/scale_with_editor_scale=false
editor/convert_colors_with_editor_theme=false
\end{lstlisting}

\subsection{monochrome\textunderscore{}packed.png.import}

\begin{lstlisting}
[remap]

importer="texture"
type="CompressedTexture2D"
uid="uid://cpign73sfbsrt"
path="res://.godot/imported/monochrome_packed.png-6b9bd1c64dd50f72acd3afd14d1ac34f.ctex"
metadata={
"vram_texture": false
}

[deps]

source_file="res://monochrome_packed.png"
dest_files=["res://.godot/imported/monochrome_packed.png-6b9bd1c64dd50f72acd3afd14d1ac34f.ctex"]

[params]

compress/mode=0
compress/high_quality=false
compress/lossy_quality=0.7
compress/hdr_compression=1
compress/normal_map=0
compress/channel_pack=0
mipmaps/generate=false
mipmaps/limit=-1
roughness/mode=0
roughness/src_normal=""
process/fix_alpha_border=true
process/premult_alpha=false
process/normal_map_invert_y=false
process/hdr_as_srgb=false
process/hdr_clamp_exposure=false
process/size_limit=0
detect_3d/compress_to=1
\end{lstlisting}

\section{GD4PoissonRPG}

% \localtableofcontents{}

\subsection{.gitattributes}

\begin{lstlisting}
# Normalize EOL for all files that Git considers text files.
* text=auto eol=lf
\end{lstlisting}

\subsection{.gitignore}

\begin{lstlisting}
# Godot 4+ specific ignores
.godot/
\end{lstlisting}

\subsection{project.godot}

\begin{lstlisting}
; Engine configuration file.
; It's best edited using the editor UI and not directly,
; since the parameters that go here are not all obvious.
;
; Format:
;   [section] ; section goes between []
;   param=value ; assign values to parameters

config_version=5

[application]

config/name="Voronoi Cells"
run/main_scene="res://tile_map.tscn"
config/features=PackedStringArray("4.0", "Forward Plus")
config/icon="res://icon.svg"

[display]

window/size/viewport_height=640

[input]

reset_position={
"deadzone": 0.5,
"events": [Object(InputEventKey,"resource_local_to_scene":false,"resource_name":"","device":-1,"window_id":0,"alt_pressed":false,"shift_pressed":false,"ctrl_pressed":false,"meta_pressed":false,"pressed":false,"keycode":71,"physical_keycode":0,"key_label":0,"unicode":103,"echo":false,"script":null)
, Object(InputEventMouseButton,"resource_local_to_scene":false,"resource_name":"","device":-1,"window_id":0,"alt_pressed":false,"shift_pressed":false,"ctrl_pressed":false,"meta_pressed":false,"button_mask":2,"position":Vector2(75, 12),"global_position":Vector2(78, 44),"factor":1.0,"button_index":2,"pressed":true,"double_click":false,"script":null)
]
}

[rendering]

environment/defaults/default_clear_color=Color(0, 0, 0, 1)
\end{lstlisting}

\subsection{tile\textunderscore{}map.tscn}

\begin{lstlisting}
[gd_scene load_steps=7 format=3 uid="uid://d6lxnr5bdh1w"]

[ext_resource type="Texture2D" uid="uid://cpign73sfbsrt" path="res://monochrome_packed.png" id="1_o183d"]
[ext_resource type="Script" path="res://tile_map.gd" id="2_lf4lw"]
[ext_resource type="PackedScene" path="res://accept_dialog.tscn" id="3_y08lj"]
[ext_resource type="PackedScene" path="res://win_dialog.tscn" id="4_fkys0"]

[sub_resource type="TileSetAtlasSource" id="TileSetAtlasSource_6h0bd"]
texture = ExtResource("1_o183d")
0:0/0 = 0
1:0/0 = 0
2:0/0 = 0
3:0/0 = 0
4:0/0 = 0
5:0/0 = 0
6:0/0 = 0
7:0/0 = 0
8:0/0 = 0
9:0/0 = 0
10:0/0 = 0
11:0/0 = 0
12:0/0 = 0
13:0/0 = 0
14:0/0 = 0
15:0/0 = 0
16:0/0 = 0
17:0/0 = 0
18:0/0 = 0
19:0/0 = 0
20:0/0 = 0
21:0/0 = 0
22:0/0 = 0
23:0/0 = 0
24:0/0 = 0
25:0/0 = 0
26:0/0 = 0
27:0/0 = 0
28:0/0 = 0
29:0/0 = 0
30:0/0 = 0
31:0/0 = 0
32:0/0 = 0
33:0/0 = 0
34:0/0 = 0
35:0/0 = 0
36:0/0 = 0
37:0/0 = 0
38:0/0 = 0
39:0/0 = 0
40:0/0 = 0
41:0/0 = 0
42:0/0 = 0
43:0/0 = 0
44:0/0 = 0
45:0/0 = 0
46:0/0 = 0
47:0/0 = 0
48:0/0 = 0
0:1/0 = 0
1:1/0 = 0
2:1/0 = 0
3:1/0 = 0
4:1/0 = 0
5:1/0 = 0
6:1/0 = 0
7:1/0 = 0
8:1/0 = 0
9:1/0 = 0
10:1/0 = 0
11:1/0 = 0
12:1/0 = 0
13:1/0 = 0
14:1/0 = 0
15:1/0 = 0
16:1/0 = 0
17:1/0 = 0
18:1/0 = 0
19:1/0 = 0
20:1/0 = 0
21:1/0 = 0
22:1/0 = 0
23:1/0 = 0
24:1/0 = 0
25:1/0 = 0
26:1/0 = 0
27:1/0 = 0
28:1/0 = 0
29:1/0 = 0
30:1/0 = 0
31:1/0 = 0
32:1/0 = 0
33:1/0 = 0
34:1/0 = 0
35:1/0 = 0
36:1/0 = 0
37:1/0 = 0
38:1/0 = 0
39:1/0 = 0
40:1/0 = 0
41:1/0 = 0
42:1/0 = 0
43:1/0 = 0
44:1/0 = 0
45:1/0 = 0
46:1/0 = 0
47:1/0 = 0
48:1/0 = 0
0:2/0 = 0
1:2/0 = 0
2:2/0 = 0
3:2/0 = 0
4:2/0 = 0
5:2/0 = 0
6:2/0 = 0
7:2/0 = 0
8:2/0 = 0
9:2/0 = 0
10:2/0 = 0
11:2/0 = 0
12:2/0 = 0
13:2/0 = 0
14:2/0 = 0
15:2/0 = 0
16:2/0 = 0
17:2/0 = 0
18:2/0 = 0
19:2/0 = 0
20:2/0 = 0
21:2/0 = 0
22:2/0 = 0
23:2/0 = 0
24:2/0 = 0
25:2/0 = 0
26:2/0 = 0
27:2/0 = 0
28:2/0 = 0
29:2/0 = 0
30:2/0 = 0
31:2/0 = 0
32:2/0 = 0
33:2/0 = 0
34:2/0 = 0
35:2/0 = 0
36:2/0 = 0
37:2/0 = 0
38:2/0 = 0
39:2/0 = 0
40:2/0 = 0
41:2/0 = 0
42:2/0 = 0
43:2/0 = 0
44:2/0 = 0
45:2/0 = 0
46:2/0 = 0
47:2/0 = 0
48:2/0 = 0
0:3/0 = 0
1:3/0 = 0
2:3/0 = 0
3:3/0 = 0
4:3/0 = 0
5:3/0 = 0
6:3/0 = 0
7:3/0 = 0
8:3/0 = 0
9:3/0 = 0
10:3/0 = 0
11:3/0 = 0
12:3/0 = 0
13:3/0 = 0
14:3/0 = 0
15:3/0 = 0
16:3/0 = 0
17:3/0 = 0
18:3/0 = 0
19:3/0 = 0
20:3/0 = 0
21:3/0 = 0
22:3/0 = 0
23:3/0 = 0
24:3/0 = 0
25:3/0 = 0
26:3/0 = 0
27:3/0 = 0
28:3/0 = 0
29:3/0 = 0
30:3/0 = 0
31:3/0 = 0
32:3/0 = 0
33:3/0 = 0
34:3/0 = 0
35:3/0 = 0
36:3/0 = 0
37:3/0 = 0
38:3/0 = 0
39:3/0 = 0
40:3/0 = 0
41:3/0 = 0
42:3/0 = 0
43:3/0 = 0
44:3/0 = 0
45:3/0 = 0
46:3/0 = 0
47:3/0 = 0
48:3/0 = 0
0:4/0 = 0
1:4/0 = 0
2:4/0 = 0
3:4/0 = 0
4:4/0 = 0
5:4/0 = 0
6:4/0 = 0
7:4/0 = 0
8:4/0 = 0
9:4/0 = 0
10:4/0 = 0
11:4/0 = 0
12:4/0 = 0
13:4/0 = 0
14:4/0 = 0
15:4/0 = 0
16:4/0 = 0
17:4/0 = 0
18:4/0 = 0
19:4/0 = 0
20:4/0 = 0
21:4/0 = 0
22:4/0 = 0
23:4/0 = 0
24:4/0 = 0
25:4/0 = 0
26:4/0 = 0
27:4/0 = 0
28:4/0 = 0
29:4/0 = 0
30:4/0 = 0
31:4/0 = 0
32:4/0 = 0
33:4/0 = 0
34:4/0 = 0
35:4/0 = 0
36:4/0 = 0
37:4/0 = 0
38:4/0 = 0
39:4/0 = 0
40:4/0 = 0
41:4/0 = 0
42:4/0 = 0
43:4/0 = 0
44:4/0 = 0
45:4/0 = 0
46:4/0 = 0
47:4/0 = 0
48:4/0 = 0
0:5/0 = 0
1:5/0 = 0
2:5/0 = 0
3:5/0 = 0
4:5/0 = 0
5:5/0 = 0
6:5/0 = 0
7:5/0 = 0
8:5/0 = 0
9:5/0 = 0
10:5/0 = 0
11:5/0 = 0
12:5/0 = 0
13:5/0 = 0
14:5/0 = 0
15:5/0 = 0
16:5/0 = 0
17:5/0 = 0
18:5/0 = 0
19:5/0 = 0
20:5/0 = 0
21:5/0 = 0
22:5/0 = 0
23:5/0 = 0
24:5/0 = 0
25:5/0 = 0
26:5/0 = 0
27:5/0 = 0
28:5/0 = 0
29:5/0 = 0
30:5/0 = 0
31:5/0 = 0
32:5/0 = 0
33:5/0 = 0
34:5/0 = 0
35:5/0 = 0
36:5/0 = 0
37:5/0 = 0
38:5/0 = 0
39:5/0 = 0
40:5/0 = 0
41:5/0 = 0
42:5/0 = 0
43:5/0 = 0
44:5/0 = 0
45:5/0 = 0
46:5/0 = 0
47:5/0 = 0
48:5/0 = 0
0:6/0 = 0
1:6/0 = 0
2:6/0 = 0
3:6/0 = 0
4:6/0 = 0
5:6/0 = 0
6:6/0 = 0
7:6/0 = 0
8:6/0 = 0
9:6/0 = 0
10:6/0 = 0
11:6/0 = 0
12:6/0 = 0
13:6/0 = 0
14:6/0 = 0
15:6/0 = 0
16:6/0 = 0
17:6/0 = 0
18:6/0 = 0
19:6/0 = 0
20:6/0 = 0
21:6/0 = 0
22:6/0 = 0
23:6/0 = 0
24:6/0 = 0
25:6/0 = 0
26:6/0 = 0
27:6/0 = 0
28:6/0 = 0
29:6/0 = 0
30:6/0 = 0
31:6/0 = 0
32:6/0 = 0
33:6/0 = 0
34:6/0 = 0
35:6/0 = 0
36:6/0 = 0
37:6/0 = 0
38:6/0 = 0
39:6/0 = 0
40:6/0 = 0
41:6/0 = 0
42:6/0 = 0
43:6/0 = 0
44:6/0 = 0
45:6/0 = 0
46:6/0 = 0
47:6/0 = 0
48:6/0 = 0
0:7/0 = 0
1:7/0 = 0
2:7/0 = 0
3:7/0 = 0
4:7/0 = 0
5:7/0 = 0
6:7/0 = 0
7:7/0 = 0
8:7/0 = 0
9:7/0 = 0
10:7/0 = 0
11:7/0 = 0
12:7/0 = 0
13:7/0 = 0
14:7/0 = 0
15:7/0 = 0
16:7/0 = 0
17:7/0 = 0
18:7/0 = 0
19:7/0 = 0
20:7/0 = 0
21:7/0 = 0
22:7/0 = 0
23:7/0 = 0
24:7/0 = 0
25:7/0 = 0
26:7/0 = 0
27:7/0 = 0
28:7/0 = 0
29:7/0 = 0
30:7/0 = 0
31:7/0 = 0
32:7/0 = 0
33:7/0 = 0
34:7/0 = 0
35:7/0 = 0
36:7/0 = 0
37:7/0 = 0
38:7/0 = 0
39:7/0 = 0
40:7/0 = 0
41:7/0 = 0
42:7/0 = 0
43:7/0 = 0
44:7/0 = 0
45:7/0 = 0
46:7/0 = 0
47:7/0 = 0
48:7/0 = 0
0:8/0 = 0
1:8/0 = 0
2:8/0 = 0
3:8/0 = 0
4:8/0 = 0
5:8/0 = 0
6:8/0 = 0
7:8/0 = 0
8:8/0 = 0
9:8/0 = 0
10:8/0 = 0
11:8/0 = 0
12:8/0 = 0
13:8/0 = 0
14:8/0 = 0
15:8/0 = 0
16:8/0 = 0
17:8/0 = 0
18:8/0 = 0
19:8/0 = 0
20:8/0 = 0
21:8/0 = 0
22:8/0 = 0
23:8/0 = 0
24:8/0 = 0
25:8/0 = 0
26:8/0 = 0
27:8/0 = 0
28:8/0 = 0
29:8/0 = 0
30:8/0 = 0
31:8/0 = 0
32:8/0 = 0
33:8/0 = 0
34:8/0 = 0
35:8/0 = 0
36:8/0 = 0
37:8/0 = 0
38:8/0 = 0
39:8/0 = 0
40:8/0 = 0
41:8/0 = 0
42:8/0 = 0
43:8/0 = 0
44:8/0 = 0
45:8/0 = 0
46:8/0 = 0
47:8/0 = 0
48:8/0 = 0
0:9/0 = 0
1:9/0 = 0
2:9/0 = 0
3:9/0 = 0
4:9/0 = 0
5:9/0 = 0
6:9/0 = 0
7:9/0 = 0
8:9/0 = 0
9:9/0 = 0
10:9/0 = 0
11:9/0 = 0
12:9/0 = 0
13:9/0 = 0
14:9/0 = 0
15:9/0 = 0
16:9/0 = 0
17:9/0 = 0
18:9/0 = 0
19:9/0 = 0
20:9/0 = 0
21:9/0 = 0
22:9/0 = 0
23:9/0 = 0
24:9/0 = 0
25:9/0 = 0
26:9/0 = 0
27:9/0 = 0
28:9/0 = 0
29:9/0 = 0
30:9/0 = 0
31:9/0 = 0
32:9/0 = 0
33:9/0 = 0
34:9/0 = 0
35:9/0 = 0
36:9/0 = 0
37:9/0 = 0
38:9/0 = 0
39:9/0 = 0
40:9/0 = 0
41:9/0 = 0
42:9/0 = 0
43:9/0 = 0
44:9/0 = 0
45:9/0 = 0
46:9/0 = 0
47:9/0 = 0
48:9/0 = 0
0:10/0 = 0
1:10/0 = 0
2:10/0 = 0
3:10/0 = 0
4:10/0 = 0
5:10/0 = 0
6:10/0 = 0
7:10/0 = 0
8:10/0 = 0
9:10/0 = 0
10:10/0 = 0
11:10/0 = 0
12:10/0 = 0
13:10/0 = 0
14:10/0 = 0
15:10/0 = 0
16:10/0 = 0
17:10/0 = 0
18:10/0 = 0
19:10/0 = 0
20:10/0 = 0
21:10/0 = 0
22:10/0 = 0
23:10/0 = 0
24:10/0 = 0
25:10/0 = 0
26:10/0 = 0
27:10/0 = 0
28:10/0 = 0
29:10/0 = 0
30:10/0 = 0
31:10/0 = 0
32:10/0 = 0
33:10/0 = 0
34:10/0 = 0
35:10/0 = 0
36:10/0 = 0
37:10/0 = 0
38:10/0 = 0
39:10/0 = 0
40:10/0 = 0
41:10/0 = 0
42:10/0 = 0
43:10/0 = 0
44:10/0 = 0
45:10/0 = 0
46:10/0 = 0
47:10/0 = 0
48:10/0 = 0
0:11/0 = 0
1:11/0 = 0
2:11/0 = 0
3:11/0 = 0
4:11/0 = 0
5:11/0 = 0
6:11/0 = 0
7:11/0 = 0
8:11/0 = 0
9:11/0 = 0
10:11/0 = 0
11:11/0 = 0
12:11/0 = 0
13:11/0 = 0
14:11/0 = 0
15:11/0 = 0
16:11/0 = 0
17:11/0 = 0
18:11/0 = 0
19:11/0 = 0
20:11/0 = 0
21:11/0 = 0
22:11/0 = 0
23:11/0 = 0
24:11/0 = 0
25:11/0 = 0
26:11/0 = 0
27:11/0 = 0
28:11/0 = 0
29:11/0 = 0
30:11/0 = 0
31:11/0 = 0
32:11/0 = 0
33:11/0 = 0
34:11/0 = 0
35:11/0 = 0
36:11/0 = 0
37:11/0 = 0
38:11/0 = 0
39:11/0 = 0
40:11/0 = 0
41:11/0 = 0
42:11/0 = 0
43:11/0 = 0
44:11/0 = 0
45:11/0 = 0
46:11/0 = 0
47:11/0 = 0
48:11/0 = 0
0:12/0 = 0
1:12/0 = 0
2:12/0 = 0
3:12/0 = 0
4:12/0 = 0
5:12/0 = 0
6:12/0 = 0
7:12/0 = 0
8:12/0 = 0
9:12/0 = 0
10:12/0 = 0
11:12/0 = 0
12:12/0 = 0
13:12/0 = 0
14:12/0 = 0
15:12/0 = 0
16:12/0 = 0
17:12/0 = 0
18:12/0 = 0
19:12/0 = 0
20:12/0 = 0
21:12/0 = 0
22:12/0 = 0
23:12/0 = 0
24:12/0 = 0
25:12/0 = 0
26:12/0 = 0
27:12/0 = 0
28:12/0 = 0
29:12/0 = 0
30:12/0 = 0
31:12/0 = 0
32:12/0 = 0
33:12/0 = 0
34:12/0 = 0
35:12/0 = 0
36:12/0 = 0
37:12/0 = 0
38:12/0 = 0
39:12/0 = 0
40:12/0 = 0
41:12/0 = 0
42:12/0 = 0
43:12/0 = 0
44:12/0 = 0
45:12/0 = 0
46:12/0 = 0
47:12/0 = 0
48:12/0 = 0
0:13/0 = 0
1:13/0 = 0
2:13/0 = 0
3:13/0 = 0
4:13/0 = 0
5:13/0 = 0
6:13/0 = 0
7:13/0 = 0
8:13/0 = 0
9:13/0 = 0
10:13/0 = 0
11:13/0 = 0
12:13/0 = 0
13:13/0 = 0
14:13/0 = 0
15:13/0 = 0
16:13/0 = 0
17:13/0 = 0
18:13/0 = 0
19:13/0 = 0
20:13/0 = 0
21:13/0 = 0
22:13/0 = 0
23:13/0 = 0
24:13/0 = 0
25:13/0 = 0
26:13/0 = 0
27:13/0 = 0
28:13/0 = 0
29:13/0 = 0
30:13/0 = 0
31:13/0 = 0
32:13/0 = 0
33:13/0 = 0
34:13/0 = 0
35:13/0 = 0
36:13/0 = 0
37:13/0 = 0
38:13/0 = 0
39:13/0 = 0
40:13/0 = 0
41:13/0 = 0
42:13/0 = 0
43:13/0 = 0
44:13/0 = 0
45:13/0 = 0
46:13/0 = 0
47:13/0 = 0
48:13/0 = 0
0:14/0 = 0
1:14/0 = 0
2:14/0 = 0
3:14/0 = 0
4:14/0 = 0
5:14/0 = 0
6:14/0 = 0
7:14/0 = 0
8:14/0 = 0
9:14/0 = 0
10:14/0 = 0
11:14/0 = 0
12:14/0 = 0
13:14/0 = 0
14:14/0 = 0
15:14/0 = 0
16:14/0 = 0
17:14/0 = 0
18:14/0 = 0
19:14/0 = 0
20:14/0 = 0
21:14/0 = 0
22:14/0 = 0
23:14/0 = 0
24:14/0 = 0
25:14/0 = 0
26:14/0 = 0
27:14/0 = 0
28:14/0 = 0
29:14/0 = 0
30:14/0 = 0
31:14/0 = 0
32:14/0 = 0
33:14/0 = 0
34:14/0 = 0
35:14/0 = 0
36:14/0 = 0
37:14/0 = 0
38:14/0 = 0
39:14/0 = 0
40:14/0 = 0
41:14/0 = 0
42:14/0 = 0
43:14/0 = 0
44:14/0 = 0
45:14/0 = 0
46:14/0 = 0
47:14/0 = 0
48:14/0 = 0
0:15/0 = 0
1:15/0 = 0
2:15/0 = 0
3:15/0 = 0
4:15/0 = 0
5:15/0 = 0
6:15/0 = 0
7:15/0 = 0
8:15/0 = 0
9:15/0 = 0
10:15/0 = 0
11:15/0 = 0
12:15/0 = 0
13:15/0 = 0
14:15/0 = 0
15:15/0 = 0
16:15/0 = 0
17:15/0 = 0
18:15/0 = 0
19:15/0 = 0
20:15/0 = 0
21:15/0 = 0
22:15/0 = 0
23:15/0 = 0
24:15/0 = 0
25:15/0 = 0
26:15/0 = 0
27:15/0 = 0
28:15/0 = 0
29:15/0 = 0
30:15/0 = 0
31:15/0 = 0
32:15/0 = 0
33:15/0 = 0
34:15/0 = 0
35:15/0 = 0
36:15/0 = 0
37:15/0 = 0
38:15/0 = 0
39:15/0 = 0
40:15/0 = 0
41:15/0 = 0
42:15/0 = 0
43:15/0 = 0
44:15/0 = 0
45:15/0 = 0
46:15/0 = 0
47:15/0 = 0
48:15/0 = 0
0:16/0 = 0
1:16/0 = 0
2:16/0 = 0
3:16/0 = 0
4:16/0 = 0
5:16/0 = 0
6:16/0 = 0
7:16/0 = 0
8:16/0 = 0
9:16/0 = 0
10:16/0 = 0
11:16/0 = 0
12:16/0 = 0
13:16/0 = 0
14:16/0 = 0
15:16/0 = 0
16:16/0 = 0
17:16/0 = 0
18:16/0 = 0
19:16/0 = 0
20:16/0 = 0
21:16/0 = 0
22:16/0 = 0
23:16/0 = 0
24:16/0 = 0
25:16/0 = 0
26:16/0 = 0
27:16/0 = 0
28:16/0 = 0
29:16/0 = 0
30:16/0 = 0
31:16/0 = 0
32:16/0 = 0
33:16/0 = 0
34:16/0 = 0
35:16/0 = 0
36:16/0 = 0
37:16/0 = 0
38:16/0 = 0
39:16/0 = 0
40:16/0 = 0
41:16/0 = 0
42:16/0 = 0
43:16/0 = 0
44:16/0 = 0
45:16/0 = 0
46:16/0 = 0
47:16/0 = 0
48:16/0 = 0
0:17/0 = 0
1:17/0 = 0
2:17/0 = 0
3:17/0 = 0
4:17/0 = 0
5:17/0 = 0
6:17/0 = 0
7:17/0 = 0
8:17/0 = 0
9:17/0 = 0
10:17/0 = 0
11:17/0 = 0
12:17/0 = 0
13:17/0 = 0
14:17/0 = 0
15:17/0 = 0
16:17/0 = 0
17:17/0 = 0
18:17/0 = 0
19:17/0 = 0
20:17/0 = 0
21:17/0 = 0
22:17/0 = 0
23:17/0 = 0
24:17/0 = 0
25:17/0 = 0
26:17/0 = 0
27:17/0 = 0
28:17/0 = 0
29:17/0 = 0
30:17/0 = 0
31:17/0 = 0
32:17/0 = 0
33:17/0 = 0
34:17/0 = 0
35:17/0 = 0
36:17/0 = 0
37:17/0 = 0
38:17/0 = 0
39:17/0 = 0
40:17/0 = 0
41:17/0 = 0
42:17/0 = 0
43:17/0 = 0
44:17/0 = 0
45:17/0 = 0
46:17/0 = 0
47:17/0 = 0
48:17/0 = 0
0:18/0 = 0
1:18/0 = 0
2:18/0 = 0
3:18/0 = 0
4:18/0 = 0
5:18/0 = 0
6:18/0 = 0
7:18/0 = 0
8:18/0 = 0
9:18/0 = 0
10:18/0 = 0
11:18/0 = 0
12:18/0 = 0
13:18/0 = 0
14:18/0 = 0
15:18/0 = 0
16:18/0 = 0
17:18/0 = 0
18:18/0 = 0
19:18/0 = 0
20:18/0 = 0
21:18/0 = 0
22:18/0 = 0
23:18/0 = 0
24:18/0 = 0
25:18/0 = 0
26:18/0 = 0
27:18/0 = 0
28:18/0 = 0
29:18/0 = 0
30:18/0 = 0
31:18/0 = 0
32:18/0 = 0
33:18/0 = 0
34:18/0 = 0
35:18/0 = 0
36:18/0 = 0
37:18/0 = 0
38:18/0 = 0
39:18/0 = 0
40:18/0 = 0
41:18/0 = 0
42:18/0 = 0
43:18/0 = 0
44:18/0 = 0
45:18/0 = 0
46:18/0 = 0
47:18/0 = 0
48:18/0 = 0
0:19/0 = 0
1:19/0 = 0
2:19/0 = 0
3:19/0 = 0
4:19/0 = 0
5:19/0 = 0
6:19/0 = 0
7:19/0 = 0
8:19/0 = 0
9:19/0 = 0
10:19/0 = 0
11:19/0 = 0
12:19/0 = 0
13:19/0 = 0
14:19/0 = 0
15:19/0 = 0
16:19/0 = 0
17:19/0 = 0
18:19/0 = 0
19:19/0 = 0
20:19/0 = 0
21:19/0 = 0
22:19/0 = 0
23:19/0 = 0
24:19/0 = 0
25:19/0 = 0
26:19/0 = 0
27:19/0 = 0
28:19/0 = 0
29:19/0 = 0
30:19/0 = 0
31:19/0 = 0
32:19/0 = 0
33:19/0 = 0
34:19/0 = 0
35:19/0 = 0
36:19/0 = 0
37:19/0 = 0
38:19/0 = 0
39:19/0 = 0
40:19/0 = 0
41:19/0 = 0
42:19/0 = 0
43:19/0 = 0
44:19/0 = 0
45:19/0 = 0
46:19/0 = 0
47:19/0 = 0
48:19/0 = 0
0:20/0 = 0
1:20/0 = 0
2:20/0 = 0
3:20/0 = 0
4:20/0 = 0
5:20/0 = 0
6:20/0 = 0
7:20/0 = 0
8:20/0 = 0
9:20/0 = 0
10:20/0 = 0
11:20/0 = 0
12:20/0 = 0
13:20/0 = 0
14:20/0 = 0
15:20/0 = 0
16:20/0 = 0
17:20/0 = 0
18:20/0 = 0
19:20/0 = 0
20:20/0 = 0
21:20/0 = 0
22:20/0 = 0
23:20/0 = 0
24:20/0 = 0
25:20/0 = 0
26:20/0 = 0
27:20/0 = 0
28:20/0 = 0
29:20/0 = 0
30:20/0 = 0
31:20/0 = 0
32:20/0 = 0
33:20/0 = 0
34:20/0 = 0
35:20/0 = 0
36:20/0 = 0
37:20/0 = 0
38:20/0 = 0
39:20/0 = 0
40:20/0 = 0
41:20/0 = 0
42:20/0 = 0
43:20/0 = 0
44:20/0 = 0
45:20/0 = 0
46:20/0 = 0
47:20/0 = 0
48:20/0 = 0
0:21/0 = 0
1:21/0 = 0
2:21/0 = 0
3:21/0 = 0
4:21/0 = 0
5:21/0 = 0
6:21/0 = 0
7:21/0 = 0
8:21/0 = 0
9:21/0 = 0
10:21/0 = 0
11:21/0 = 0
12:21/0 = 0
13:21/0 = 0
14:21/0 = 0
15:21/0 = 0
16:21/0 = 0
17:21/0 = 0
18:21/0 = 0
19:21/0 = 0
20:21/0 = 0
21:21/0 = 0
22:21/0 = 0
23:21/0 = 0
24:21/0 = 0
25:21/0 = 0
26:21/0 = 0
27:21/0 = 0
28:21/0 = 0
29:21/0 = 0
30:21/0 = 0
31:21/0 = 0
32:21/0 = 0
33:21/0 = 0
34:21/0 = 0
35:21/0 = 0
36:21/0 = 0
37:21/0 = 0
38:21/0 = 0
39:21/0 = 0
40:21/0 = 0
41:21/0 = 0
42:21/0 = 0
43:21/0 = 0
44:21/0 = 0
45:21/0 = 0
46:21/0 = 0
47:21/0 = 0
48:21/0 = 0

[sub_resource type="TileSet" id="TileSet_3drs5"]
sources/0 = SubResource("TileSetAtlasSource_6h0bd")

[node name="TileMap" type="TileMap"]
tile_set = SubResource("TileSet_3drs5")
format = 2
script = ExtResource("2_lf4lw")

[node name="AcceptDialog" parent="." instance=ExtResource("3_y08lj")]

[node name="WinDialog" parent="." instance=ExtResource("4_fkys0")]
\end{lstlisting}

\subsection{tile\textunderscore{}map.gd}

\begin{lstlisting}
extends TileMap

const buildings: Array[Vector2i] = [
	Vector2i(0, 19),
	Vector2i(1, 19),
	Vector2i(2, 19),
	Vector2i(3, 19),
	Vector2i(4, 19),
	Vector2i(5, 19),
	Vector2i(6, 19),
	Vector2i(7, 19),
	Vector2i(8, 20),
	Vector2i(0, 20),
	Vector2i(1, 20),
	Vector2i(2, 20),
	Vector2i(3, 20),
	Vector2i(4, 20),
	Vector2i(5, 20),
	Vector2i(6, 20),
	Vector2i(7, 20),
	Vector2i(8, 20),
	Vector2i(0, 21),
	Vector2i(1, 21),
	Vector2i(2, 21),
	Vector2i(3, 21),
	Vector2i(4, 21),
	Vector2i(5, 21),
	Vector2i(6, 21),
	Vector2i(7, 21),
	Vector2i(8, 21)
]
const trees: Array[Vector2i] = [
	Vector2i(0,1),
	Vector2i(1,1),
	Vector2i(2,1),
	Vector2i(3,1),
	Vector2i(4,1),
	Vector2i(5,1),
	Vector2i(6,1),
	Vector2i(7,1),
	Vector2i(0,2),
	Vector2i(1,2),
	Vector2i(2,2),
	Vector2i(3,2),
	Vector2i(4,2)
]
const PLAYER_SPRITE: Vector2i = Vector2i(24, 7)
var player_placement_cell: Vector2i
const rings: Array[Vector2i] = [
	Vector2i(43, 6),
	Vector2i(44, 6),
	Vector2i(45, 6),
	Vector2i(46, 6)
]
var ring_placement_cell: Vector2i

var points: Array[Dictionary] = []
const EUCLIDEAN: String = "Euclidean distance"
const MANHATTAN: String = "Manhattan distance"
@export_enum(EUCLIDEAN, MANHATTAN) var distance: String = MANHATTAN
@export_range(10, 40, 1) var random_starting_points: int = 20
var x_tile_range: int = ProjectSettings.get_setting("display/window/size/viewport_width") / tile_set.tile_size.x
var y_tile_range: int = ProjectSettings.get_setting("display/window/size/viewport_height") / tile_set.tile_size.y

# Called when the node enters the scene tree for the first time.
func _ready() -> void:
	randomize()
	var start_time: float = Time.get_ticks_msec()
	define_points(random_starting_points)
	paint_points()
	place_player()
	place_ring()
	var new_time: float = Time.get_ticks_msec() - start_time
	print("Time taken: " + str(new_time) + "ms")
	$AcceptDialog.dialog_text = "You're a hollow Golem who seeks the ultimate treasure; a ring that's got something on top of it. It's somewhere in this large village and barely visible to your naked eyes, which took us " + str(new_time) + " milliseconds to generate (" + str(new_time / 1000.0) + " seconds), but you'll stop at nothing to get what you want. You can chow down every tree and fauna that stands in your way of the ring, but your Achilles heel is any bricks and mortar, which WILL make you stop at your tracks. Since it's easy to get lost in here, we'll tell you that you're in position " + str(player_placement_cell) + " in this big village of size " + str(Vector2i(x_tile_range, y_tile_range)) + ". It's also easy to get stuck here, so either press the G key or right click to spawn somewhere else where there is fauna (or even the ring!!), because this game actually WANTS you to win it. Ultimately, though, it is YOUR job to find the ring, so are you ready to attain the treasure that is rightfully yours?!"
	$AcceptDialog.visible = true
	$AcceptDialog.confirmed.connect(_on_AcceptDialog_closed)
	$AcceptDialog.canceled.connect(_on_AcceptDialog_closed)
	$WinDialog.confirmed.connect(_on_WinDialog_confirmed)
	$WinDialog.canceled.connect(_on_WinDialog_canceled)
	get_tree().paused = true

func _on_WinDialog_confirmed() -> void:
	get_tree().reload_current_scene()

func _on_WinDialog_canceled() -> void:
	get_tree().quit()

func _on_AcceptDialog_closed() -> void:
	$AcceptDialog.visible = false
	get_tree().paused = false

func _get_random_placement_cell() -> Vector2i:
	return Vector2i(randi() % x_tile_range, randi() % y_tile_range)

func place_player() -> void:
	player_placement_cell = _get_random_placement_cell()
	while buildings.has(get_cell_atlas_coords(0, player_placement_cell)) or player_placement_cell == ring_placement_cell:
		player_placement_cell = _get_random_placement_cell()
	set_cell(0, player_placement_cell, 0, PLAYER_SPRITE)

func place_ring() -> void:
	ring_placement_cell = _get_random_placement_cell()
	while buildings.has(get_cell_atlas_coords(0, ring_placement_cell)) or ring_placement_cell == player_placement_cell:
		ring_placement_cell = _get_random_placement_cell()
	set_cell(0, ring_placement_cell, 0, rings.pick_random())

func _is_not_out_of_bounds(cell: Vector2i) -> bool:
	return cell.x >= 0 and cell.x < x_tile_range and cell.y >= 0 and cell.y < y_tile_range

func _physics_process(_delta) -> void:
	var previous_cell: Vector2i = player_placement_cell
	var direction: Vector2i = Vector2i.ZERO
	if Input.is_action_pressed("ui_up"): direction = Vector2i.UP
	elif Input.is_action_pressed("ui_down"): direction = Vector2i.DOWN
	elif Input.is_action_pressed("ui_left"): direction = Vector2i.LEFT
	elif Input.is_action_pressed("ui_right"): direction = Vector2i.RIGHT
	elif Input.is_action_just_pressed("reset_position"): # Respawn player in a different part of the map
		player_placement_cell = _get_random_placement_cell()
		while buildings.has(get_cell_atlas_coords(0, player_placement_cell)): # This time, since we're not STARTING the game, we don't care whether or not the player magically lands on the ring
			player_placement_cell = _get_random_placement_cell()
		set_cell(0, player_placement_cell, 0, PLAYER_SPRITE)
		set_cell(0, previous_cell, 0) # replace the previous sprite
		return
	var new_placement_cell: Vector2i = player_placement_cell + direction
	if (not get_used_cells(0).has(new_placement_cell) or trees.has(get_cell_atlas_coords(0, new_placement_cell)) or new_placement_cell == ring_placement_cell) and _is_not_out_of_bounds(new_placement_cell):
		player_placement_cell = new_placement_cell
		set_cell(0, previous_cell, 0) # deletes contents of previous cell (atlas_coords = Vector2i(-1, -1))
		set_cell(0, player_placement_cell, 0, PLAYER_SPRITE)
		if player_placement_cell == ring_placement_cell:
			$WinDialog.visible = true
			get_tree().paused = true

# ALGORITHM BEGINS HERE

func paint_points() -> void:
	for point in points:
		set_cell(0, Vector2(point["x"], point["y"]), 0, point["type"])
		for citizen in point["citizens"]:
			if _is_in_bounds(point["x"], citizen["dx"], point["y"], citizen["dy"]):
				set_cell(0, Vector2(point["x"] + citizen["dx"], point["y"] + citizen["dy"]), 0, point["type"])

func _is_in_bounds(x: int, dx: int, y: int, dy: int) -> bool:
	return x + dx >= 0 and x + dx < x_tile_range and y + dy >= 0 and  y + dy < y_tile_range

func _squared(x: int) -> int:
	return x ** 2

func calculate_points_delta(x: int, y: int, p: int) -> float:
	if distance == EUCLIDEAN:
		return sqrt(_squared(points[p]["x"] - x) + _squared(points[p]["y"] - y))
	return abs(points[p]["x"] - x) + abs(points[p]["y"] - y)

func define_points(num_points: int) -> void:
	var types: Array[Vector2i] = trees.duplicate()
	types.append_array(buildings)
	for i in range(num_points):
		var x: int = randi_range(0, x_tile_range)
		var y: int = randi_range(0, y_tile_range)
		var type: Vector2i = types.pick_random()
		types.erase(type)
		points.append(
			{
				"type": type,
				"x": x,
				"y": y,
				"citizens": []
			}
		)
	for x in range(x_tile_range):
		for y in range(y_tile_range):
			var lowest_delta: Dictionary = {
				"point_id": 0,
				"delta": x_tile_range * y_tile_range
			}
			for p in range(len(points)):
				var delta: float = calculate_points_delta(x, y, p)
				if delta < lowest_delta["delta"]:
					lowest_delta = {
						"point_id": p,
						"delta": delta
					}
				var active_point: Dictionary = points[lowest_delta["point_id"]]
				var dx: int = x - active_point["x"]
				var dy: int = y - active_point["y"]
				active_point["citizens"].append(
					{
						"dx": dx,
						"dy": dy
					}
				)
\end{lstlisting}

\subsection{accept\_dialog.tscn}

\begin{lstlisting}
[gd_scene format=3 uid="uid://cau5jgogdnf53"]

[node name="AcceptDialog" type="AcceptDialog"]
title = "Tree-Munching Time!"
position = Vector2i(326, 100)
size = Vector2i(500, 421)
mouse_passthrough = true
ok_button_text = "Bring it on!"
dialog_text = "You're a hollow Golem who seeks the ultimate treasure; a ring that's got something on top of it. It's somewhere in this large village and barely visible to your naked eyes, but you'll stop at nothing to get what you want. You can chow down every tree and fauna that stands in your way of the ring, but your Achilles heel is any bricks and mortar, which will make you stop at your tracks. Are you ready to attain your treasure?w Golem in a black-and-white world, in search for your most desired treasure. It's a ring with something on top of it. And you'll stop at nothing to get what you want. You can chow down every tree and fauna that stands in your way of the ring, but your Achilles heel is any bricks and mortar, which will make you stop at your tracks. Are you ready to attain the treasure that is rightfully yours?!"
dialog_autowrap = true
\end{lstlisting}

\subsection{win\_dialog.tscn}

\begin{lstlisting}
[gd_scene format=3 uid="uid://b5q8ovcigrvyr"]

[node name="WinDialog" type="ConfirmationDialog"]
title = "You Found the Treasure!"
position = Vector2i(326, 100)
size = Vector2i(500, 421)
mouse_passthrough = true
ok_button_text = "Get Me a New Village"
dialog_text = "You found your treasure! Well done, you!

Would you like to travel to a new village in the hopes of finding another ring? Or would you like to take your treasure home now?"
dialog_autowrap = true
cancel_button_text = "Get Me Out of Here"
\end{lstlisting}

\subsection{icon.svg.import}

\begin{lstlisting}
[remap]

importer="texture"
type="CompressedTexture2D"
uid="uid://uotfe6soknht"
path="res://.godot/imported/icon.svg-218a8f2b3041327d8a5756f3a245f83b.ctex"
metadata={
"vram_texture": false
}

[deps]

source_file="res://icon.svg"
dest_files=["res://.godot/imported/icon.svg-218a8f2b3041327d8a5756f3a245f83b.ctex"]

[params]

compress/mode=0
compress/high_quality=false
compress/lossy_quality=0.7
compress/hdr_compression=1
compress/normal_map=0
compress/channel_pack=0
mipmaps/generate=false
mipmaps/limit=-1
roughness/mode=0
roughness/src_normal=""
process/fix_alpha_border=true
process/premult_alpha=false
process/normal_map_invert_y=false
process/hdr_as_srgb=false
process/hdr_clamp_exposure=false
process/size_limit=0
detect_3d/compress_to=1
svg/scale=1.0
editor/scale_with_editor_scale=false
editor/convert_colors_with_editor_theme=false
\end{lstlisting}

\subsection{monochrome\textunderscore{}packed.png.import}

\begin{lstlisting}
[remap]

importer="texture"
type="CompressedTexture2D"
uid="uid://c3bpsm4r8t504"
path="res://.godot/imported/monochrome_packed.png-6b9bd1c64dd50f72acd3afd14d1ac34f.ctex"
metadata={
"vram_texture": false
}

[deps]

source_file="res://monochrome_packed.png"
dest_files=["res://.godot/imported/monochrome_packed.png-6b9bd1c64dd50f72acd3afd14d1ac34f.ctex"]

[params]

compress/mode=0
compress/high_quality=false
compress/lossy_quality=0.7
compress/hdr_compression=1
compress/normal_map=0
compress/channel_pack=0
mipmaps/generate=false
mipmaps/limit=-1
roughness/mode=0
roughness/src_normal=""
process/fix_alpha_border=true
process/premult_alpha=false
process/normal_map_invert_y=false
process/hdr_as_srgb=false
process/hdr_clamp_exposure=false
process/size_limit=0
detect_3d/compress_to=1
\end{lstlisting}

\section{GD4NoiseRPG}

% \localtableofcontents{}

\subsection{.gitattributes}

\begin{lstlisting}
# Normalize EOL for all files that Git considers text files.
* text=auto eol=lf
\end{lstlisting}

\subsection{.gitignore}

\begin{lstlisting}
# Godot 4+ specific ignores
.godot/
\end{lstlisting}

\subsection{project.godot}

\begin{lstlisting}
; Engine configuration file.
; It's best edited using the editor UI and not directly,
; since the parameters that go here are not all obvious.
;
; Format:
;   [section] ; section goes between []
;   param=value ; assign values to parameters

config_version=5

[application]

config/name="Noise Demo"
run/main_scene="res://tile_map.tscn"
config/features=PackedStringArray("4.0", "Forward Plus")
config/icon="res://icon.svg"

[display]

window/size/viewport_height=640

[rendering]

environment/defaults/default_clear_color=Color(0, 0, 0, 1)
\end{lstlisting}

\subsection{tile\textunderscore{}map.tscn}

\begin{lstlisting}
[gd_scene load_steps=7 format=3 uid="uid://d4jdcavluwx6s"]

[ext_resource type="Texture2D" uid="uid://m662wwd4prmn" path="res://monochrome_packed.png" id="1_ld7xx"]
[ext_resource type="Script" path="res://tile_map.gd" id="2_o1dn1"]
[ext_resource type="PackedScene" uid="uid://cau5jgogdnf53" path="res://accept_dialog.tscn" id="3_e0ur6"]
[ext_resource type="PackedScene" uid="uid://b5q8ovcigrvyr" path="res://win_dialog.tscn" id="4_ecfaa"]

[sub_resource type="TileSetAtlasSource" id="TileSetAtlasSource_1e80b"]
texture = ExtResource("1_ld7xx")
0:0/0 = 0
1:0/0 = 0
2:0/0 = 0
3:0/0 = 0
4:0/0 = 0
5:0/0 = 0
6:0/0 = 0
7:0/0 = 0
8:0/0 = 0
9:0/0 = 0
10:0/0 = 0
11:0/0 = 0
12:0/0 = 0
13:0/0 = 0
14:0/0 = 0
15:0/0 = 0
16:0/0 = 0
17:0/0 = 0
18:0/0 = 0
19:0/0 = 0
20:0/0 = 0
21:0/0 = 0
22:0/0 = 0
23:0/0 = 0
24:0/0 = 0
25:0/0 = 0
26:0/0 = 0
27:0/0 = 0
28:0/0 = 0
29:0/0 = 0
30:0/0 = 0
31:0/0 = 0
32:0/0 = 0
33:0/0 = 0
34:0/0 = 0
35:0/0 = 0
36:0/0 = 0
37:0/0 = 0
38:0/0 = 0
39:0/0 = 0
40:0/0 = 0
41:0/0 = 0
42:0/0 = 0
43:0/0 = 0
44:0/0 = 0
45:0/0 = 0
46:0/0 = 0
47:0/0 = 0
48:0/0 = 0
0:1/0 = 0
1:1/0 = 0
2:1/0 = 0
3:1/0 = 0
4:1/0 = 0
5:1/0 = 0
6:1/0 = 0
7:1/0 = 0
8:1/0 = 0
9:1/0 = 0
10:1/0 = 0
11:1/0 = 0
12:1/0 = 0
13:1/0 = 0
14:1/0 = 0
15:1/0 = 0
16:1/0 = 0
17:1/0 = 0
18:1/0 = 0
19:1/0 = 0
20:1/0 = 0
21:1/0 = 0
22:1/0 = 0
23:1/0 = 0
24:1/0 = 0
25:1/0 = 0
26:1/0 = 0
27:1/0 = 0
28:1/0 = 0
29:1/0 = 0
30:1/0 = 0
31:1/0 = 0
32:1/0 = 0
33:1/0 = 0
34:1/0 = 0
35:1/0 = 0
36:1/0 = 0
37:1/0 = 0
38:1/0 = 0
39:1/0 = 0
40:1/0 = 0
41:1/0 = 0
42:1/0 = 0
43:1/0 = 0
44:1/0 = 0
45:1/0 = 0
46:1/0 = 0
47:1/0 = 0
48:1/0 = 0
0:2/0 = 0
1:2/0 = 0
2:2/0 = 0
3:2/0 = 0
4:2/0 = 0
5:2/0 = 0
6:2/0 = 0
7:2/0 = 0
8:2/0 = 0
9:2/0 = 0
10:2/0 = 0
11:2/0 = 0
12:2/0 = 0
13:2/0 = 0
14:2/0 = 0
15:2/0 = 0
16:2/0 = 0
17:2/0 = 0
18:2/0 = 0
19:2/0 = 0
20:2/0 = 0
21:2/0 = 0
22:2/0 = 0
23:2/0 = 0
24:2/0 = 0
25:2/0 = 0
26:2/0 = 0
27:2/0 = 0
28:2/0 = 0
29:2/0 = 0
30:2/0 = 0
31:2/0 = 0
32:2/0 = 0
33:2/0 = 0
34:2/0 = 0
35:2/0 = 0
36:2/0 = 0
37:2/0 = 0
38:2/0 = 0
39:2/0 = 0
40:2/0 = 0
41:2/0 = 0
42:2/0 = 0
43:2/0 = 0
44:2/0 = 0
45:2/0 = 0
46:2/0 = 0
47:2/0 = 0
48:2/0 = 0
0:3/0 = 0
1:3/0 = 0
2:3/0 = 0
3:3/0 = 0
4:3/0 = 0
5:3/0 = 0
6:3/0 = 0
7:3/0 = 0
8:3/0 = 0
9:3/0 = 0
10:3/0 = 0
11:3/0 = 0
12:3/0 = 0
13:3/0 = 0
14:3/0 = 0
15:3/0 = 0
16:3/0 = 0
17:3/0 = 0
18:3/0 = 0
19:3/0 = 0
20:3/0 = 0
21:3/0 = 0
22:3/0 = 0
23:3/0 = 0
24:3/0 = 0
25:3/0 = 0
26:3/0 = 0
27:3/0 = 0
28:3/0 = 0
29:3/0 = 0
30:3/0 = 0
31:3/0 = 0
32:3/0 = 0
33:3/0 = 0
34:3/0 = 0
35:3/0 = 0
36:3/0 = 0
37:3/0 = 0
38:3/0 = 0
39:3/0 = 0
40:3/0 = 0
41:3/0 = 0
42:3/0 = 0
43:3/0 = 0
44:3/0 = 0
45:3/0 = 0
46:3/0 = 0
47:3/0 = 0
48:3/0 = 0
0:4/0 = 0
1:4/0 = 0
2:4/0 = 0
3:4/0 = 0
4:4/0 = 0
5:4/0 = 0
6:4/0 = 0
7:4/0 = 0
8:4/0 = 0
9:4/0 = 0
10:4/0 = 0
11:4/0 = 0
12:4/0 = 0
13:4/0 = 0
14:4/0 = 0
15:4/0 = 0
16:4/0 = 0
17:4/0 = 0
18:4/0 = 0
19:4/0 = 0
20:4/0 = 0
21:4/0 = 0
22:4/0 = 0
23:4/0 = 0
24:4/0 = 0
25:4/0 = 0
26:4/0 = 0
27:4/0 = 0
28:4/0 = 0
29:4/0 = 0
30:4/0 = 0
31:4/0 = 0
32:4/0 = 0
33:4/0 = 0
34:4/0 = 0
35:4/0 = 0
36:4/0 = 0
37:4/0 = 0
38:4/0 = 0
39:4/0 = 0
40:4/0 = 0
41:4/0 = 0
42:4/0 = 0
43:4/0 = 0
44:4/0 = 0
45:4/0 = 0
46:4/0 = 0
47:4/0 = 0
48:4/0 = 0
0:5/0 = 0
1:5/0 = 0
2:5/0 = 0
3:5/0 = 0
4:5/0 = 0
5:5/0 = 0
6:5/0 = 0
7:5/0 = 0
8:5/0 = 0
9:5/0 = 0
10:5/0 = 0
11:5/0 = 0
12:5/0 = 0
13:5/0 = 0
14:5/0 = 0
15:5/0 = 0
16:5/0 = 0
17:5/0 = 0
18:5/0 = 0
19:5/0 = 0
20:5/0 = 0
21:5/0 = 0
22:5/0 = 0
23:5/0 = 0
24:5/0 = 0
25:5/0 = 0
26:5/0 = 0
27:5/0 = 0
28:5/0 = 0
29:5/0 = 0
30:5/0 = 0
31:5/0 = 0
32:5/0 = 0
33:5/0 = 0
34:5/0 = 0
35:5/0 = 0
36:5/0 = 0
37:5/0 = 0
38:5/0 = 0
39:5/0 = 0
40:5/0 = 0
41:5/0 = 0
42:5/0 = 0
43:5/0 = 0
44:5/0 = 0
45:5/0 = 0
46:5/0 = 0
47:5/0 = 0
48:5/0 = 0
0:6/0 = 0
1:6/0 = 0
2:6/0 = 0
3:6/0 = 0
4:6/0 = 0
5:6/0 = 0
6:6/0 = 0
7:6/0 = 0
8:6/0 = 0
9:6/0 = 0
10:6/0 = 0
11:6/0 = 0
12:6/0 = 0
13:6/0 = 0
14:6/0 = 0
15:6/0 = 0
16:6/0 = 0
17:6/0 = 0
18:6/0 = 0
19:6/0 = 0
20:6/0 = 0
21:6/0 = 0
22:6/0 = 0
23:6/0 = 0
24:6/0 = 0
25:6/0 = 0
26:6/0 = 0
27:6/0 = 0
28:6/0 = 0
29:6/0 = 0
30:6/0 = 0
31:6/0 = 0
32:6/0 = 0
33:6/0 = 0
34:6/0 = 0
35:6/0 = 0
36:6/0 = 0
37:6/0 = 0
38:6/0 = 0
39:6/0 = 0
40:6/0 = 0
41:6/0 = 0
42:6/0 = 0
43:6/0 = 0
44:6/0 = 0
45:6/0 = 0
46:6/0 = 0
47:6/0 = 0
48:6/0 = 0
0:7/0 = 0
1:7/0 = 0
2:7/0 = 0
3:7/0 = 0
4:7/0 = 0
5:7/0 = 0
6:7/0 = 0
7:7/0 = 0
8:7/0 = 0
9:7/0 = 0
10:7/0 = 0
11:7/0 = 0
12:7/0 = 0
13:7/0 = 0
14:7/0 = 0
15:7/0 = 0
16:7/0 = 0
17:7/0 = 0
18:7/0 = 0
19:7/0 = 0
20:7/0 = 0
21:7/0 = 0
22:7/0 = 0
23:7/0 = 0
24:7/0 = 0
25:7/0 = 0
26:7/0 = 0
27:7/0 = 0
28:7/0 = 0
29:7/0 = 0
30:7/0 = 0
31:7/0 = 0
32:7/0 = 0
33:7/0 = 0
34:7/0 = 0
35:7/0 = 0
36:7/0 = 0
37:7/0 = 0
38:7/0 = 0
39:7/0 = 0
40:7/0 = 0
41:7/0 = 0
42:7/0 = 0
43:7/0 = 0
44:7/0 = 0
45:7/0 = 0
46:7/0 = 0
47:7/0 = 0
48:7/0 = 0
0:8/0 = 0
1:8/0 = 0
2:8/0 = 0
3:8/0 = 0
4:8/0 = 0
5:8/0 = 0
6:8/0 = 0
7:8/0 = 0
8:8/0 = 0
9:8/0 = 0
10:8/0 = 0
11:8/0 = 0
12:8/0 = 0
13:8/0 = 0
14:8/0 = 0
15:8/0 = 0
16:8/0 = 0
17:8/0 = 0
18:8/0 = 0
19:8/0 = 0
20:8/0 = 0
21:8/0 = 0
22:8/0 = 0
23:8/0 = 0
24:8/0 = 0
25:8/0 = 0
26:8/0 = 0
27:8/0 = 0
28:8/0 = 0
29:8/0 = 0
30:8/0 = 0
31:8/0 = 0
32:8/0 = 0
33:8/0 = 0
34:8/0 = 0
35:8/0 = 0
36:8/0 = 0
37:8/0 = 0
38:8/0 = 0
39:8/0 = 0
40:8/0 = 0
41:8/0 = 0
42:8/0 = 0
43:8/0 = 0
44:8/0 = 0
45:8/0 = 0
46:8/0 = 0
47:8/0 = 0
48:8/0 = 0
0:9/0 = 0
1:9/0 = 0
2:9/0 = 0
3:9/0 = 0
4:9/0 = 0
5:9/0 = 0
6:9/0 = 0
7:9/0 = 0
8:9/0 = 0
9:9/0 = 0
10:9/0 = 0
11:9/0 = 0
12:9/0 = 0
13:9/0 = 0
14:9/0 = 0
15:9/0 = 0
16:9/0 = 0
17:9/0 = 0
18:9/0 = 0
19:9/0 = 0
20:9/0 = 0
21:9/0 = 0
22:9/0 = 0
23:9/0 = 0
24:9/0 = 0
25:9/0 = 0
26:9/0 = 0
27:9/0 = 0
28:9/0 = 0
29:9/0 = 0
30:9/0 = 0
31:9/0 = 0
32:9/0 = 0
33:9/0 = 0
34:9/0 = 0
35:9/0 = 0
36:9/0 = 0
37:9/0 = 0
38:9/0 = 0
39:9/0 = 0
40:9/0 = 0
41:9/0 = 0
42:9/0 = 0
43:9/0 = 0
44:9/0 = 0
45:9/0 = 0
46:9/0 = 0
47:9/0 = 0
48:9/0 = 0
0:10/0 = 0
1:10/0 = 0
2:10/0 = 0
3:10/0 = 0
4:10/0 = 0
5:10/0 = 0
6:10/0 = 0
7:10/0 = 0
8:10/0 = 0
9:10/0 = 0
10:10/0 = 0
11:10/0 = 0
12:10/0 = 0
13:10/0 = 0
14:10/0 = 0
15:10/0 = 0
16:10/0 = 0
17:10/0 = 0
18:10/0 = 0
19:10/0 = 0
20:10/0 = 0
21:10/0 = 0
22:10/0 = 0
23:10/0 = 0
24:10/0 = 0
25:10/0 = 0
26:10/0 = 0
27:10/0 = 0
28:10/0 = 0
29:10/0 = 0
30:10/0 = 0
31:10/0 = 0
32:10/0 = 0
33:10/0 = 0
34:10/0 = 0
35:10/0 = 0
36:10/0 = 0
37:10/0 = 0
38:10/0 = 0
39:10/0 = 0
40:10/0 = 0
41:10/0 = 0
42:10/0 = 0
43:10/0 = 0
44:10/0 = 0
45:10/0 = 0
46:10/0 = 0
47:10/0 = 0
48:10/0 = 0
0:11/0 = 0
1:11/0 = 0
2:11/0 = 0
3:11/0 = 0
4:11/0 = 0
5:11/0 = 0
6:11/0 = 0
7:11/0 = 0
8:11/0 = 0
9:11/0 = 0
10:11/0 = 0
11:11/0 = 0
12:11/0 = 0
13:11/0 = 0
14:11/0 = 0
15:11/0 = 0
16:11/0 = 0
17:11/0 = 0
18:11/0 = 0
19:11/0 = 0
20:11/0 = 0
21:11/0 = 0
22:11/0 = 0
23:11/0 = 0
24:11/0 = 0
25:11/0 = 0
26:11/0 = 0
27:11/0 = 0
28:11/0 = 0
29:11/0 = 0
30:11/0 = 0
31:11/0 = 0
32:11/0 = 0
33:11/0 = 0
34:11/0 = 0
35:11/0 = 0
36:11/0 = 0
37:11/0 = 0
38:11/0 = 0
39:11/0 = 0
40:11/0 = 0
41:11/0 = 0
42:11/0 = 0
43:11/0 = 0
44:11/0 = 0
45:11/0 = 0
46:11/0 = 0
47:11/0 = 0
48:11/0 = 0
0:12/0 = 0
1:12/0 = 0
2:12/0 = 0
3:12/0 = 0
4:12/0 = 0
5:12/0 = 0
6:12/0 = 0
7:12/0 = 0
8:12/0 = 0
9:12/0 = 0
10:12/0 = 0
11:12/0 = 0
12:12/0 = 0
13:12/0 = 0
14:12/0 = 0
15:12/0 = 0
16:12/0 = 0
17:12/0 = 0
18:12/0 = 0
19:12/0 = 0
20:12/0 = 0
21:12/0 = 0
22:12/0 = 0
23:12/0 = 0
24:12/0 = 0
25:12/0 = 0
26:12/0 = 0
27:12/0 = 0
28:12/0 = 0
29:12/0 = 0
30:12/0 = 0
31:12/0 = 0
32:12/0 = 0
33:12/0 = 0
34:12/0 = 0
35:12/0 = 0
36:12/0 = 0
37:12/0 = 0
38:12/0 = 0
39:12/0 = 0
40:12/0 = 0
41:12/0 = 0
42:12/0 = 0
43:12/0 = 0
44:12/0 = 0
45:12/0 = 0
46:12/0 = 0
47:12/0 = 0
48:12/0 = 0
0:13/0 = 0
1:13/0 = 0
2:13/0 = 0
3:13/0 = 0
4:13/0 = 0
5:13/0 = 0
6:13/0 = 0
7:13/0 = 0
8:13/0 = 0
9:13/0 = 0
10:13/0 = 0
11:13/0 = 0
12:13/0 = 0
13:13/0 = 0
14:13/0 = 0
15:13/0 = 0
16:13/0 = 0
17:13/0 = 0
18:13/0 = 0
19:13/0 = 0
20:13/0 = 0
21:13/0 = 0
22:13/0 = 0
23:13/0 = 0
24:13/0 = 0
25:13/0 = 0
26:13/0 = 0
27:13/0 = 0
28:13/0 = 0
29:13/0 = 0
30:13/0 = 0
31:13/0 = 0
32:13/0 = 0
33:13/0 = 0
34:13/0 = 0
35:13/0 = 0
36:13/0 = 0
37:13/0 = 0
38:13/0 = 0
39:13/0 = 0
40:13/0 = 0
41:13/0 = 0
42:13/0 = 0
43:13/0 = 0
44:13/0 = 0
45:13/0 = 0
46:13/0 = 0
47:13/0 = 0
48:13/0 = 0
0:14/0 = 0
1:14/0 = 0
2:14/0 = 0
3:14/0 = 0
4:14/0 = 0
5:14/0 = 0
6:14/0 = 0
7:14/0 = 0
8:14/0 = 0
9:14/0 = 0
10:14/0 = 0
11:14/0 = 0
12:14/0 = 0
13:14/0 = 0
14:14/0 = 0
15:14/0 = 0
16:14/0 = 0
17:14/0 = 0
18:14/0 = 0
19:14/0 = 0
20:14/0 = 0
21:14/0 = 0
22:14/0 = 0
23:14/0 = 0
24:14/0 = 0
25:14/0 = 0
26:14/0 = 0
27:14/0 = 0
28:14/0 = 0
29:14/0 = 0
30:14/0 = 0
31:14/0 = 0
32:14/0 = 0
33:14/0 = 0
34:14/0 = 0
35:14/0 = 0
36:14/0 = 0
37:14/0 = 0
38:14/0 = 0
39:14/0 = 0
40:14/0 = 0
41:14/0 = 0
42:14/0 = 0
43:14/0 = 0
44:14/0 = 0
45:14/0 = 0
46:14/0 = 0
47:14/0 = 0
48:14/0 = 0
0:15/0 = 0
1:15/0 = 0
2:15/0 = 0
3:15/0 = 0
4:15/0 = 0
5:15/0 = 0
6:15/0 = 0
7:15/0 = 0
8:15/0 = 0
9:15/0 = 0
10:15/0 = 0
11:15/0 = 0
12:15/0 = 0
13:15/0 = 0
14:15/0 = 0
15:15/0 = 0
16:15/0 = 0
17:15/0 = 0
18:15/0 = 0
19:15/0 = 0
20:15/0 = 0
21:15/0 = 0
22:15/0 = 0
23:15/0 = 0
24:15/0 = 0
25:15/0 = 0
26:15/0 = 0
27:15/0 = 0
28:15/0 = 0
29:15/0 = 0
30:15/0 = 0
31:15/0 = 0
32:15/0 = 0
33:15/0 = 0
34:15/0 = 0
35:15/0 = 0
36:15/0 = 0
37:15/0 = 0
38:15/0 = 0
39:15/0 = 0
40:15/0 = 0
41:15/0 = 0
42:15/0 = 0
43:15/0 = 0
44:15/0 = 0
45:15/0 = 0
46:15/0 = 0
47:15/0 = 0
48:15/0 = 0
0:16/0 = 0
1:16/0 = 0
2:16/0 = 0
3:16/0 = 0
4:16/0 = 0
5:16/0 = 0
6:16/0 = 0
7:16/0 = 0
8:16/0 = 0
9:16/0 = 0
10:16/0 = 0
11:16/0 = 0
12:16/0 = 0
13:16/0 = 0
14:16/0 = 0
15:16/0 = 0
16:16/0 = 0
17:16/0 = 0
18:16/0 = 0
19:16/0 = 0
20:16/0 = 0
21:16/0 = 0
22:16/0 = 0
23:16/0 = 0
24:16/0 = 0
25:16/0 = 0
26:16/0 = 0
27:16/0 = 0
28:16/0 = 0
29:16/0 = 0
30:16/0 = 0
31:16/0 = 0
32:16/0 = 0
33:16/0 = 0
34:16/0 = 0
35:16/0 = 0
36:16/0 = 0
37:16/0 = 0
38:16/0 = 0
39:16/0 = 0
40:16/0 = 0
41:16/0 = 0
42:16/0 = 0
43:16/0 = 0
44:16/0 = 0
45:16/0 = 0
46:16/0 = 0
47:16/0 = 0
48:16/0 = 0
0:17/0 = 0
1:17/0 = 0
2:17/0 = 0
3:17/0 = 0
4:17/0 = 0
5:17/0 = 0
6:17/0 = 0
7:17/0 = 0
8:17/0 = 0
9:17/0 = 0
10:17/0 = 0
11:17/0 = 0
12:17/0 = 0
13:17/0 = 0
14:17/0 = 0
15:17/0 = 0
16:17/0 = 0
17:17/0 = 0
18:17/0 = 0
19:17/0 = 0
20:17/0 = 0
21:17/0 = 0
22:17/0 = 0
23:17/0 = 0
24:17/0 = 0
25:17/0 = 0
26:17/0 = 0
27:17/0 = 0
28:17/0 = 0
29:17/0 = 0
30:17/0 = 0
31:17/0 = 0
32:17/0 = 0
33:17/0 = 0
34:17/0 = 0
35:17/0 = 0
36:17/0 = 0
37:17/0 = 0
38:17/0 = 0
39:17/0 = 0
40:17/0 = 0
41:17/0 = 0
42:17/0 = 0
43:17/0 = 0
44:17/0 = 0
45:17/0 = 0
46:17/0 = 0
47:17/0 = 0
48:17/0 = 0
0:18/0 = 0
1:18/0 = 0
2:18/0 = 0
3:18/0 = 0
4:18/0 = 0
5:18/0 = 0
6:18/0 = 0
7:18/0 = 0
8:18/0 = 0
9:18/0 = 0
10:18/0 = 0
11:18/0 = 0
12:18/0 = 0
13:18/0 = 0
14:18/0 = 0
15:18/0 = 0
16:18/0 = 0
17:18/0 = 0
18:18/0 = 0
19:18/0 = 0
20:18/0 = 0
21:18/0 = 0
22:18/0 = 0
23:18/0 = 0
24:18/0 = 0
25:18/0 = 0
26:18/0 = 0
27:18/0 = 0
28:18/0 = 0
29:18/0 = 0
30:18/0 = 0
31:18/0 = 0
32:18/0 = 0
33:18/0 = 0
34:18/0 = 0
35:18/0 = 0
36:18/0 = 0
37:18/0 = 0
38:18/0 = 0
39:18/0 = 0
40:18/0 = 0
41:18/0 = 0
42:18/0 = 0
43:18/0 = 0
44:18/0 = 0
45:18/0 = 0
46:18/0 = 0
47:18/0 = 0
48:18/0 = 0
0:19/0 = 0
1:19/0 = 0
2:19/0 = 0
3:19/0 = 0
4:19/0 = 0
5:19/0 = 0
6:19/0 = 0
7:19/0 = 0
8:19/0 = 0
9:19/0 = 0
10:19/0 = 0
11:19/0 = 0
12:19/0 = 0
13:19/0 = 0
14:19/0 = 0
15:19/0 = 0
16:19/0 = 0
17:19/0 = 0
18:19/0 = 0
19:19/0 = 0
20:19/0 = 0
21:19/0 = 0
22:19/0 = 0
23:19/0 = 0
24:19/0 = 0
25:19/0 = 0
26:19/0 = 0
27:19/0 = 0
28:19/0 = 0
29:19/0 = 0
30:19/0 = 0
31:19/0 = 0
32:19/0 = 0
33:19/0 = 0
34:19/0 = 0
35:19/0 = 0
36:19/0 = 0
37:19/0 = 0
38:19/0 = 0
39:19/0 = 0
40:19/0 = 0
41:19/0 = 0
42:19/0 = 0
43:19/0 = 0
44:19/0 = 0
45:19/0 = 0
46:19/0 = 0
47:19/0 = 0
48:19/0 = 0
0:20/0 = 0
1:20/0 = 0
2:20/0 = 0
3:20/0 = 0
4:20/0 = 0
5:20/0 = 0
6:20/0 = 0
7:20/0 = 0
8:20/0 = 0
9:20/0 = 0
10:20/0 = 0
11:20/0 = 0
12:20/0 = 0
13:20/0 = 0
14:20/0 = 0
15:20/0 = 0
16:20/0 = 0
17:20/0 = 0
18:20/0 = 0
19:20/0 = 0
20:20/0 = 0
21:20/0 = 0
22:20/0 = 0
23:20/0 = 0
24:20/0 = 0
25:20/0 = 0
26:20/0 = 0
27:20/0 = 0
28:20/0 = 0
29:20/0 = 0
30:20/0 = 0
31:20/0 = 0
32:20/0 = 0
33:20/0 = 0
34:20/0 = 0
35:20/0 = 0
36:20/0 = 0
37:20/0 = 0
38:20/0 = 0
39:20/0 = 0
40:20/0 = 0
41:20/0 = 0
42:20/0 = 0
43:20/0 = 0
44:20/0 = 0
45:20/0 = 0
46:20/0 = 0
47:20/0 = 0
48:20/0 = 0
0:21/0 = 0
1:21/0 = 0
2:21/0 = 0
3:21/0 = 0
4:21/0 = 0
5:21/0 = 0
6:21/0 = 0
7:21/0 = 0
8:21/0 = 0
9:21/0 = 0
10:21/0 = 0
11:21/0 = 0
12:21/0 = 0
13:21/0 = 0
14:21/0 = 0
15:21/0 = 0
16:21/0 = 0
17:21/0 = 0
18:21/0 = 0
19:21/0 = 0
20:21/0 = 0
21:21/0 = 0
22:21/0 = 0
23:21/0 = 0
24:21/0 = 0
25:21/0 = 0
26:21/0 = 0
27:21/0 = 0
28:21/0 = 0
29:21/0 = 0
30:21/0 = 0
31:21/0 = 0
32:21/0 = 0
33:21/0 = 0
34:21/0 = 0
35:21/0 = 0
36:21/0 = 0
37:21/0 = 0
38:21/0 = 0
39:21/0 = 0
40:21/0 = 0
41:21/0 = 0
42:21/0 = 0
43:21/0 = 0
44:21/0 = 0
45:21/0 = 0
46:21/0 = 0
47:21/0 = 0
48:21/0 = 0

[sub_resource type="TileSet" id="TileSet_qtrb6"]
sources/0 = SubResource("TileSetAtlasSource_1e80b")

[node name="TileMap" type="TileMap"]
texture_filter = 1
tile_set = SubResource("TileSet_qtrb6")
format = 2
script = ExtResource("2_o1dn1")

[node name="AcceptDialog" parent="." instance=ExtResource("3_e0ur6")]

[node name="WinDialog" parent="." instance=ExtResource("4_ecfaa")]
title = "You Found the Treasure!"
\end{lstlisting}

\subsection{tile\textunderscore{}map.gd}

\begin{lstlisting}
extends TileMap

const buildings: Array[Vector2i] = [
	Vector2i(0, 19),
	Vector2i(1, 19),
	Vector2i(2, 19),
	Vector2i(3, 19),
	Vector2i(4, 19),
	Vector2i(5, 19),
	Vector2i(6, 19),
	Vector2i(7, 19),
	Vector2i(8, 20),
	Vector2i(0, 20),
	Vector2i(1, 20),
	Vector2i(2, 20),
	Vector2i(3, 20),
	Vector2i(4, 20),
	Vector2i(5, 20),
	Vector2i(6, 20),
	Vector2i(7, 20),
	Vector2i(8, 20),
	Vector2i(0, 21),
	Vector2i(1, 21),
	Vector2i(2, 21),
	Vector2i(3, 21),
	Vector2i(4, 21),
	Vector2i(5, 21),
	Vector2i(6, 21),
	Vector2i(7, 21),
	Vector2i(8, 21)
]
const trees: Array[Vector2i] = [
	Vector2i(0,1),
	Vector2i(1,1),
	Vector2i(2,1),
	Vector2i(3,1),
	Vector2i(4,1),
	Vector2i(5,1),
	Vector2i(6,1),
	Vector2i(7,1),
	Vector2i(0,2),
	Vector2i(1,2),
	Vector2i(2,2),
	Vector2i(3,2),
	Vector2i(4,2)
]
const PLAYER_SPRITE: Vector2i = Vector2i(24, 7)
var player_placement_cell: Vector2i
const rings: Array[Vector2i] = [
	Vector2i(43, 6),
	Vector2i(44, 6),
	Vector2i(45, 6),
	Vector2i(46, 6)
]
var ring_placement_cell: Vector2i

var noise: FastNoiseLite
@export_enum("Perlin", "Simplex", "Simplex Smooth", "Value", "Value Cubic") var noise_type: String = "Simplex Smooth"
@export var fractal_type: FastNoiseLite.FractalType
@export var cellular_distance_type: FastNoiseLite.CellularDistanceFunction
#@export_range(1, 10, 1) var octaves: int = 5 
@export_range(0.0, 1.0) var noise_frequency: float = 0.894

#@onready var timer: Timer = Timer.new()
#@export_range(10, 200, 10) var player_movement_speed: int = 100 
@export_range(-1.0, 1.0) var tree_cap: float = -0.048
@export_range(-1.0, 1.0) var building_cap: float = -0.252
@export_range(0.0, 0.5) var building_overtakes_tree: float = 0.12
var x_tile_range: int = ProjectSettings.get_setting("display/window/size/viewport_width") / tile_set.tile_size.x
var y_tile_range: int = ProjectSettings.get_setting("display/window/size/viewport_height") / tile_set.tile_size.y

# Called when the node enters the scene tree for the first time.
func _ready() -> void:
	randomize()
	var start_time: float = Time.get_ticks_msec()
	set_noise()
	paint_tiles()
	place_player()
	place_ring()
	var new_time: float = Time.get_ticks_msec() - start_time
	print("Time taken: " + str(new_time) + "ms")
	$AcceptDialog.dialog_text = "You're a hollow Golem who seeks the ultimate treasure; a ring that's got something on top of it. It's somewhere in this large village and barely visible to your naked eyes, which took us " + str(new_time) + " milliseconds to generate (" + str(new_time / 1000.0) + " seconds), but you'll stop at nothing to get what you want. You can chow down every tree and fauna that stands in your way of the ring, but your Achilles heel is any bricks and mortar, which WILL make you stop at your tracks. Since it's easy to get lost in here, we'll tell you that you're in position " + str(player_placement_cell) + " in this big village of size " + str(Vector2i(x_tile_range, y_tile_range)) + ". However, it is YOUR job to find the ring, so are you ready to attain the treasure that is rightfully yours?!"
	$AcceptDialog.visible = true
	$AcceptDialog.confirmed.connect(_on_AcceptDialog_closed)
	$AcceptDialog.canceled.connect(_on_AcceptDialog_closed)
	$WinDialog.confirmed.connect(_on_WinDialog_confirmed)
	$WinDialog.canceled.connect(_on_WinDialog_canceled)
	get_tree().paused = true

func _on_WinDialog_confirmed() -> void:
	get_tree().reload_current_scene()

func _on_WinDialog_canceled() -> void:
	get_tree().quit()

func _on_AcceptDialog_closed() -> void:
	$AcceptDialog.visible = false
	get_tree().paused = false

func _get_random_placement_cell() -> Vector2i:
	return Vector2i(randi() % x_tile_range, randi() % y_tile_range)

func place_player() -> void:
	player_placement_cell = _get_random_placement_cell()
	while get_used_cells(0).has(player_placement_cell):
		player_placement_cell = _get_random_placement_cell()
	set_cell(0, player_placement_cell, 0, PLAYER_SPRITE)

func place_ring() -> void:
	ring_placement_cell = _get_random_placement_cell()
	while get_used_cells(0).has(ring_placement_cell):
		ring_placement_cell = _get_random_placement_cell()
	set_cell(0, ring_placement_cell, 0, rings.pick_random())

func _is_not_out_of_bounds(cell: Vector2i) -> bool:
	return cell.x >= 0 and cell.x < x_tile_range and cell.y >= 0 and cell.y < y_tile_range

func _physics_process(_delta: float) -> void:
	var previous_cell: Vector2i = player_placement_cell
	var direction: Vector2i = Vector2i.ZERO
	if Input.is_action_pressed("ui_up"): direction = Vector2i.UP
	elif Input.is_action_pressed("ui_down"): direction = Vector2i.DOWN
	elif Input.is_action_pressed("ui_left"): direction = Vector2i.LEFT
	elif Input.is_action_pressed("ui_right"): direction = Vector2i.RIGHT
	var new_placement_cell: Vector2i = player_placement_cell + direction
	if (not get_used_cells(0).has(new_placement_cell) or trees.has(get_cell_atlas_coords(0, new_placement_cell)) or new_placement_cell == ring_placement_cell) and _is_not_out_of_bounds(new_placement_cell):
		player_placement_cell = new_placement_cell
		set_cell(0, previous_cell, 0) # deletes contents of previous cell (atlas_coords = Vector2i(-1, -1))
		set_cell(0, player_placement_cell, 0, PLAYER_SPRITE)
		if player_placement_cell == ring_placement_cell:
			$WinDialog.visible = true
			get_tree().paused = true

# ALGORITHM BEGINS HERE

func _get_noise_type() -> int:
	match noise_type:
		"Perlin": return 3
		"Simplex": return 0
		"Value": return 5
		"Value Cubic": return 4
		_: return 1 # Return Simplex Smooth by default

func set_noise() -> void:
	noise = FastNoiseLite.new()
	noise.frequency = noise_frequency
	noise.noise_type = _get_noise_type() as FastNoiseLite.NoiseType
	noise.fractal_type = fractal_type
	noise.cellular_distance_function = cellular_distance_type
#	noise.fractal_octaves = octaves
	noise.seed = randi()

func paint_tiles() -> void:
	for x in range(x_tile_range):
		for y in range(y_tile_range):
			var noise_point: float = noise.get_noise_2d(x * tile_set.tile_size.x, y * tile_set.tile_size.y)
			if noise_point < tree_cap and not get_used_cells(0).has(Vector2i(x, y)):
				set_cell(0, Vector2i(x, y), 0, trees.pick_random())
			if ((building_cap <= tree_cap and randf() < building_overtakes_tree) or (building_cap > tree_cap and noise_point < building_cap)) and not get_used_cells(0).has(Vector2i(x, y)):
				set_cell(0, Vector2i(x, y), 0, buildings.pick_random())
\end{lstlisting}

\subsection{accept\_dialog.tscn}

\begin{lstlisting}
[gd_scene format=3 uid="uid://cau5jgogdnf53"]

[node name="AcceptDialog" type="AcceptDialog"]
title = "Tree-Munching Time!"
position = Vector2i(326, 100)
size = Vector2i(500, 421)
mouse_passthrough = true
ok_button_text = "Bring it on!"
dialog_text = "You're a hollow Golem who seeks the ultimate treasure; a ring that's got something on top of it. It's somewhere in this large village and barely visible to your naked eyes, but you'll stop at nothing to get what you want. You can chow down every tree and fauna that stands in your way of the ring, but your Achilles heel is any bricks and mortar, which will make you stop at your tracks. Are you ready to attain your treasure?w Golem in a black-and-white world, in search for your most desired treasure. It's a ring with something on top of it. And you'll stop at nothing to get what you want. You can chow down every tree and fauna that stands in your way of the ring, but your Achilles heel is any bricks and mortar, which will make you stop at your tracks. Are you ready to attain the treasure that is rightfully yours?!"
dialog_autowrap = true
\end{lstlisting}

\subsection{win\_dialog.tscn}

\begin{lstlisting}
[gd_scene format=3 uid="uid://b5q8ovcigrvyr"]

[node name="WinDialog" type="ConfirmationDialog"]
title = "Tree-Munching Time!"
position = Vector2i(326, 100)
size = Vector2i(500, 421)
mouse_passthrough = true
ok_button_text = "Get Me a New Village"
dialog_text = "You found your treasure! Well done, you!

Would you like to travel to a new village in the hopes of finding another ring? Or would you like to take your treasure home now?"
dialog_autowrap = true
cancel_button_text = "Get Me Out of Here"
\end{lstlisting}

\subsection{icon.svg.import}

\begin{lstlisting}
[remap]

importer="texture"
type="CompressedTexture2D"
uid="uid://crgf6ascxsdt0"
path="res://.godot/imported/icon.svg-218a8f2b3041327d8a5756f3a245f83b.ctex"
metadata={
"vram_texture": false
}

[deps]

source_file="res://icon.svg"
dest_files=["res://.godot/imported/icon.svg-218a8f2b3041327d8a5756f3a245f83b.ctex"]

[params]

compress/mode=0
compress/high_quality=false
compress/lossy_quality=0.7
compress/hdr_compression=1
compress/normal_map=0
compress/channel_pack=0
mipmaps/generate=false
mipmaps/limit=-1
roughness/mode=0
roughness/src_normal=""
process/fix_alpha_border=true
process/premult_alpha=false
process/normal_map_invert_y=false
process/hdr_as_srgb=false
process/hdr_clamp_exposure=false
process/size_limit=0
detect_3d/compress_to=1
svg/scale=1.0
editor/scale_with_editor_scale=false
editor/convert_colors_with_editor_theme=false
\end{lstlisting}

\subsection{monochrome\textunderscore{}packed.png.import}

\begin{lstlisting}
[remap]

importer="texture"
type="CompressedTexture2D"
uid="uid://m662wwd4prmn"
path="res://.godot/imported/monochrome_packed.png-6b9bd1c64dd50f72acd3afd14d1ac34f.ctex"
metadata={
"vram_texture": false
}

[deps]

source_file="res://monochrome_packed.png"
dest_files=["res://.godot/imported/monochrome_packed.png-6b9bd1c64dd50f72acd3afd14d1ac34f.ctex"]

[params]

compress/mode=0
compress/high_quality=false
compress/lossy_quality=0.7
compress/hdr_compression=1
compress/normal_map=0
compress/channel_pack=0
mipmaps/generate=false
mipmaps/limit=-1
roughness/mode=0
roughness/src_normal=""
process/fix_alpha_border=true
process/premult_alpha=false
process/normal_map_invert_y=false
process/hdr_as_srgb=false
process/hdr_clamp_exposure=false
process/size_limit=0
detect_3d/compress_to=1
\end{lstlisting}

\section{LSystemGrammarDemo}

% \localtableofcontents{}

\subsection{.gitattributes}

\begin{lstlisting}
# Normalize EOL for all files that Git considers text files.
* text=auto eol=lf
\end{lstlisting}

\subsection{.gitignore}

\begin{lstlisting}
# Godot 4+ specific ignores
.godot/
\end{lstlisting}

\subsection{project.godot}

\begin{lstlisting}
; Engine configuration file.
; It's best edited using the editor UI and not directly,
; since the parameters that go here are not all obvious.
;
; Format:
;   [section] ; section goes between []
;   param=value ; assign values to parameters

config_version=5

[application]

config/name="LSystemGrammarDemo"
run/main_scene="res://DemoNode.tscn"
config/features=PackedStringArray("4.0")

[display]

window/stretch/mode="canvas_items"
window/stretch/aspect="expand"

[gui]

common/drop_mouse_on_gui_input_disabled=true

[physics]

common/enable_pause_aware_picking=true
\end{lstlisting}

\subsection{DemoNode.tscn}

\begin{lstlisting}
[gd_scene load_steps=2 format=3 uid="uid://bu380we4od0ln"]

[ext_resource type="Script" path="res://DemoNode.gd" id="1"]

[node name="DemoNode" type="Node"]
script = ExtResource("1")
choices = "deterministic"

[node name="Timer" type="Timer" parent="."]

[node name="TextLabel" type="Label" parent="."]
offset_right = 1152.0
offset_bottom = 23.0
autowrap_mode = 3

[connection signal="timeout" from="Timer" to="." method="_on_Timer_timeout"]
\end{lstlisting}

\subsection{DemoNode.gd}

\begin{lstlisting}
extends Node

# Basic: https://youtu.be/feNVBEPXAcE?t=77 (L = +)
# Choices: http://paulbourke.net/fractals/lsys/
# Deterministic: https://www1.biologie.uni-hamburg.de/b-online/e28_3/lsys.html#D0L-system

@export_enum("basic", "choices", "deterministic") var choices: String = "choices"
@export var axiom: String
@onready var string: String
@onready var timer = $Timer
@onready var label = $TextLabel
@onready var rules: Array[Dictionary]

func set_values():
	if choices == "basic":
		rules = [
			{
				"from": "F",
				"to": "F+F"
			}
		]
		axiom = "F+"
	elif choices == "choices":
		rules = [
			{
				"from": "F",
				"to": "F+F−F−FF+F+F−F"
			}
		]
		axiom = "F+F+F+F"
	elif choices == "deterministic":
		rules = [
			{
				"from": "a",
				"to": "ab"
			},
			{
				"from": "b",
				"to": "a"
			}
		]
		axiom = "b"

func _ready():
	set_values()
	string = axiom
	label.size.x = get_viewport().size.x
	label.text = string
	print(len(string))
	timer.start()

func get_new_replacement(character: String) -> String:
	for rule in rules:
		if rule["from"] == character:
			return rule["to"]
	return ""

func _on_Timer_timeout():
	var new_string = ""
	for character in string:
		new_string += get_new_replacement(character)
	string = new_string
	label.text = string
	print(len(string))
\end{lstlisting}

\subsection{icon.svg.import}

\begin{lstlisting}
[remap]

importer="texture"
type="CompressedTexture2D"
uid="uid://cwnnuqmejj04q"
path="res://.godot/imported/icon.svg-218a8f2b3041327d8a5756f3a245f83b.ctex"
metadata={
"vram_texture": false
}

[deps]

source_file="res://icon.svg"
dest_files=["res://.godot/imported/icon.svg-218a8f2b3041327d8a5756f3a245f83b.ctex"]

[params]

compress/mode=0
compress/high_quality=false
compress/lossy_quality=0.7
compress/hdr_compression=1
compress/normal_map=0
compress/channel_pack=0
mipmaps/generate=false
mipmaps/limit=-1
roughness/mode=0
roughness/src_normal=""
process/fix_alpha_border=true
process/premult_alpha=false
process/normal_map_invert_y=false
process/hdr_as_srgb=false
process/hdr_clamp_exposure=false
process/size_limit=0
detect_3d/compress_to=1
svg/scale=1.0
editor/scale_with_editor_scale=false
editor/convert_colors_with_editor_theme=false
\end{lstlisting}
\end{document}
