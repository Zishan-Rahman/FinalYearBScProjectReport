\chapter{User Guide} \label{Guide}
\section{Instructions}
% You must provide an adequate user guide for your software. The guide should provide easily understood instructions on how to use your software. A particularly useful approach is to treat the user guide as a walk-through of a typical session, or set of sessions, which collectively display all of the features of your package. Technical details of how the package works are rarely required. Keep the guide concise and simple. The extensive use of diagrams, illustrating the package in action, can often be particularly helpful. The user guide is sometimes included as a chapter in the main body of the report, but is often better included in an appendix to the main report.

To run the projects in the .zip file, extract the projects in one folder. Then open Godot 4 (all projects in the source code listings folder are Godot 4 projects, \textbf{not} Godot 3 projects), and, when opening the Godot editor, click "Scan", then go to that folder and select it. The projects can then be opened in the project manager and edited as needed in Godot. When you click on some of the scenes in the projects, there may be some "exported" variables from scripts that are visible to you in the editor (examples include the "Distance" and "Random Starting Points" variables in the Voronoi Cells project). You can hover over the variable names in the editor and it will show a brief description of what the variable correlates to in the script.


