Procedural generation refers to content in a medium that is produced algorithmically in lieu of by hand. Most notably, procedural generation algorithms are implemented in video games, for generating levels, terrain and other game contents programmatically. This project takes some of the more prominent algorithms for procedural generation- Lindenmayer Systems, Voronoï Points, Poisson Disk Generation and Simplex Noise- and implements them in a 2D tile-map-oriented RPG-like game in the open-source Godot game engine, and compares their workings and performance. My aim with this project is to (1) increase my knowledge of procedural generation in games beyond the surface level, by going in-depth into some of the algorithms that are used, and (2) use this knowledge to implement said algorithms in a 2D tiled RPG scenario in Godot, then compare how each algorithm works and performs.